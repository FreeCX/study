\input{../../.preambles/01-semester_work}
\input{../../.preambles/10-russian}
\input{../../.preambles/20-math}
\input{../../.preambles/30-physics}
\begin{document}
\maketitlepage{Факультет электроники и вычислительной техники}{физики}
{Электродинамика}{студент группы Ф-369\\Голубев~А.~В.}{доцент Грецов~М.~В.}{№3}

\newcommand{\grad}{\mathrm{grad}\,}
\renewcommand{\div}{\mathrm{div}\,}

\newpage

\emph{Задача №1.15}: Определить поле, создаваемое заряженным проводящим шаром 
радиуса \( a \). Заряд его \( Q \). Диэлектрическая проницаемость окружающей 
среды \( \eps = \eps(r) \), где \( r \) -- расстояние от центра шара.

\emph{Решение:} 

Для решения задачи воспользуемся сферической системой координат. Учтём, что 
задача имеет сферическую симметрию и запишем следующие соотношения:
\[
	D_r = D(r);\quad
	D_\theta = D_\psi = 0
\]

Электрическая индукция поля может быть записана в следующем виде:
\[
	\div{\vec{D}} = \rho
\]

Воспользуемся теоремой Остроградского-Гаусса, перепишем предыдущее равенство 
в виде:
\[
	\int \div\vec{D} dV = \oint \vec{D}\vec{dS} = \int \rho dV = q
\]

Перепишем предыдущее в виде:
\[
	D \oint\vec{dS} = DS = q
\]

Поле внутри шара (\( r < a, q = 0 \)):
\[
	D = E = 0
\]

Поле вне шара (\( r > r, q = Q \)):
\[
	D = \frac{Q}{r^2}
\]

C учётом \( D = \eps E \), получим:
\[
	E = \frac{Q}{\eps(r) r^2}
\]

\emph{Ответ:} \( D = \cfrac{Q}{r^2},\quad E = \cfrac{Q}{\eps(r) r^2}\)
 
\newpage

\emph{Задача №1.58}: Вычислить ёмкость цилиндрического конденсатора. Длина 
его \( l \), радиусы обкладок \( R_1 \) и \( R_2 \). Между обкладками два 
коаксиальных слоя однородных диэлектриков с проницаемостями \( \eps_1 \) 
и \( \eps_2 \), граница раздела между ними -- цилиндрическая поверхность 
радиуса \( R_0 \). Краевыми эффектами пренебречь.

\emph{Решение:}

\emph{Ответ:}

\newpage

\emph{Задача №1.99}: Точечный заряд \( q \) находится на одинаковом 
расстоянии \( a \) от двух взаимно перпендикулярных заземленных проводящих 
полуплоскостей. Определить создаваемое поле.

\emph{Решение:}

\emph{Ответ:}

\newpage

\emph{Задача №2.35}: Полупространство заполнено однородным магнетиком с 
проницаемостью \( \mu_1 \), а второе полупространство -- однородным 
магнетиком с проницаемостью \( \mu_2 \). В первой среде имеется плоский 
контур \( L \) с током \( I \), расположенный параллельно плоскости 
раздела обеих сред на расстоянии \( h \) от неё. Определить создаваемое 
током магнитное поле.

\emph{Решение:}

\emph{Ответ:}
\end{document}