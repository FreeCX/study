\input{../.preambles/01-semester_work}
\input{../.preambles/10-russian}
\input{../.preambles/20-math}
\input{../.preambles/30-physics}
\begin{document}
\maketitlepage{Факультет электроники и вычислительной техники}{физики}
{Электродинамика}{студент группы Ф-369\\Голубев~А.~В.}
{доцент Грецов~М.~В.}{№1}
\par\textbf{Задача №83:} Заряд электрона распределен в атоме водорода, находящемся в
нормальном состоянии, с плотностью
\[
  \rho(r) = -\frac{e_0}{\pi a^3}e^{-\frac{2r}{a}},
\]
где \( a = 0,529\cdot10^{-8} \)~см -- боровский радиус электрона,
\( e_0 = 4,80\cdot10^{-10} \)~CGSE -- элементарный заряд. Найти потенциал
\( \phi_e \) и напряженность \( E_{er} \) электрического поля электронного заряда,
а также полные потенциал \( \phi \) и напряженность поля \( \vec{E} \) в атоме,
считая, что протонный заряд сосредоточен в начале координат. Построить
приблизительный ход величин \( \phi \) и \( E \).

%\vspace*{2em}
\textbf{Решение:}
    
    Из теоремы Гаусса:
    \[
        E_{er}(r) = \frac{1}{\Ezero r^2} \int\limits_0^r \rho(r'){r'}^2\d r'.
    \]
    
    Потенциал точечного заряда на бесконечности стремится к нулю, поэтому
    \[
        \phi_e(r) = \int\limits_r^\infty \frac{1}{\Ezero r^2}\int\limits_0^r \rho(r')
        {r'}^2\d r' = -\frac{1}{\Ezero}\int\limits_r^\infty \d\left(\frac{1}{r}
        \right)\int\limits_0^r \rho(r'){r'}^2\d r'.
    \]
    
    Взяв интеграл по частям, получим:
    \[
        \phi_e(r) = \frac{1}{\Ezero r}\int\limits_0^r \rho(r'){r'}^2\d r' +
        \frac{1}{\Ezero}\int\limits_r^\infty \rho(r')r'\d r'.
    \]
    
    Подставляя плотность распределения заряда \( \rho \) в формулы для
    \( E_{er}(r) \) и \( \phi_r(r) \), получим:
    \begin{align*}
        & \phi_e(r) = \frac{e_0}{r}\left[e^{-\frac{2r}{a}} - 1\right] +
        \frac{e_0}{a}e^{-\frac{2r}{a}}, \\
        & E_{er}(r) = \frac{e_0}{r^2}\left[\left(1 + \frac{2r}{a}\right)
        e^{-\frac{2r}{a}} - 1\right] + \frac{2e_0}{a^2}e^{-\frac{2r}{a}}.
    \end{align*}
    
    Воспользовавшись принципом суперпозиции, найдем потенциал \( \phi(r) \) и
    напряженность поля \( E(r) \) в атоме:
    \begin{align*}
        & \phi(r) = \phi_e(r) + \phi_\emph{п}(r) = \phi_e(r) + \frac{e_0}{r} =
        e_0e^{-\frac{2r}{a}}\left[\frac{1}{r} + \frac{1}{a}\right], \\
        & E(r) = E_{er}(r) + E_\emph{п}(r) = E_{er}(r) + \frac{e_0}{r^2} =
        e_0e^{-\frac{2r}{a}}\left[\frac{1}{r^2} + \frac{2}{ar} + \frac{2}{a^2}
        \right].
    \end{align*}

    \begin{figure}[h!]
        \center
        \includegraphics[width=.47\textwidth]{field}\hfill
        \includegraphics[width=.47\textwidth]{potential}
        \parbox{.47\textwidth}{\centering Зависимость \( E(r) \)}\hfill
        \parbox{.47\textwidth}{\centering Зависимость \( \phi(r) \)}
    \end{figure}

\vspace*{2em}        
\emph{Ответ:}
    \begin{align*}
        & \phi_e(r) = \frac{e_0}{r}\left[e^{-\frac{2r}{a}} - 1\right] +
        \frac{e_0}{a}e^{-\frac{2r}{a}}, \\
        & E_{er}(r) = \frac{e_0}{r^2}\left[\left(1 + \frac{2r}{a}\right)
        e^{-\frac{2r}{a}} - 1\right] + \frac{2e_0}{a^2}e^{-\frac{2r}{a}}; \\
        & \phi(r) = e_0e^{-\frac{2r}{a}}\left[\frac{1}{r} + \frac{1}{a}\right], \\
        & E(r) = e_0e^{-\frac{2r}{a}}\left[\frac{1}{r^2} + \frac{2}{ar} +
        \frac{2}{a^2}\right].
    \end{align*}
\newpage

%-------------------------------------------------------------------------------

\par\textbf{Задача №248}:  Определить магнитное поле \( \vec{H} \) 
в цилиндрической полости, вырезанной в бесконечно длинном цилиндрическом 
проводнике. Радиусы полостии проводника соответственно \( a \) и \( b \), 
расстояние между их параллельными осями \( d \) \( ( b > a + d) \). 
Ток \( I \) распределен равномерно по сечению. \\

\textbf{Решение:}
\par Магнитное поле внутри сплошного цилиндра с постоянной плотностью тока в точке
\( \vec{r} \) равно (по теореме Стокса): \\
\[ \vec{H} = \frac{2\pi}{c}\vec{I}\times\vec{r}\]
Используя принцип суперпозиции и считая что отверстие -- это пространство,
через которое идут два тока \( \vec{I} \) и \( -\vec{I} \). Тогда в этой
цилиндрической области 
\[ 
    \vec{H} = \frac{2\pi}{c}(\vec{I}\times\vec{r} - \vec{I}\times\vec{r'}) =
    \frac{2\pi}{c}\vec{I}\times(\vec{r}-\vec{r'})
\]
Учитывая, что \( \vec{d}=\vec{r}-\vec{r'} \), получим
\[ \vec{H} = \frac{2\pi}{c}\vec{I}\times\vec{d} \]
\newpage

%-------------------------------------------------------------------------------

\par\textbf{Задача №549}: <<Поезд>> \( A'B' \), длина которого \( l_0 = 8.64\cdot10^8 \) км в системе,
где он покоится, идет со скоростью \( V = 240000 \) км/сек мимо <<платформы>>, имеющей
такую же длину в своей системе покоя. В голове \( B' \) и хвосте \( A' \) <<поезда>> 
имеются одинаковыечасы, синхронизованные между собой. Такие же часы установлены в 
начале \( A \) и в конце \( B \) <<платформы>>. В тот момент, когда голова <<поезда>> 
поравнялась с началом <<платформы>>, совпадающие часы показывали 12 час 00 мин.
\parОтветить на следующие вопросы:
\begin{itemize}\itemsep-8pt
    \item[а)] можно ли утверждать, что в этот момент в какой-либо системе отсчёта все
            часы показывают 12 час 00 мин;
    \item[б)] сколько показывают каждые из часов в момент, когда хвост <<поезда>>
            поравнялся с началом <<платформы>>;
    \item[в)] сколько показывают часы в момент, когда голова <<поезда>> поравнялась
            с концом <<платформы>>?
\end{itemize}
\textbf{Решение:}
\begin{itemize}\itemsep-8pt
    \item[а)] Нельзя. 12 часов 00 минут могут показывать одновременно двое часов
            в одной из систем отсчёта и только одни часы в другой системе отсчёта.
    \item[б)] Показания пространственно совпадающих часов не зависят от выбора 
            системы отсчёта: 
            \[ t_{A'} = 12 \text{ часов } 00 \text{ минут } 
               + \frac{l_0}{V} = 13 \text{ часов } 00 \text{ минут}; \]
            \[ t_{A} = 12 \text{ часов } 00 \text{ минут } + \frac{l_0}{V}
            \sqrt{1-\frac{V^2}{c^2}} = 12 \text{ часов } 36 \text{ минут}. \]
            Показания оставшихся часов \( B \) и \( B' \) будут зависеть от выбора
            системы отсчёта вследствие относительности одновременности.
            % insert image %
            С точки зрения наблюдателя на <<на платформе>>
            \[ t_{B'} = 12 \text{ часов } 21.6 \text{ минут } \]
            \[ t_{B} = t_{A} = 12 \text{ часов } 36 \text{ минут } \]
            С точки зрения наблюдателя в <<поезде>>
            \[ t_{B'} = t_{A'} = 13 \text{ часов } 00 \text{ минут } \]
            \[ t_{B} = 13 \text{ часов } 14.4 \text{ минут } \]
    \item[в)] С точки зрения наблюдателя на <<платформе>>:
            \[ t_A = t_B = 13 \text{ часов } 00 \text{ минут } \]
            \[ t_{B'} = 12 \text{ часов } 36 \text{ минут } \]
            \[ t_{A'} = 13 \text{ часов } 14.4 \text{ минут } \]
            С точки зрения наблюдателя в <<поезде>>:
            \[ t_A = t_B = 12 \text{ часов } 21.6 \text{ минут } \]
            \[ t_{A'} = t_{B'} = 12 \text{ часов } 36 \text{ минут } \]
            \[ t_{B'} = 13 \text{ часов } 00 \text{ минут } \]
            Во всех остальных случаях отстают те часы, показания которых 
            приходится сравнивать с показаниями двух часов в другой системе отсчёта.
\end{itemize}
\end{document}
