\section{Теория Бора.}

% 5.1 --------------
\begin{enumerate}
\item Оценить время, за которое электрон, движущийся вокруг ядра водорода по
орбите радиусом \( 0,\!5 \cdot 10^{-8} \)~см, упал бы на ядро, если бы он терял
энергию на излучение в соответствии с классической теорией:
\[
    \der{E}{t} = -k\frac{2e^2}{3c^3}a^2,
\]
где \( a \) -- ускорение электрона. Считать, что вектор \( \vec{a} \) все время
направлен к центру атома.

\emph{Решение:}

\newpage

% 5.2 --------------
\item Какие спектральные линии появятся при возбуждении атомарного водорода
электронами с энергией в 12,\!5~эВ?

\emph{Решение:}

\newpage

% 5.3 --------------
\item Вычислить энергию, которую надо сообщить атому водорода, чтобы его серия
Бальмера содержала только одну спектральную линию.

\emph{Решение:}

\newpage

% 5.4 --------------
\item Какие спектральные линии появятся в спектре атомарного водорода при
облучении его ультрафиолетовым светом с длиной волны 100~нм?

\emph{Решение:}

\newpage

% 5.5 --------------
\item Первоначально неподвижный атом водорода испустил фотон с частотой,
соответствующей головной линии серии Лаймана. Найти скорость \( v \) атома после
излучения фотона.

\emph{Решение:}

\newpage

% 5.6 --------------
\item Определить наименьшую энергию, которую надо сообщить в основном состоянии
трижды ионизованному атому бериллия, чтобы возбудить полный спектр этого атома.

\emph{Решение:}

\newpage

% 5.7 --------------
\item Фотон головной серии Лаймана иона гелия \( \mathrm{He}^+ \) поглощается
водородным атомом в основном состоянии и ионизует его. Определить кинетическую
энергию \( \eps \), которую получит электрон при такой ионизации.

\emph{Решение:}

\newpage

% 5.8 --------------
\item В спектрах некоторых звезд наблюдается \( m \approx 30 \) линий водородной
серии Бальмера. При каком наименьшем числе \( N \) штрихов дифракционной решетки
можно разрешить эти линии в спектре первого порядка?

\emph{Решение:}

\newpage

% 5.9 --------------
\item На сколько вольт ионизационный потенциал дейтерия \( (D)\) больше
ионизационного потенциала водорода \( (H) \)? Выразить разность между энергиями
ионизации \( D \) и \( H \) в джоулях на моль.

\emph{Решение:}

\newpage

% 5.10 -------------
\item Определить квантовое число \( n \) возбужденного состояния атома водорода,
если известно, что при переходе в основное состояние атом излучил:
\begin{enumerate}
    \item фотон с длиной волны \( \lambda = 97,\!25 \)~нм;
    \item два фотона, с \(\lambda_1 = 656,\!3\)~нм и \(\lambda_2 = 121,\!6\)~нм.
\end{enumerate}

\emph{Решение:}

\newpage

% 5.11 -------------
\item У какого водородоподобного иона разность длин волн головных линий серии
Бальмера и Лаймана равна 59,\!3~нм?

\emph{Решение:}

\newpage

% 5.12 -------------
\item Вычислить для мезоатома водорода (в котором вместо электрона движется
мезон, имеющий тот же заряд, но массу в 207 раз больше):
\begin{enumerate}
    \item расстояние между мезоном и ядром в основном состоянии;
    \item длину волны резонансной линии;
    \item энергии связи основных состояний мезоатомов, ядра которых протон и дейтрон.
\end{enumerate}

\emph{Решение:}

\newpage

% 5.13 -------------
\item Найти для позитрония (система из электрона и позитрона, вращающаяся вокруг
центра инерции):
\begin{enumerate}
    \item расстояние между частицами в основном состоянии;
    \item ионизационный потенциал и первый потенциал возбуждения;
    \item постоянную Ридберга и длину волны резонансной линии.
\end{enumerate}

\emph{Решение:}

\newpage

% 5.14 -------------
\item Вычислить отношение массы протона к массе электрона, если известно, что
отношение постоянных Ридберга тяжелого и легкого водорода \(\eta = 1,\!000272\),
а отношение масс ядер \( n = 2,\!00 \).

\emph{Решение:}

\newpage

% 5.15 -------------
\item Энергия связи электрона в атоме He равно \( E_0 = 24,\!6 \)~эВ. Найти
энергию, необходимую для удаления обоих электронов из этого атома.

\emph{Решение:}

\newpage

% 5.16 -------------
\item Атом водорода, двигающийся со скоростью \( v_0 = 3,\!26 \)~м/с, испустил
фотон, соответствующий переходу из первого возбужденного состояния в основное.
Найти угол \( \phi \) между направлением движения атома, если кинетическая
энергия атома осталась прежней.

\emph{Решение:}

\newpage

% 5.17 -------------
\item При наблюдении излучения пучка возбужденных атомов водорода под углом
\( \theta = 45^\circ \) к направлению их движения длина волны резонансной линии
оказалась смещенной на \( \Delta\lambda = 0,\!20 \)~нм. Найти скорость атомов
водорода.

\emph{Решение:}

\end{enumerate}
