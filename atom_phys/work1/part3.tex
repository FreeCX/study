\section{Корпускулярные свойства электромагнитного излучения.}
\begin{enumerate}

% 3.1 --------------
\item Короткий импульс света энергией \( E = 7,\!5 \)~Дж падает узким пучком на
зеркальную пластинку с коэффициентом отражения \( \rho = 0,\!60 \). Угол падения
\( \theta = 30^\circ \). Найти импульс, переданный пластинке.

\emph{Решение:}

\newpage

% 3.2 --------------
\item Найти с помощью корпускулярных представлений силу светового давления,
которую оказывает плоский световой поток с интенсивностью
\( I = 1,\!0\text{~Вт/см}^2 \) на плоскую зеркальную поверхность,если угол
падения \( \theta = 30^\circ \) и площадь освещаемой поверхности
\( S = 10\text{~см}^2 \).

\emph{Решение:}

\newpage

% 3.3 --------------
\item Плоский световой поток интенсивностью \( I \text{~Вт/см}^2 \), освещает
одну половину шара с зеркальной поверхностью. Радиус шара \( R \). Найти с
помощью корпускулярных представлений силу светового давления, испытываемую
шаром.

\emph{Решение:}

\newpage

% 3.4 --------------
\item Над центром круглой абсолютно зеркальной пластинки радиусом \( R \)
находится точечный источник света мощностью \( P \). Расстояние между источником
и пластинкой \( l \). Найти с помощью корпускулярных представлений силу
светового давления, которую испытывает пластинка. Рассмотреть также случаи
\( R \ll l \) и \( R \gg l \).

\emph{Решение:}

\newpage

% 3.5 --------------
\item Найти длину волны коротковолновой границы сплошного рентгеновского
спектра, если известно ,что после увеличения напряжения на рентгеновской трубке
в \( \eta = 2,\!0 \)~раза эта длина волны изменилась на
\( \Delta\lambda = 50 \)~пм.

\emph{Решение:}

\newpage

% 3.6 --------------
\item Вычислить скорость электронов, подлетающих к антикатоду рентгеновской
трубки, если длина волны коротковолновой границы сплошного рентгеновского
спектра \( \lambda_{min} = 157 \)~пм.

\emph{Решение:}

\newpage

% 3.7 --------------
\item Найти работу выхода с поверхности некоторого металла, если при поочерёдном
освещении его электромагнитным излучением с длинами волн \( \lambda_1 =
0,\!35 \)~мкм и \( \lambda_2 = 0,\!54 \)~мкм максимальные скорости
фотоэлектронов отличаются в \( \eta = 2,\!0 \)~раза.

\emph{Решение:}

\newpage

% 3.8 --------------
\item Медный шарик, отделённый от других тел, облучают электромагнитным
излучением с длиной волны \( \lambda = 0,\!200 \)~мкм. До какого максимального
потенциала зарядится шарик?

\emph{Решение:}

\newpage

% 3.9 --------------
\item Фотон с длиной волны \( \lambda = 17,\!0 \)~пм вырывает из покоящегося
атома электрон, энергия связи которого \( E = 69,\!3 \)~кэВ. Найти импульс,
переданный атому в результате этого процесса, если электрон вылетел под прямым
углом к направлению налетающего фотона.

\emph{Решение:}

\newpage

% 3.10 -------------
\item Показать, что свободный электрон не может излучить световой квант, так как
если предположить, что электрон излучает световой квант, то не будут выполняться
одновременно закон сохранения импульса и закон сохранения энергии.

\emph{Решение:}

\newpage

% 3.11 -------------
\item Фотон с длиной волны \( \lambda = 3,\!64 \)~пм рассеялся на покоящемся
свободном электроне так, что кинетическая энергия электрона отдачи составила
\( \eta = 25\% \) от энергии налетевшего фотона. Найти:
\begin{enumerate}
    \item комптоновское смещение длины волны рассеянного фотона;
    \item угол \( \theta \), под которым рассеялся фотон.
\end{enumerate}

\emph{Решение:}

\newpage

% 3.12 -------------
\item Фотон испытал рассеяние на покоящемся свободном электроне. Найти импульс
налетевшего фотона, если энергия рассеянного фотона равна кинетической энергии
электрона отдачи при угле \( 90^\circ \) между направлениями их разлета.

\emph{Решение:}

\newpage

% 3.13 -------------
\item В результате столкновения фотона с покоящимся свободным электроном углы,
под которыми рассеялся фотон и отлетел электрон отдачи, оказались одинаковыми и
угол между направлениями их движения \( \theta = 100^\circ \). Найти длину волны
фотона до столкновения.

\emph{Решение:}

\newpage

% 3.14 -------------
\item Фотон с энергией \( \hbar\omega \) испытал столкновение с электроном,
который двигался ему на встречу. В результате столкновения направление движения
фотона изменилось на противоположное, а его энергия осталась прежней. Найти
скорость электрона до и после столкновения.

\emph{Решение:}

\newpage

% 3.15 -------------
\item Средняя длина волны излучения лампочки накаливания с металлической
спиралью равна 1200~нм. Найти число фотонов, испускаемых 200-ваттной
лампочкой в единицу времени.

\emph{Решение:}

\newpage

% 3.16 -------------
\item Во сколько раз изменение длины волны фотона при комптоновском рассеянии на
свободном электроне превосходит аналогичное изменение при рассеянии на свободном
протоне при одинаковых углах рассеяния?

\emph{Решение:}

\newpage

% 3.17 -------------
\item Фотон с длиной волны \( \lambda = 0,\!0024 \)~нм после рассеяния на
электроне движется в прямо противоположном направлении. С какой скоростью
\( v \) должен двигаться электрон, чтобы частота фотона при рассеянии не
изменилась?

\emph{Решение:}

\end{enumerate}
