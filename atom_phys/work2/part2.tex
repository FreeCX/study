
    \emph{2.1: Протон с длиной волны \( \lambda = 1,7 \) пм упруго
    	рассеялся под углом \( 90^{\circ} \) на первоначально покоившейся 
    	частице, масса которой в \( n = 4,0 \) раза больше массы протона. 
    	Определить длину волы рассеянного протона.}

        Закон сохранения энергии: 
        \[
            \frac{mv^2}{2} = \frac{mv'^2}{2} + \frac{m_2u^2}{2}
        \]
        т.к. \( m_2 = nm \), то
        \[ v^2 = v'^2 + nu^2 \]
        По теореме Пифагора: \( (mv)^2 + (mv')^2 = (m_2u)^2 \)
        \[ v^2 + v'^2 = (nu)^2 \]
        \begin{equation*}
            \left\{\begin{aligned}
                v^2 + v'^2 = (nu)^2 \\
                v^2 = v'^2 + (nu)^2
            \end{aligned}\right.
        \end{equation*}
        Решая систему уравнение получаем:
        \[ v' = v\sqrt\frac{n-1}{n+1} \]
        С учётом соотношения:
        \[ p = \hbar k = \hbar\frac{2\pi}{\lambda} = mv \]
        Получим: 
        \[ \lambda' = \lambda\sqrt\frac{n+1}{n-1} \]
    
    \emph{2.2: Нейтрон с кинетической энергией \( T = 0,25 \) эВ 
        испытал упругое соударение с первоначально покившемся ядром атома 
        \( ^4He \). Найти длину волн обеих частиц в их Ц-системе до и 
        после соударения.}\\

    \emph{2.3: Два атома, \( ^1H \) и \( ^4He \), движутся в одном 
        направлении, причём дейбройлевская длина волны каждого атома 
        \( \lambda = 60 \) пм. Найти длины волн обоих атомов в их 
        Ц-системе.}\\

    \emph{2.4: Найти кинетическую энергию электронов, падающих нормально 
        на диафрагму с двумя узкими щелями, если на экране, отстоящем от 
        диафрагмы на \( l = 75 \) см, расстояние между соседними 
        максимумами \( \Delta x = 7,5 \). Расстояние между щелями 
        \( d = 25 \) мкм.}\\

	\emph{2.5: Узкий пучок моноэнергетических электронов падает под углом
		скольжения \( \theta = 30^\circ \) на грань монокристалла алюминия. 
		Расстояние между соседними кристаллическими плоскостями, 
		параллельными этой грани монокристалла, \( d = 0,20 \) нм. 
		При ускоряющем напряжении \( U_0 \) наблюдали
		максимум зеркального отражения. Найти \( U_0 \), если следующий 
		максимум зеркально отражения возникал при увеличении ускоряющего 
		напряжения в \( \eta = 2,25 \) раза.} \\

	\emph{2.6: Узкий пучок электронов с кинетической энергией \( K = 10 \) 
		кэВ проходит через поликристаллическую алюминиевую фольгу, 
		образуя на экране систему дифракционных колец. 
		Вычислить межплоскостное расстояние, соответствующее отражение 
		третьего порядка от некоторой системы кристаллических плоскостей, 
		если ему отвечает дифракционное кольцо диаметра \( D = 3,20 \) см. 
		Расстояние между экраном и фольгой \( l = 10,0 \) см.}

        \[ \Delta = d\sin\varphi \]
        \[ 
            \Delta = n\lambda = \frac{2n\pi\hbar}{p} = 
            \frac{2n\pi\hbar}{\sqrt{2mT}} 
        \]
        \[ d\sin\varphi = \frac{2n\pi\hbar}{\sqrt{2mT}} \]
        \[ d = \frac{2n\pi\hbar}{\sqrt{2mT}\sin\varphi} \]

	\emph{2.7: Интерпретировать квантовые условия Бора на основе волновых
		представлений: показать, что электрон в атоме водорода может 
		двигаться только по тем круговым орбитам, на которых укладывается 
		целое число дебройлевских волн.}

        \[ 
            \lambda = \frac{h}{p};\quad 
            p = mv = \frac{h}{p\lambda}
        \]
        \[ mv(2\pi R) = \frac{h}{p\lambda}(2\pi R) \]
        \[ mv(2\pi R) = hn \]
        Условие квантования: \( pRmv = (nh)/(2\pi) \)
        \[
            \frac{h}{p\lambda}(2\pi R) = nh;\quad
            2\pi R = n\lambda;\quad
            R = \frac{n\lambda}{2\pi}
        \]

	\emph{2.8: Убедиться, что измерения координаты частицы с помощью 
		микроскопа вносит неопределенность в её импульс \( \Delta p_x \), 
		такую, что \[ \Delta x\cdot\Delta p_x \geq h \] Иметь в виду, 
		что разрешение микроскопа \( d = \lambda/\sin\theta \), 
		где \( \lambda \) - длина волны используемого света.}

        У фотона, рассеянного на микрочастице и прошедшего через 
        объектив O, проекция импульса \( p_x \) не превышает, как 
        видно из рисунка :), значение 
        \( p\sin\theta = \hbar k\sin\theta \), где 
        \( k = 2\pi/\lambda \). Эта величина характеризует и 
        неопределенность \( \Delta p_x \) фотона. Но при рассеянии 
        фотона на микрочастице последняя испытывает отдачу, в 
        результате чего её импульс получит такую же неопределенность 
        \( \Delta p_x \), как и фотон: 
        \( \Delta p_x \approx \hbar k\sin\theta \). \\
        Имея, кроме того, в виду, что неопределенность координаты 
        \( x \) микрочастицы \( \Delta x \approx d = \lambda/\sin\theta \) 
        получим в результате: 
        \[ 
            \Delta x\cdot\Delta p_x \approx \frac{\lambda}{\sin\theta}
            \frac{2\pi\hbar}{\lambda}\sin\theta = 2\pi\hbar
        \]
        в чём и следовало убедиться.\\

    \emph{2.9: Плоский поток частиц падает нормально на диафрагму 
       	с двумя узкими щелями, образуя на экране дифракционную 
       	картину. Показать, что попытка определить, через какую 
       	щель прошла та или иная частица (например, с помощью 
       	введения индикатора И), приводит к разрушению дифракционной 
       	картины. Для простоты считать углы дифракции малыми.}\\

    \emph{2.10: Оценить минимально возможную энергию электронов 
       	в атоме \( Не \) и соответствующее расстояние электронов 
       	от ядра.}\\

    \emph{2.11: Оценить с помощью соотношения неопределенностей 
       	неопределенность скорости электрона в атоме водорода, 
       	полагая размер атома \( l = 0,10 \) нм. Сравнить полученную 
       	величину со скоростью электрона на первой боровской 
       	орбите данного атома.}\\

    \emph{2.12: Оценить с помощью соотношения неопределенностей 
       	минимальную кинетическую энергию электрона, локализованного 
       	в области размером \( l = 0,20 \) нм.}\\

    \emph{2.13: Электрон с кинетической энергией \( K \approx 4 \) эВ 
       	локализован в области размером \( l \approx 1 \) мкм. 
        Оценить с помощью соотношения неопределенностей 
        относительную еопределенность его скорости.}\\

    \emph{2.14: Электрон находится в одномерной прямоугольной 
      	потенциальной яме с бесконечно высокими стенками. 
       	Ширина ямы \( l \). Оценить с помощью соотношения 
       	неопределенностей силу давления электрона на стенки этой
		ямы при минимально возможной его энергии.}\\

	\emph{2.15: След пучка электронов на экране электронно-лучевой 
		трубки имеет диаметр \( d \approx 0,5 \) мм. Расстояние от 
		электронной пушки до экрана \( l \approx 20 \) см, 
		ускоряющее напряжение \( U = 10 \) кВ. Оценить с помощью 
		соотношения неопределенность координаты электрона на экране.}\\

	\emph{2.16.Частица массы m движется в одномерном потенциальном 
		поле \( U = \cfrac{\chi x^2}{2} \) (гармонический осциллятор). 
		Оценить с помощью соотношения неопределенностей минимально 
		возможную энергию частицы в таком поле.}\\

	\emph{2.17: Параллельный пучок атомов водорода со скоростью 
		\( \nu = 600 \) м/с падает нормально на узкую щель, за которой 
		на расстоянии \( l = 1,0 \) м расположен экран. Оценить с помощью 
		соотношения неопределенностей ширину \( b \) щели, при которой 
		ширина изображения её на экране будет минимальной.}\\
