\input{../../university/.preambles/01-semester_work}
\input{../../university/.preambles/10-russian}
\input{../../university/.preambles/20-math}
\input{../../university/.preambles/30-physics}
\begin{document}

	Последнее обновление: %%%date%%%
    \\

	\emph{ 1.1: Как изменится полная волновая функция \( \Psi(x,t) \),
        описывающая стационарные состояния, если изменить начало отсчёта
        потенциальной энергии на некоторую величину \( \Delta U \)? }

		Измениться лишь временной множитель полной волновой функции. 
		А так как физический смысл имеет лишь квадрат модуля этой функции,
		то изменение временного множителя никак не проявляется. \\
	
    \emph{1.2: Объясните с позиции квантовой теории: почему невозможно 
    	состояние, в котором частица находилась бы в состоянии покоя?}

        Состояние, в котором частица находится в полном покое невозможно.
        В макроскопической физике импульс частицы определяется формулой
        \( p = mv \). Для нахождения скорости \( v \) измеряют 
        координаты частицы \( x_1 \) и \( x_2 \) в два близких момента
        времени \( t_1 \) и \( t_2 \), затем находят частное
        \( (x_2 - x_1)/(t_2 - t_1) \) и выполняют предельный переход
        \( t_2 \rightarrow t_1 \). Такой метод не годится для частицы, 
        т.к. предельный переход требует точного измерения \( x_1 \) и 
        \( x_2 \), т.е. измерение координаты существенно меняет импульс
        частицы. \\

    \emph{1.3: Имеет ли смысл делить полную энергию на кинетическую и 
        потенциальную с позиции квантовой теории?}

        В квантовой механике теряет смысл деление полной энергии \( E \)
        на кинетическую и потенциальную, т.к. одна из этих величин
        зависит от импульса, а другая от координаты. Эти переменные не 
        могут иметь одновременно опредлённые значения. \\

    \emph{1.4: Объясните, почему электрон не падает на ядро с позиции 
    	квантовой теории.}

        Бор предположил модель атома, похожую на модель Резерфорда, но 
        с тем отличием, что электроны располагались вокруг ядра 
        на строго определенных, постоянных орбитах. Эта модель 
        напоминает устройство солнечной системы, где электроны вращаются 
        вокруг ядра так же, как планеты вокруг Солнца. \\\\\\\\

    \emph{1.5: Поясните причину возникновения <<тунельного эффекта>>.}

        Волновая функция \( \psi \) есть величина вспомогательная: 
        все реально наблюдаемые величины связаны с ней вероятностными
        соотношениями. Поскольку функция \( \psi \) всюду отлична от нуля, 
        существует конечная вероятность обнаружения частицы как внутри 
        барьера, так и за его пределами. \\

    \emph{1.6: Объясните основыне отличия между квантовым и классическим 
        гармоническим осциллятором.}
        
        Квантовый осциллятор имеет дискретный спектр, а классический -- 
        сплошной. \\

    \emph{1.7: Может ли частица находится на дне потенциальной ямы? 
    	Определяется ли это формой ямы?}

        В следствии принципа неопределенности Гейзенберга: 
        \[ \Delta x \cdot \Delta p_x \ge \hbar \]
        Заметим, что минимальная энергия, которой может обладать 
        частица в яме:
        \[ E_{min} = \frac{\pi^2\hbar^2}{2ml^2} \neq 0 \]
        Если \( E = 0 \Rightarrow p = 0 \), то 
        \( \Delta x \rightarrow \infty \). Частица не может находится
        на дне <<потенциальной ямы>>, причём этот вывод не зависит 
        от формы ямы. В самом деле, падение на дно ямы связано с 
        обращением в ноль импульса частицы. Тогда неопрделенность 
        координаты становится сколько угодно большой, что противоречит 
        соотношению Гейзенберга. \\
    
    \emph{1.8: В чём состояла ценность опытов Штерна и Герлаха?}

        Наличие у атомов магнитных моментов и их квантование было 
        доказано прямыми опытами Штерна и Герлаха. \\

    \emph{1.9: Дайте наглядное истолкование спин-орбитальному 
    	взаимодействию.}
        
        Воспользуемся теорие Бора для атома водорода: электрона, 
        вращающийся по орбите, обладает спином и тем самым обладает 
        спиновым магнитным моментом. Электрическое поле ядра оказывает 
        воздействие на спиновый магнитный момент, в этом легко 
        убедится, если перейти в систему отсчёта электрон покоится, а 
        ядро -- движется, создавая магнитное поле, которое будет 
        взаимодействовать со спиновым магнитным моментом. \\

    \emph{1.10: Почему расщепление дублетов резкой серии в спектрах 
    	щелочных металлов одинаково для всех линий, а главной серии -- 
        неодинаково?}
        
        Тонка структура уровней и спектральных щелочных металлов в 
        основном обусловлена спин-орбитальным взаимодействием. Главная 
        серия возникает в результате переходов на наиболее глубокий 
        уровень S с вышележащих P-уровенй. Уровень S простой, а все 
        P-уровни двойные, причём расстояние между компонентами этих 
        уровней убывает с возрастанием главного квантового числа n. 
        Поэтому и сами спектральные линии главной серии получаются 
        двойными -- дублетами. Линии резкой серии возникают в 
        результате переходов с простых S-уровенй на лежащий ниже 
        двойной P-уровень, состоящий из подуровней. Поэтому расстояния 
        между компонентами дублетов одни и те же для всей серии, причём 
        сами компоненты являются резкими линиями.\\

    \emph{1.11: Почему количество линий, наблюдаемых при нормально эффекте 
        Зеемана различно при наблюдении вдоль и перпендикулярно 
        магнитному полю?}

        Эффект Зеемана -- это расщепление энергетических уровней под 
        действием магнитного поля. При наложении магнитного поля 
        движение электрона становится сложным, а также будет сложным 
        спектре излучения. Его можно представить как совокупность трёх 
        монохроматических волн разной частоты 
        \( (\nu_0 - \Delta \nu), \nu_0, (\nu_0 + \Delta \nu) \) 
        в разных состояниях поляризации. При наблюдении в магнитном 
        поле в направлении, перпендикулярном полно (поперечный эффект 
        Зеемана) в спектрах излучения и поглощения обнаруживает 
        триплет: три линейно поляризованныне спектральные линии: 
        несмещенную линии первоначальной частоты с электрическим вектором 
        \( \vec{E} \), направленным вдоль \( \vec{B} \), и две 
        смещенные линии с частотами \( (\nu_0 - \Delta \nu) \) и 
        \( \nu_0 + \Delta \nu \) и электрическим вектором \( \perp \) 
        магнитному полю.\\
        
        При наблюдении вдоль магнитного поля (продольный эффект Зеемана) 
        в спектрах обнаруживается дублет -- две симметрично смещенные 
        спектральные линии с частотами 
        \( (\nu_0 - \Delta \nu) \) и \( (\nu_0 + \Delta \nu) \). 
        Обе линии оказываются поляризованными по кругу. Линия с 
        \( (\nu_0 - \Delta \nu) \) поляризована по левому кругу, а
        \( (\nu_0 + \Delta \nu) \) по правому. При продольном наблдюдении 
        несмещенная спектральная компонента отсутствует. 
   
    \emph{1.12: Как будет видоизменяться спектр поглощения рентгеновского 
        излучения веществом при уменьшении энергии излучения?}
        
        При высоких \( E \) возбуждены все серии, при уменьшении энергии 
        происходит прекращение возбуждения K-серии. Появлятся край 
        полосы поглощения, при дальнейшем уменьшении энергии на кривой 
        поглощения появляется L-край.\\

    \emph{1.13: Почему L-край полосы поглощения рентгеноского излучения 
    	веществом состоит из 3-ёх <<зубчиков>>?}
        
        Появление зубчиков связано с тонкой структурой рентгеновских 
        спектров.\\

    \emph{1.14: Объясните, почему при комнатных температурах интенсивность 
        красных спутников заметно выше, чем фиолетовых в явлении 
        комбинационного рассеяния света.}
        
        При обычных температурах, согласно распределению Больцмана, 
        число молекул в возбужденном состоянии значительно меньше, 
        чем в основном. Поэтому в основном будут происходить процессы 
        поглощения энергии. Значит интенсивность красных спутников будет 
        больше.\\

    \emph{1.15: Можно ли наблюдать для молекулы водорода вращетаельный и 
        колебательно-вращательный спектр.}
        
        Нет, т.к. вращательный и колебательно-вращательный спектры 
        наблюдаются на опыте только для несимметричных молекул.\\

    \emph{1.16: Почему при электронных переходах в молекулах меняется 
        колебательный и вращательный характер движения?}

        При электронном переходе изменяется электронная конфигурация 
        оболочки, следовательно, изменяются силы, действующие между 
        ядрами. Следовательно меняются и колебательный и вращательный 
        характер движерия. Т.е. при электронном переходе меняются все 
        три составляющие энергии.\\

    \emph{1.17: Дайте наглядное истолкование принципу Франка-Кондона.}

        Электронных переход, происходящий наиболее вероятно без изменения 
        положения ядер в молекуле.\\

    \emph{2.1: Протон с длиной волны \( \lambda = 1,7 \) пм упруго
    	рассеялся под углом \( 90^{\circ} \) на первоначально покоившейся 
    	частице, масса которой в \( n = 4,0 \) раза больше массы протона. 
    	Определить длину волы рассеянного протона.}

        Закон сохранения энергии: 
        \[
            \frac{mv^2}{2} = \frac{mv'^2}{2} + \frac{m_2u^2}{2}
        \]
        т.к. \( m_2 = nm \), то
        \[ v^2 = v'^2 + nu^2 \]
        По теореме Пифагора: \( (mv)^2 + (mv')^2 = (m_2u)^2 \)
        \[ v^2 + v'^2 = (nu)^2 \]
        \begin{equation*}
            \left\{\begin{aligned}
                v^2 + v'^2 = (nu)^2 \\
                v^2 = v'^2 + (nu)^2
            \end{aligned}\right.
        \end{equation*}
        Решая систему уравнение получаем:
        \[ v' = v\sqrt\frac{n-1}{n+1} \]
        С учётом соотношения:
        \[ p = \hbar k = \hbar\frac{2\pi}{\lambda} = mv \]
        Получим: 
        \[ \lambda' = \lambda\sqrt\frac{n+1}{n-1} \]
    
    \emph{2.2: Нейтрон с кинетической энергией \( T = 0,25 \) эВ 
        испытал упругое соударение с первоначально покившемся ядром атома 
        \( ^4He \). Найти длину волн обеих частиц в их Ц-системе до и 
        после соударения.}\\

    \emph{2.3: Два атома, \( ^1H \) и \( ^4He \), движутся в одном 
        направлении, причём дейбройлевская длина волны каждого атома 
        \( \lambda = 60 \) пм. Найти длины волн обоих атомов в их 
        Ц-системе.}\\

    \emph{2.4: Найти кинетическую энергию электронов, падающих нормально 
        на диафрагму с двумя узкими щелями, если на экране, отстоящем от 
        диафрагмы на \( l = 75 \) см, расстояние между соседними 
        максимумами \( \Delta x = 7,5 \). Расстояние между щелями 
        \( d = 25 \) мкм.}\\

	\emph{2.5: Узкий пучок моноэнергетических электронов падает под углом
		скольжения \( \theta = 30^\circ \) на грань монокристалла алюминия. 
		Расстояние между соседними кристаллическими плоскостями, 
		параллельными этой грани монокристалла, \( d = 0,20 \) нм. 
		При ускоряющем напряжении \( U_0 \) наблюдали
		максимум зеркального отражения. Найти \( U_0 \), если следующий 
		максимум зеркально отражения возникал при увеличении ускоряющего 
		напряжения в \( \eta = 2,25 \) раза.} \\

	\emph{2.6: Узкий пучок электронов с кинетической энергией \( K = 10 \) 
		кэВ проходит через поликристаллическую алюминиевую фольгу, 
		образуя на экране систему дифракционных колец. 
		Вычислить межплоскостное расстояние, соответствующее отражение 
		третьего порядка от некоторой системы кристаллических плоскостей, 
		если ему отвечает дифракционное кольцо диаметра \( D = 3,20 \) см. 
		Расстояние между экраном и фольгой \( l = 10,0 \) см.}

        \[ \Delta = d\sin\varphi \]
        \[ 
            \Delta = n\lambda = \frac{2n\pi\hbar}{p} = 
            \frac{2n\pi\hbar}{\sqrt{2mT}} 
        \]
        \[ d\sin\varphi = \frac{2n\pi\hbar}{\sqrt{2mT}} \]
        \[ d = \frac{2n\pi\hbar}{\sqrt{2mT}\sin\varphi} \]

	\emph{2.7: Интерпретировать квантовые условия Бора на основе волновых
		представлений: показать, что электрон в атоме водорода может 
		двигаться только по тем круговым орбитам, на которых укладывается 
		целое число дебройлевских волн.}

        \[ 
            \lambda = \frac{h}{p};\quad 
            p = mv = \frac{h}{p\lambda}
        \]
        \[ mv(2\pi R) = \frac{h}{p\lambda}(2\pi R) \]
        \[ mv(2\pi R) = hn \]
        Условие квантования: \( pRmv = (nh)/(2\pi) \)
        \[
            \frac{h}{p\lambda}(2\pi R) = nh;\quad
            2\pi R = n\lambda;\quad
            R = \frac{n\lambda}{2\pi}
        \]

	\emph{2.8: Убедиться, что измерения координаты частицы с помощью 
		микроскопа вносит неопределенность в её импульс \( \Delta p_x \), 
		такую, что \[ \Delta x\cdot\Delta p_x \geq h \] Иметь в виду, 
		что разрешение микроскопа \( d = \lambda/\sin\theta \), 
		где \( \lambda \) - длина волны используемого света.}

        У фотона, рассеянного на микрочастице и прошедшего через 
        объектив O, проекция импульса \( p_x \) не превышает, как 
        видно из рисунка :), значение 
        \( p\sin\theta = \hbar k\sin\theta \), где 
        \( k = 2\pi/\lambda \). Эта величина характеризует и 
        неопределенность \( \Delta p_x \) фотона. Но при рассеянии 
        фотона на микрочастице последняя испытывает отдачу, в 
        результате чего её импульс получит такую же неопределенность 
        \( \Delta p_x \), как и фотон: 
        \( \Delta p_x \approx \hbar k\sin\theta \). \\
        Имея, кроме того, в виду, что неопределенность координаты 
        \( x \) микрочастицы \( \Delta x \approx d = \lambda/\sin\theta \) 
        получим в результате: 
        \[ 
            \Delta x\cdot\Delta p_x \approx \frac{\lambda}{\sin\theta}
            \frac{2\pi\hbar}{\lambda}\sin\theta = 2\pi\hbar
        \]
        в чём и следовало убедиться.\\

    \emph{2.9: Плоский поток частиц падает нормально на диафрагму 
       	с двумя узкими щелями, образуя на экране дифракционную 
       	картину. Показать, что попытка определить, через какую 
       	щель прошла та или иная частица (например, с помощью 
       	введения индикатора И), приводит к разрушению дифракционной 
       	картины. Для простоты считать углы дифракции малыми.}\\

    \emph{2.10: Оценить минимально возможную энергию электронов 
       	в атоме \( Не \) и соответствующее расстояние электронов 
       	от ядра.}\\

    \emph{2.11: Оценить с помощью соотношения неопределенностей 
       	неопределенность скорости электрона в атоме водорода, 
       	полагая размер атома \( l = 0,10 \) нм. Сравнить полученную 
       	величину со скоростью электрона на первой боровской 
       	орбите данного атома.}\\

    \emph{2.12: Оценить с помощью соотношения неопределенностей 
       	минимальную кинетическую энергию электрона, локализованного 
       	в области размером \( l = 0,20 \) нм.}\\

    \emph{2.13: Электрон с кинетической энергией \( K \approx 4 \) эВ 
       	локализован в области размером \( l \approx 1 \) мкм. 
        Оценить с помощью соотношения неопределенностей 
        относительную еопределенность его скорости.}\\

    \emph{2.14: Электрон находится в одномерной прямоугольной 
      	потенциальной яме с бесконечно высокими стенками. 
       	Ширина ямы \( l \). Оценить с помощью соотношения 
       	неопределенностей силу давления электрона на стенки этой
		ямы при минимально возможной его энергии.}\\

	\emph{2.15: След пучка электронов на экране электронно-лучевой 
		трубки имеет диаметр \( d \approx 0,5 \) мм. Расстояние от 
		электронной пушки до экрана \( l \approx 20 \) см, 
		ускоряющее напряжение \( U = 10 \) кВ. Оценить с помощью 
		соотношения неопределенность координаты электрона на экране.}\\

	\emph{2.16.Частица массы m движется в одномерном потенциальном 
		поле \( U = \cfrac{\chi x^2}{2} \) (гармонический осциллятор). 
		Оценить с помощью соотношения неопределенностей минимально 
		возможную энергию частицы в таком поле.}\\

	\emph{2.17: Параллельный пучок атомов водорода со скоростью 
		\( \nu = 600 \) м/с падает нормально на узкую щель, за которой 
		на расстоянии \( l = 1,0 \) м расположен экран. Оценить с помощью 
		соотношения неопределенностей ширину \( b \) щели, при которой 
		ширина изображения её на экране будет минимальной.}\\

    \emph{4.17: Вычислить энергию связи K -- электрона ванадия, для
        которого длина волны L -- края поглощения 
        \( \lambda_L = 2,4 \) нм.}

        С помощью схемы :) можно записать, что искомая энергия связи: 
        \[ E_K = \hbar\omega_L + \hbar\omega_{K\alpha} \]
        где \( \omega_L = 2\pi c/\lambda_L \) и 
        \( \omega_{K\alpha} \) -- частота определяемая законом Мозли:
        \[ \omega_{K\alpha} = \frac{3}{4}R(Z-\sigma)^2 \]
        В результате получаем:
        \[ 
            E_K = \hbar
            \Big( 
                \frac{2\pi c}{\lambda_L} + \frac{3}{4}R(Z-1)^2
            \Big)
        \]

	\emph{5.1: Найти магнитный момент \( \mu \) и возможные проекции 
        \( \mu_z \) атома в состоянии:\\
        а) \(^1F \) \\
        б) \(^2D_{3/2} \)} 
    
    Основные формулы:
		\[ 
			\mu_L = -\mu_\text{Б}\sqrt{L(L+1)};\quad
			\mu_{LZ} = m_L\mu_\text{Б};\quad
			m_L = 0, \pm1, \pm2, ..., \pm L 
		\]
		\[ 
			\mu_S = -2\mu_\text{Б}\sqrt{S(S+1)};\quad
			\mu_{SZ} = 2m_S\mu_\text{Б};\quad
			m_S = -S, -S+1, ..., +S  
		\]
		\[ 
			\mu_J = -\mu_\text{Б}g\sqrt{J(J+1)};\quad
			\mu_{JZ} = m_J g\mu_\text{Б};\quad
			m_J = -J, -J+1, ..., +J  
		\]
		\[
			g = 1 + \cfrac{J(J+1)+S(S+1)-L(L+1)}{2J(J+1)} 
		\]
	\begin{itemize}\itemsep-8pt
		\item[а)] \( ^1F: L = 3, S = 0, J = 3 \)
			\[ g = 0 \]
			\[ \mu_L = -2\mu_\text{Б}\sqrt{3} \]
			\[ \mu_S = 0 \]
			\[ \mu_J = 0 \]
			\[ m_L = 0, \pm1, \pm2, \pm3 \]
			\[ m_S = 0 \]
			\[ m_J = -3, -2, -1, 0, 1, 2, 3 \]
		\item[б)] \( ^2D_{3/2}: L = 2, S = \cfrac{1}{2}, J = \cfrac{3}{2} \)
			\[ g = \cfrac{12}{5} \]
			\[ \mu_L = -\mu_\text{Б}\sqrt{6} \]
			\[ \mu_S = -\mu_\text{Б}\sqrt{3} \]
			\[ \mu_J = -\cfrac{6}{5}\mu_\text{Б}\sqrt{15} \]
			\[ m_L = 0, \pm1, \pm2 \]
			\[ m_S = -\cfrac{1}{2}, \cfrac{1}{2} \]
			\[ m_J = -\cfrac{3}{2}, -\cfrac{1}{2}, \cfrac{1}{2}, \cfrac{3}{2} \]
	\end{itemize}

	\emph{5.2: Вычислить магнитный момент атома водорода в основном 
        состоянии.} 
    
    Используем формулы из предыдущей задачи.
		\[
			S = \cfrac{1}{2} \text{ из условия }
			m_S = \cfrac{1}{2} \text{ и } S = \sum{m_S} = \cfrac{1}{2} 
		\]
		\[ 
			L = 0;\quad
			J = L + S \Rightarrow J = \cfrac{1}{2}
		\]
		\[
			g = 2;\quad 
			\mu_L = 0;\quad
			\mu_S = -\mu_\text{Б}\sqrt{3};\quad
			\mu_J = -\mu_\text{Б}\sqrt{3};\quad
		\]

	\emph{5.3: Найти механические моменты атомов в состоянии 
        \( ^5F \) и \( ^7H \), если известно, что в этих состояниях 
        магнитные моменты равны нулю.} 
    
    Основные формулы:
		\[
			M_L = \hbar\sqrt{L(L+1)};\quad
			M_S = \hbar\sqrt{S(S+1)};\quad
			M_J = \hbar\sqrt{J(J+1)}
		\]
		\begin{itemize}\itemsep-8pt
			\item[а)] \( ^5F \)
				\[ L = 3,\quad S = 2,\quad J = 0 \] 
				\[ 
					M_L = 2\hbar\sqrt{3};\quad
					M_S = \hbar\sqrt{6};\quad
					M_J = 0
				\]
			\item[б)] \( ^7H \)
				\[ L = 4,\quad S = 3,\quad J = 0 \] 
				\[ 
					M_L = 2\hbar\sqrt{5};\quad
					M_S = 2\hbar\sqrt{3};\quad
					M_J = 0
				\]
		\end{itemize}

	\emph{5.4: Механический момент атома в состоянии \( ^3F \)
        прецессирует в магнинтном поле \( B = 500 \) Гс с 
        угловой скоростью \( \omega = 5,5\cdot10^9 \) рад/с. 
        Определить механический и магнитный моменты атома.}
		
        \[ \mu_\text{Б} = \frac{e\hbar}{2m} \]
		\[ ^3F: S = 1, L = 3, J = L+S = 4 \]
		\[ \dot{\vec{M_J}} = \vec{N} = \vec{\mu_J}\times\vec{B} \]
		\[ 
            M_J\cdot\frac{d\phi}{dt}\cdot\sin\alpha = 
            \mu_J\cdot B\cdot\sin\alpha 
        \]
		\[ M_J\omega = \mu_J\cdot B \]
		\[ M_J = \frac{\mu_JB}{\omega} \]
		\[ g = 1 + \frac{S(S+1)+J(J+1)-L(L+1)}{2J(J+1)} = \frac{5}{4} \]
		\[ 
            \mu_J = \mu_\text{Б}g\sqrt{J(J+1)} 
            = \frac{5\sqrt{5}}{2}\mu_\text{Б} 
        \]
		В результате получаем:
		\[ M_J = \frac{5\sqrt{5}}{2}\cdot\frac{\mu_\text{Б}B}{\omega} \]

	\emph{5.5: Объяснить с помощью векторной модели, почему механический момент
        атома, находящегося в состоянии \( ^6F_{1/2} \) прецессирует в магнитном 
        поле \( B \) с угловой скоростью \( \omega \), вектор которого направлен 
        противоположно вектору \( \vec{B} \).} \\

	\emph{5.6: Узкий пучок атомов пропускают по методу Штерна и 
        Герлаха через резко неоднородное магнитное поле. Определить:\\ 
        а) максимальные значения проекций магнитных моментов атомов 
        в состояниях \( ^4F \), \( ^6S \) и \( ^5D \), если известно, 
        что пучок расщепляется соответственно на 4, 6 и 9 компонент;\\
        б) на сколько компонент расщепится пучок атомов, 
        находящихся в состояниях \( ^3D_2 \) и \( ^5F_1 \)?}

		\vspace*{-1em}
		\begin{itemize}\itemsep-8pt
			\item[а)] Используемые формулы:
			\[ 
				\mu_{LZ} = m_L\mu_\text{Б};\quad
				m_L = 0, \pm1, \pm2, ..., \pm L 
			\]
			\[ 
				\mu_{SZ} = 2m_S\mu_\text{Б};\quad
				m_S = -S, -S+1, ..., +S  
			\]
			\[ 
				\mu_{JZ} = m_J g\mu_\text{Б};\quad
				m_J = -J, -J+1, ..., +J  
			\]
			\[ g = 1 + \frac{S(S+1)+J(J+1)-L(L+1)}{2J(J+1)} \]

			Рассмотрим каждое состояние атома в отдельности:
			\[ ^4F: L=3, S=\frac{3}{2}, J= \frac{3}{2} \]
			\[ 
				g = 1 + \frac{\cfrac{3}{2}\cdot\cfrac{5}{2} 
				+ \cfrac{3}{2}\cdot\cfrac{5}{2} - 3\cdot4}{3\cdot\cfrac{5}{2}} 
				= \frac{2}{5}  
			\]
			\[ 
				m_L = 0, \pm1, \pm2, \pm3;\quad
				m_S = -\frac{3}{2}, -\frac{1}{2}, \frac{1}{2}, \frac{3}{2}
			\]
			\[
				m_J = -\frac{3}{2}, -\frac{1}{2}, \frac{1}{2}, \frac{3}{2} \leftarrow 
				\text{расщепление на 4 компоненты}
			\]
			\[ 
				\mu_{maxLZ} = 3\mu_\text{Б};\quad
				\mu_{maxSZ} = 3\mu_\text{Б};\quad
				\mu_{maxJZ} = \frac{6}{5}\mu_\text{Б};
			\] \\

			\[ ^6S: L=0, S=\frac{5}{2}, J= \frac{5}{2} \]
			\[ 
                g = 1 + \frac{\cfrac{35}{4}+\cfrac{35}{4}}{\cfrac{35}{2}} 
                = 2
            \]
			\[ 
				m_L = 0;\quad
				m_S = -\frac{5}{2}, -\frac{3}{2}, -\frac{1}{2}, 
				\frac{1}{2}, \frac{3}{2}, \frac{5}{2}
			\]
			\[
				m_J = -\frac{5}{2}, -\frac{3}{2}, -\frac{1}{2}, \frac{1}{2}, 
				\frac{3}{2}, \frac{5}{2} \leftarrow \text{расщепление на 6 компонент}
			\]
			\[ 
				\mu_{maxLZ} = 0;\quad
				\mu_{maxSZ} = 5\mu_\text{Б};\quad
				\mu_{maxJZ} = 10\mu_\text{Б}
			\]

			\[ ^5D: L=2, S=2, J=4\]
			\[ 
                g = 1 + \frac{2\cdot3 + 4\cdot5 - 2\cdot3}{8\cdot5} = 
                \frac{3}{2}
            \]
			\[ 
				m_L = 0, \pm1, \pm2;\quad
				m_S = -2, -1, 0, 1, 2
			\]
			\[
				m_J = -4, -3, -2, -1, 0, 1, 2, 3, 4 \leftarrow 
				\text{расщепление на 9 компонент}
			\]
			\[ 
				\mu_{maxLZ} = 2\mu_\text{Б};\quad
				\mu_{maxSZ} = \mu_\text{Б};\quad
				\mu_{maxJZ} = 3\mu_\text{Б}
			\]
			\item[б)] Запишем для каждого терма значение \( m_J \)
			\[ ^3D_2: L=2, S=1, J=2 \]
			\[ 
                m_J = -2, -1, 0, 1, 2 \leftarrow 
                \text{расщепление на 5 компонент} \]
			\[ ^5F_1: L=3, S=2, J=1 \]
			\[ 
                m_J = -1, 0, 1 \leftarrow 
                \text{расщепление на 3 компоненты} 
            \]
		\end{itemize}

    \emph{5.7: Атом находится в магнитном поле \( B = 3,00 \) кГс. 
        Определить:\\
        а) полное расщепление, \( \text{см}^{-1} \), терма \( ^1D \);\\
        б) спектральный символ синглетного терма, полная ширина расщепления 
        которого составляет 0,84 \( \text{см}^{-1} \).}

        Используемые формулы: 
		\[ 
            \Delta\omega = \frac{\mu_\text{Б}B}{\hbar}(m_{J2}g_{2}-m_{J1}
            g_{1}) 
        \]
		\[ m_J = -J, -J+1, ..., +J \]
		\[ g = 1 + \frac{S(S+1)+J(J+1)-L(L+1)}{2J(J+1)} \]
		\begin{itemize}\itemsep-8pt
			\item[а)] 
			\[ 
				^1D: L=2, S=0, J=2;\quad
				m_J = -2, -1, 0, 1, 2;\quad
				g = 1
			\]
			В предположении, что полное расщепление образуется в случае разности
			\( m_J \) и постоянства числа \( g \), получаем:
			\[ 
				\Delta\omega = \frac{\mu_\text{Б}B}{\hbar}(2+2) 
				= 4\frac{\mu_\text{Б}B}{\hbar} 
			\]
			\item[б)]
		\end{itemize}

    \emph{5.8: Спектральная линия \( \lambda = 0.612 \) мкм обусловлена переходом
        между двумя синглетными термами атома. Определить интервал \( \Delta\lambda \)
        между крайними компонентами этой линии в магнитном поле 
        \( B = 10,00 \) кГс.}\\

	\emph{5.9: Построить схему возможных переходов между термами \( ^2P_{3/2} \) 
        и \( ^2S_{1/2} \) в слабом магнитном поле. Вычислить для соответствующей 
        спектральной линии: \\
        а) смещения зеемановских компонент в единицах 
        \( \mu_\text{Б}B/\hbar \);\\
        б) интервал, \( \text{см}^{-1} \), между крайними компонентами, если \( B = 5,00 \) кГс.} 

        Используемые формул:
		\[ \Delta\omega = \frac{\mu_\text{Б}B}{\hbar}(m_{J2}g_{2}-m_{J1}g_{1}) \]
		\[ m_J = -J, -J+1, ..., +J \]
		\[ g = 1 + \frac{S(S+1)+J(J+1)-L(L+1)}{2J(J+1)} \]
		Распишем значения двух термов:
		\[ 
			^2P_\frac{3}{2}: L = 1, S = \frac{1}{2}, J = \frac{3}{2};\quad
			m_J = -\frac{3}{2}, -\frac{1}{2}, \frac{1}{2}, \frac{3}{2};\quad
			g = \frac{1}{3}
		\]
		\[ 
			^2S_\frac{1}{2}: L = 0, S = \frac{1}{2}, J = \frac{1}{2};\quad
			m_J = -\frac{1}{2}, \frac{1}{2};\quad
			g = 2
		\]
		\begin{itemize}\itemsep-8pt
			\item[а)]
			\[ 
				\Delta\omega = \frac{\mu_\text{Б}B}{\hbar}
				(\frac{3}{2}\cdot\frac{1}{3} - 2\cdot\frac{1}{2}) = 
				-\frac{1}{2}\frac{\mu_\text{Б}B}{\hbar}
			\]
			\item[б)]
		\end{itemize}

	\emph{5.10: Изобразить схему возможных переходов в слабом магнитном поле и
        вычислить смещения (в единицах \( \mu_\text{Б}B/\hbar \)) 
        зеемановских компонент спектральной линии: \\
        а) \( ^2D_{3/2} \rightarrow ^2P_{3/2} \);\\
        б) \( ^2D_{5/2} \rightarrow ^2P_{3/2} \).} 

        Используемые формулы: 
		\[ 
			\Delta\omega = \frac{\mu_\text{Б}B}{\hbar}(m_{J2}g_2 - m_{J1}g_1)
			= \Delta\omega_0 (m_{J2}g_2 - m_{J1}g_1)
		\]
		\[ g = 1 + \cfrac{J(J+1)+S(S+1)-L(L+1)}{2J(J+1)} \]
		Рассмотрим пункт решение для пункта а:
		\[ ^2D_{3/2}: L=2, S=\frac{1}{2}, J=\frac{3}{2} \]
		\[ m_{J2} = -\frac{3}{2}, -\frac{1}{2}, \frac{1}{2}, \frac{3}{2} \]
		\[ 
			g_2 = 1 + \frac{\cfrac{3}{2}\cdot\cfrac{5}{2} 
			+ \cfrac{1}{2}\cdot\cfrac{3}{2} - 2\cdot3}{\cfrac{15}{2}} 
			= \frac{24}{30}
		\]
		\[ ^2P_{3/2}: L=1, S=\frac{1}{2}, J=\frac{3}{2} \]
		\[ m_{J1} = -\frac{3}{2}, -\frac{1}{2}, \frac{1}{2}, \frac{3}{2} \]
		\[ g_1 = 1 + \frac{\cfrac{18}{4} - 2}{\cfrac{15}{2}} = \frac{2}{3} \]
		Правило отбора: \( \Delta m_J = 0, \pm1 \).
		\[ -\frac{3}{2} \rightarrow -\frac{3}{2}: 
			\Delta\omega = \Delta\omega_0 (-\frac{3}{2}\cdot\frac{24}{30} 
			+ \frac{3}{2}\cdot\frac{2}{3}) = -\frac{1}{5}\Delta\omega_0 
		\]
		Считаем \( \Delta\omega \) для чисел:
		\[ 
			-\frac{3}{2} \rightarrow -\frac{1}{2};\quad
			-\frac{1}{2} \rightarrow -\frac{1}{2};\quad
			-\frac{1}{2} \rightarrow \frac{1}{2};\quad
		\]
		\[ 
			\frac{1}{2} \rightarrow \frac{1}{2};\quad
			\frac{1}{2} \rightarrow \frac{3}{2};\quad
			\frac{3}{2} \rightarrow \frac{3}{2};\quad
		\]
		И находим разность между двумя \( \Delta\omega \). \\
		Пункт б считается по аналогии.

	\emph{5.11: Найти значения температуры, при которых средняя кинетическая 
        энергия поступательного движения молекул \( H_2 \) и \( N_2 \) равна их вращательной энергии в состоянии с квантовым числом \( J = 1 \).}
		
        \emph{Основная идея}: записываем энергию вращательного движения молекулы
		( \( E_J = \frac{\hbar^2}{2I}J(J+1) \) ), где \( I \) -- 
		момент инерции молекулы, \( J \) -- вращательное квантовое число; 
		приравниваем к значению \( \frac{3}{2} kT \), где k -- постоянная Больцмана 
		(\( 1.38\cdot10^{-23} \text{Дж}\cdot\text{К}^{-1} \)), T -- 
		температура; выражаем и находим значение для каждой из молекул.

	\emph{5.12: } \\

	\emph{5.13: } \\

	\emph{5.14: } \\

	\emph{5.15: Для двухатомной молекулы известны интервалы между 
        тремя последовательными вращательными уровнями \( \Delta E_1 = 
        0,20 \) МэВ }
        
		Используемая формула: \[ E_r = \frac{\hbar^2}{2I}J(J+1) \]
		Запишем формулы для вращательных уровней:
		\[ E_{J1} = \frac{\hbar^2}{2I}J(J+1) \]
		\[ E_{J2} = \frac{\hbar^2}{2I}(J+1)(J+2) = \frac{\hbar^2}{2I}n(n+1) \]
		\[ E_{J3} = \frac{\hbar^2}{2I}(J+2)(J+3) \]
		Сделав обозначение \( n = J + 1 \). \\
		Распишем разность энергий через формулы для уровней:
		\[ 
			\Delta E_1 = E_{J2} - E_{J1} = 
			\frac{\hbar^2}{2I}((J+1)(J+2) - J(J+1)) = \frac{\hbar^2}{I}(J+1) 
		\]
		\[ 
			\Delta E_1 = E_{J2} - E_{J1} =
			\frac{\hbar^2}{2I}((J+2)(J+3)-(J+1)(J+2)) = \frac{\hbar^2}{I}(J+2)
		\]
		Найдём значение \( J \) поделив \( \Delta E_1 \) на \( \Delta E_2 \):
		\[ \frac{\Delta E_1}{\Delta E_2} = \frac{J+1}{J+2} \]
		Сделав преобразования относительно \( J \) получим:
		\[ J = \frac{\Delta E_2 - 2\Delta E_1}{\Delta E_1 - \Delta E_2} = 1 \]
		Отсюда получаем значение для среднего уровня \( n = J+1 = 2 \). \\
		Для определение момента запишем: 
		\[ 
			\Delta E_2 - \Delta E_1 = \frac{\hbar^2}{I}(J+2-J-1) 
			\Rightarrow I = \frac{\hbar^2}{\Delta E_2 - \Delta E_1}
		\]

	\emph{5.16: Оценить, сколько линий содержит чисто вращательный 
        спектр молекул \( CO \), момент инерции которых равен 
        \( I = 1,44\cdot10^{-39} \text{г}\cdot\text{см}^2 \).}

        Возможно задача не допоставлена и нехватает
        \( \omega = 4.1\cdot10^{14} \text{с}^{-1} \). 

        \emph{Решение:}

        Искомое число линий должно быть равно числу вращательный уровней 
        между нулевым и первым возбужденным колебательными уровняит 
        (\( \nu = 0 \) и \( \nu = 1 \)), интервал между которыми согласно 
        формуле: 
        \[ E_\nu = (\nu + \frac{1}{2})\hbar\omega \]
        равен \( \hbar\omega \). Задача, таким образом, сводиться к 
        определению максимального вращательного квантового числа \( r \) 
        уровня с энергией \( \hbar\omega \). Учитывая формулу:
        \[ E_r = \frac{\hbar^2}{2I}r(r+1) \]
        запишем 
        \[ \hbar\omega = \frac{\hbar^2}{2I}r(r+1) \]
        откуда 
        \[ r^2 + r - \frac{2I\omega}{\hbar} = 0 \]
        Решение этого уравнения даёт \( r_{\text{макс}} \):
        \[ 
            r_\text{макс} = \frac{-1 + 
            \sqrt{1+4(2I\omega/\hbar}}{2} \approx 
            \frac{2I\omega}{\hbar} = 33
        \]
        Следовательно, чисто вращательных спектр данной молекулы 
        содержит около 30 линий.

	\emph{5.17: } \\
\end{document}
