\section{Качественные задачи.}

\begin{enumerate}

% 1.1 --------------
\item Как изменится полная волновая функция \( \Psi(x,t) \), описывающая
стационарные состояния, если изменить начало отсчёта потенциальной энергии на
некоторую величину \( \Delta U \)?

\emph{Ответ:}
Измениться лишь временной множитель полной волновой функции. А так как
физический смысл имеет лишь квадрат модуля этой функции, то изменение временного
множителя никак не проявится.

\vspace*{1.5em}
% 1.2 --------------
\item Объясните с позиции квантовой теории: почему невозможно состояние, в
котором частица находилась бы в состоянии покоя?

\emph{Ответ:}

\vspace*{1.5em}
% 1.3 --------------
\item Имеет ли смысл делить полную энергию на кинетическую и потенциальную с
позиции квантовой теории?

\emph{Ответ:}
В квантовой механике теряет смысл деление полной энергии \( E \) на кинетическую
и потенциальную, т.к. одна из этих величин зависит от импульса, а другая -- от
координаты. Следовательно, в силу соотношения неопределенностей, эти величины не
могут иметь одновременно определённые значения.

\vspace*{1.5em}
% 1.4 --------------
\item Объясните, почему электрон не падает на ядро с позиции квантовой теории.

\emph{Ответ:}

\vspace*{1.5em}
% 1.5 --------------
\item Поясните причину возникновения <<туннельного эффекта>>.

\emph{Ответ:}
Волновая функция \( \psi \) есть величина вспомогательная: все реально
наблюдаемые величины связаны с ней вероятностными соотношениями. Поскольку
функция \( \psi \) всюду отлична от нуля, существует конечная вероятность
обнаружения частицы как внутри барьера, так и за его пределами.

\vspace*{1.5em}
% 1.6 --------------

\item Объясните основные отличия между квантовым и классическим гармоническим
осциллятором.

\emph{Ответ:}
Квантовый осциллятор имеет дискретный спектр, а классический -- сплошной.
% дополнить

\vspace*{1.5em}
% 1.7 --------------
\item Может ли частица находится на дне потенциальной ямы? Определяется ли это
формой ямы?

\emph{Ответ:}
Падение на дно потенциальной ямы связано с обращением в ноль импульса частицы.
Тогда неопределенность координаты становится сколько угодно большой, что
противоречит соотношению неопределенностей Гейзенберга:
\[
    \Delta x \cdot \Delta p_x \ge \hbar.
\]

\vspace*{1.5em}
% 1.8 --------------
\item В чём состояла ценность опытов Штерна и Герлаха?

\emph{Ответ:}
Наличие у атомов магнитных моментов и их квантование было доказано прямыми
опытами Штерна и Герлаха.
% дополнить, ценность не раскрыта -- опыты были одним из подтверждений
% неполноты квантовой теории на тот момент, а именно - не учитывался спин частиц

\vspace*{1.5em}
% 1.9 --------------
\item Дайте наглядное истолкование спин-орбитальному взаимодействию.

\emph{Ответ:}
Воспользуемся теорией Бора для атома водорода: электрон, вращающийся по орбите,
обладает спином и тем самым обладает спиновым магнитным моментом. Электрическое
поле ядра оказывает воздействие на спиновый магнитный момент, в этом легко
убедится, если перейти в систему отсчёта, где электрон покоится, а ядро
движется, создавая магнитное поле, которое будет взаимодействовать со спиновым
магнитным моментом.

\vspace*{1.5em}
% 1.10 -------------
\item Почему расщепление дублетов резкой серии в спектрах щелочных металлов
одинаково для всех линий, а главной серии -- неодинаково?

\emph{Ответ:}
Тонкая структура уровней и спектральных щелочных металлов в основном обусловлена
спин-орбитальным взаимодействием. Главная серия возникает в результате переходов
на наиболее низкий уровень S с вышележащих P-уровней. Уровень S синглетный, а
все P-уровни двойные, причём расстояние между компонентами этих уровней убывает
с возрастанием главного квантового числа n. Поэтому и сами спектральные линии
главной серии получаются дублетами, расстояния между отдельными линиями которых
различны. Линии резкой серии возникают в результате переходов с S-уровней на
двойной P-уровень. Поэтому расстояния между компонентами дублетов одни и те же
для всей серии.
% месиво, просмотреть логику и терминологию

\vspace*{1.5em}
% 1.11 -------------
\item Почему количество линий, наблюдаемых при нормальном эффекте Зеемана
различно при наблюдении вдоль и перпендикулярно магнитному полю?

\emph{Ответ:}
Эффект Зеемана -- это расщепление энергетических уровней под действием
магнитного поля. При наложении магнитного поля движение электрона становится
сложным, а также будет сложным спектр его излучения. Его можно представить как
совокупность трёх монохроматических волн разной частоты \( (\nu_0 - \Delta \nu),
\nu_0, (\nu_0 + \Delta \nu) \) в разных состояниях поляризации. При наблюдении в
магнитном поле в направлении, перпендикулярном полно (поперечный эффект Зеемана)
в спектрах излучения и поглощения обнаруживается триплет -- три линейно
поляризованные спектральные линии: несмещенная линия первоначальной частоты
с электрическим вектором \( \vec{E} \), направленным вдоль \( \vec{B} \), и две
смещенные линии с частотами \( (\nu_0 - \Delta \nu) \) и \( \nu_0 + \Delta \nu\)
и электрическим вектором \( \perp \) магнитному полю.

При наблюдении вдоль магнитного поля (продольный эффект Зеемана) в спектрах
обнаруживается дублет -- две симметрично смещенные спектральные линии с
частотами \( (\nu_0 - \Delta \nu) \) и \( (\nu_0 + \Delta \nu) \). Обе линии
оказываются поляризованными по кругу. Линия с \( (\nu_0 - \Delta \nu) \)
поляризована по левому кругу, а \( (\nu_0 + \Delta \nu) \) по правому. При
продольном наблюдении несмещенная спектральная компонента отсутствует.
% месиво, но в этот раз все вроде нормально написано

\vspace*{1.5em}
% 1.12 -------------
\item Как будет видоизменяться спектр поглощения рентгеновского излучения
веществом при уменьшении энергии излучения?

\emph{Ответ:}
При высоких \( E \) возбуждены все серии, при уменьшении энергии происходит
прекращение возбуждения K-серии. Появляется край полосы поглощения, при
дальнейшем уменьшении энергии на кривой поглощения появляется L-край.

\vspace*{1.5em}
% 1.13 -------------
\item Почему L-край полосы поглощения рентгеновского излучения веществом
состоит из трёх <<зубчиков>>?

\emph{Ответ:}
Появление зубчиков связано с тонкой структурой рентгеновских спектров.
% дополнить

\vspace*{1.5em}
% 1.14 -------------
\item Объясните, почему при комнатных температурах интенсивность красных спутников
заметно выше, чем фиолетовых в явлении комбинационного рассеяния света.
        
\emph{Ответ:}
При обычных температурах, согласно распределению Больцмана, число молекул в
возбужденном состоянии значительно меньше, чем в основном. Поэтому в основном
будут происходить процессы поглощения энергии. Значит интенсивность красных
спутников будет больше.

\vspace*{1.5em}
% 1.15 -------------
\item Можно ли наблюдать для молекулы водорода вращательный и
колебательно-вращательный спектр?
        
\emph{Ответ:}
Нет, т.к. вращательный и колебательно-вращательный спектры наблюдаются на опыте
только для несимметричных молекул.
% дополнить

\vspace*{1.5em}
% 1.16 -------------
\item Почему при электронных переходах в молекулах меняется колебательный и
вращательный характер движения?

\emph{Ответ:} При электронном переходе изменяется электронная конфигурация
оболочки, следовательно, изменяются силы, действующие между ядрами.
Следовательно, меняются и колебательный, и вращательный характер движения. Т.е.
при электронном переходе меняются все три составляющие энергии.

\vspace*{1.5em}
% 1.17 -------------
\item Дайте наглядное истолкование принципу Франка-Кондона.

\emph{Ответ:}
Электронных переход, происходящий наиболее вероятно без изменения положения ядер
в молекуле.
% дополнить, можно даже рисунок

\end{enumerate}
