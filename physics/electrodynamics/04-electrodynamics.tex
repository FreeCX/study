\input{../../.preambles/01-semester_work}
\input{../../.preambles/10-russian}
\input{../../.preambles/20-math}
\input{../../.preambles/22-vectors}
\input{../../.preambles/30-physics}
\usepackage{wrapfig}

\begin{document}
\maketitlepage{Факультет электроники и вычислительной техники}{физики}
{Электродинамика}{4}{}{студент группы Ф-369\\Голубев~А.~В.}
{}{доцент Грецов~М.~В.}{}{}

\newpage
\emph{Задача №3.32}: В диэлектрике имеется постоянное магнитное поле 
\( H_0 \). Исследовать монохроматическую волну, которая распространяется:
а) вдоль и б) поперёк магнитного поля. Диэлектрик считать прозрачным и 
немагнитным. 

\emph{Решение:}

\[
	\begin{array}{ll}
	\left[\cfrac{v^2}{c^2}\right]^2 \eps_{zz}\left( e^2_{xx} - 
		\vert \eps_{xy} \vert^2 \right) - \cfrac{v^2}{c^2}
		\Big[ \left( 1 + \cos^2\theta \right) \eps_{xx}\eps_{zz} + \\
		\left( 1 - \cos^2\theta \right) \left( \eps^2_{xx} - 
		\vert\eps_{xy}\vert^2 \right)\Big] + \left( \eps_{xx}\sin^2\theta + 
		\eps_{zz}\cos^2\theta \right) = 0
	\end{array}
\]
где \( \theta \) -- угол между направлением волнового вектора \( k_0 \) 
и направлением поля \( H_0 \)

Рассмотрим первый случай, когда волна распространяется вдоля поля 
\( \vec{H_0} \). В этом случае угол \( \theta = 0 \) и уравнение преобразуется 
к виду (сокртив на общий множитель \( \eps_{zz} \)):
\[
	\Big[ \frac{v^2}{c^2} \Big]^2 \Big( \eps^2_{xx} - 
		\vert\eps_{xy}\vert^2 \Big) - 2\frac{v^2}{c^2}\eps_{xx} + 1 = 0
\] 

Решая систему относительно \( v^2 / c^2 \):
\[
	\frac{v^2}{c^2} = \frac{\eps_{xx} \pm \vert \eps_{xy} \vert}
		{\eps^2_{xx} - \vert\eps_{xy}\vert^2} = 
		\frac{1}{\eps_{xx} \pm \vert\eps_{xy}\vert}
\]

Подставляя в уравнение ...

Рассмотрим случай, когда волна распространяется поперёк поля. В этом случае 
угол \( \theta = \pi / 2 \) и фазовая скорость определяется уравнением:
\[
	\Big[ \cfrac{v^2}{c^2} \Big]^2 \eps_{zz}\left( \eps^2_{xx} - 
		\vert\eps_{xy}\vert^2 \right) - \cfrac{v^2}{c^2}
		\Big[ \eps_{xx}\eps_{zz} + \eps^2_{xx} - 
		\vert\eps_{xy}\vert^2\Big] + \eps_{xx} = 0
\]

Решая систему относительно \( v^2 / c^2 \):
\[
	\frac{v^2_1}{c^2} = \frac{1}{\eps_{zz}};\quad
	\frac{v^2_2}{c^2} = \frac{\eps_{xx}}{\eps^2_{xx} - 
		\vert\eps_{xy}\vert^2}
\]

\newpage

\emph{Задача №3.51}: В вакууме распространяется плоская монохроматичная 
волна, которая под углом \( \alpha \) падает на плоскую границу ионосферы. 
Рассматривая ионосферу как разреженный электронный газ (наличие тяжелых 
ионов можно не учитывать), вычислить коэффициенты отражения и прохождения. 
Показать, что если \( \omega^2 < \cfrac{4\pi Ne^2}{m} \) (\(N\) -- число 
электронов в единице объёма; \( \omega \) -- частота световой волны), то 
имеет место полное отражение.

\emph{Решение:}

Запишем формулы Френеля для случая изотропных, непроводящих и немагнитных 
сред:
\[
	\begin{array}{ll}
		R_{\Vert} = A_{\Vert}\cfrac{\tg(\alpha-\beta)}{\tg(\alpha+\beta)};
		\hspace{13ex}
		R_{\bot} = -A_{\bot}\cfrac{\sin(\alpha-\beta)}{\sin(\alpha+\beta)} \\
		D_{\Vert} = A_{\Vert}\cfrac{2\cos\alpha\cos\beta}
			{\sin(\alpha+\beta)\cos(\alpha-\beta)};\quad
		D_{\bot} = A_{\bot}\cfrac{2\cos\alpha\sin\beta}{\sin(\alpha+\beta)}
	\end{array}
\]

где \( A, R, D \) -- амплитуды электрических векторов падающей, отраженной 
и преломленной волн.

Из закона Снеллиуса (полагая \( n_0 = 1 \)): \( n\sin\beta = \sin\alpha \). 
Отсюда же можно записать соотношение для \( \cos\beta \):
\[
	\cos\beta = \sqrt{1-\frac{\sin^2\alpha}{n^2}}
\]

Преобразуя систему уравнений получим:
\[
	\begin{array}{ll}
		R_{\Vert} = \cfrac{\cos\alpha\sin\alpha - \cos\beta\sin\beta}
			{\cos\alpha\sin\alpha + \cos\beta\sin\beta}A_{\Vert} = 
			\cfrac{n^2\cos\alpha - \sqrt{n^2-\sin^2\alpha}}
			{n^2\cos\alpha + \sqrt{n^2-\sin^2\alpha}} A_{\Vert} = 
			r_{\Vert} A_{\Vert} \\
		R_{\bot} = \cfrac{\cos\alpha\sin\beta - \sin\alpha\cos\beta}
			{\cos\alpha\sin\beta + \sin\alpha\cos\beta}A_{\bot} = 
			\cfrac{\cos\alpha - \sqrt{n^2-\sin^2\alpha}}
			{\cos\alpha + \sqrt{n^2-\sin^2\alpha}}A_{\bot} = 
			r_{\bot} A_{\bot} \\
		D_{\Vert} = \cfrac{2\cos\alpha\sin\beta}
			{\cos\beta\sin\beta + \cos\alpha\sin\alpha}A_{\Vert} = 
			\cfrac{2n\cos\alpha}{n^2\cos\alpha + \sqrt{n^2-\sin^2\alpha}}
			A_{\Vert} = d_{\Vert} A_{\Vert} \\
		D_{\bot} = \cfrac{2\cos\alpha\sin\beta}
			{\cos\alpha\sin\beta + \sin\alpha\cos\beta}A_{\bot} =
			\cfrac{2\cos\alpha}{\cos\alpha + \sqrt{n^2-\sin^2\alpha}}
			A_{\bot} = d_{\bot} A_{\bot}
	\end{array}
\]

Для определения \( n \) рассмотрим распространение волны:
\[
	\vec{E} = \vec{E}_0 e^{i(\vec{k}\vec{r} - \omega t)}
\]

При соотношении \( \cfrac{v}{c} \gg 1 \) движение частицы можно представить:
\[
	m\dder{r}{t} = e\vec{E}
\]

Выберем начало координат в начальном положении частицы:
\[
	\vec{E} = \vec{E}_0 e^{-i\omega t} \quad \vec{k}\vec{r} \ll \omega t
\]

Полагая начальную скорость равную нулю:
\[
	\vec{r} = -\frac{e}{m\omega^2}\vec{E}
\]

Скорость частицы:
\[
	\vec{v} = \der{r}{t} = -\frac{e}{m\omega^2}\der{E}{t} = i\frac{e}{m\omega}\vec{E}
\]

Отсюда плотность тока:
\[
	\vec{j} = Ne\vec{v} = i\frac{Ne^2}{m\omega}\vec{E}
\]

И проводимость:
\[
	\sigma = i\frac{Ne^2}{m\omega}
\]

Тогда вектор электрической индукции:
\[
	\eps\vec{E} = \vec{D} = \vec{E} + 4\pi N e \vec{r} = 
		E - 4\pi N \frac{e^2}{m\omega^2}\vec{E} 
\]

Отсюда получаем формулу для диэлектрической проницаемости:
\[
	n^2 = \eps = 1 - 4\pi N \frac{e^2}{m\omega^2}
\]

Если \( \omega^2 < \cfrac{4\pi Ne^2}{m} \), то показатель преломления 
\( n^2 < 0 \) и величина \\ \( \sqrt{n^2 - \sin^2\alpha} \) -- чисто мнимая 
величина для любых значений угла \( \alpha \) и тогда
\[
	\vert r_{\Vert} \vert = \vert r_{\bot} \vert = 1 
\]

Тогда коэффициент отражения
\[
	\rho = \frac{\vert r_{\Vert} A_{\Vert} \vert^2 + 
		\vert r_{\bot} A_{\bot} \vert^2}{ \vert A_{\Vert} \vert^2 + 
		\vert A_{\bot} \vert^2} = 1
\]

Коэффициент прохождения
\[
	\tau = \frac{\vert d_{\Vert} A_{\Vert} \vert^2 + 
		\vert d_{\bot} A_{\bot} \vert^2}{ \vert A_{\Vert} \vert^2 + 
		\vert A_{\bot} \vert^2}
\]

\newpage

\emph{Задача №3.73}: Показать, что для распространяющейся в волноводе 
\( H \)--волны коэффициент поглощения
\[
	\alpha = \frac{c\kappa^2\theta'}{2k\omega\mu}
	\frac{\oint\limits_{l} 
		\left\{ |H_z|^2 + \frac{k^2}{\kappa^4}|\nabla_2 H_z|^2 \right\}dl
	}{\oint\limits_{S} |H_z|^2 dS}
\]

\emph{Решение:}

Распространяющееся \( H \)--волны в волноводе с поглощением:
\[
	H = H_0 e^{-\alpha z}
\]

Используя уравнения Максвелла
\[
	\rot\vec{E} = -\frac{\mu}{c}\pder{\vec{H}}{t};\quad
	\rot\vec{H} = \frac{\eps}{c}\pder{\vec{E}}{t}
\]

положим \( E_z = 0 \) и выразим \( E_x, E_y, H_x, H_y \) через
компоненту \( H_z \):
\[
	\left\{ \begin{array}{ll}
		E_x = i\cfrac{\mu\omega}{c\kappa^2}\cfrac{\partial H_z}{\partial y} \\
		E_y = -i\cfrac{\mu\omega}{c\kappa^2}\cfrac{\partial H_z}{\partial x}
	\end{array} \right. \quad
	\left\{ \begin{array}{ll}
		H_x = \cfrac{ik}{\kappa^2}\cfrac{\partial H_z}{\partial x} \\
		H_y = \cfrac{ik}{\kappa^2}\cfrac{\partial H_z}{\partial y}
	\end{array} \right. \eqno{1}
\]
\[
	\Delta_2 H_z + \kappa^2 H_z = 0 \eqno{2}
\]

где
\(
	\kappa^2 = \cfrac{\omega^2}{v^2} - k^2, 
	v^2 = \cfrac{c^2}{\eps\mu}, 
	\Delta_2 = \cfrac{\partial^2}{\partial x^2} + 
		\cfrac{\partial^2}{\partial y^2} 
\)

Запишем модуль вектора \( \vec{H} \) через его компоненты:
\[
	\vert \vec{H}_t \vert^2 = \vert \vec{H} \vert^2 = 
		H^*_x H_x + H^*_y H_y + H^*_z H_z
\]

с учётом уравнения \( [1] \) получим:
\[
	\vert \vec{H} \vert^2 = \vert H_z \vert^2 + 
		\frac{k^2}{\kappa^2} \vert \nabla_2 H_z \vert^2
\]

Полный поток через сечение волновода можно записать в виде:
\[
	J = \int S^+_z dx dy = \frac{c}{8\pi} \int 
		\left( E_x H^*_y - E_y H^*_x \right) dS
\]

подставляя значения из формулы \( [1] \), получим
\[
	J = \frac{\mu k \omega}{8\pi\kappa^4} \int \vert 
		\nabla_2 H_z \vert^2 dS 
\]

Интегрируя по частям с учётом того, что на контуре сечения 
\( H_z = 0 \)
\[
	J = -\frac{\mu k \omega}{8\pi\kappa^4} \int 
		E^*_z \delta_2 H_z dS
\]

и принимая во внимание уравнение \( [2] \), получим:
\[
	J = \frac{\mu k \omega}{8\pi\kappa^2} \int 
		\vert H_z \vert^2 dS
\]

Энергия поглощаемая стенками волновода не единице его длины за 
единицу времени выражается формулой:
\[
	-\frac{dJ}{dz} = \frac{c\theta'}{8\pi} \oint
		\vert \vec{H}_t \vert^2 dl = \frac{c\theta'}{8\pi} \oint 
		\vert H_z \vert^2 +  \frac{k^2}{\kappa^2} 
		\vert \nabla_2 H_z \vert^2
\]

Деля последнее условие на полный поток найдём коэффициент поглощения:
\[
	2\alpha = \frac{c\kappa^2\theta'}{k\omega\mu}
	\frac{\oint\limits_{l} 
		\left\{ |H_z|^2 + \frac{k^2}{\kappa^4}|\nabla_2 H_z|^2 \right\}dl
	}{\oint\limits_{S} |H_z|^2 dS}
\]

\newpage

\emph{Задача №3.74}: Определить собственные электромагнитные колебания 
в полом (\(\eps = \mu = 1\)) резонаторе, имеющих форму прямоугольного 
параллелепипеда с идеально проводящими стенками, ребра которого равны 
\( a_1, a_2 \) и \( a_3 \). Найти наименьшую собственную частоту.

\emph{Решение:}
\begin{wrapfigure}[10]{l}{0.5\textwidth}
	\vspace{-2ex}
	\includegraphics[width=0.5\textwidth]{pdf/image4_4}
\end{wrapfigure}
Воспользуемся волновым уравнение:
\[
	\nabla^2 E + \frac{\omega^2}{c^2} E = 0 \eqno{1}
\]

где 
\( 
	\nabla^2 = \cfrac{\partial^2}{\partial x^2} + 
		\cfrac{\partial^2}{\partial y^2} + 
		\cfrac{\partial^2}{\partial z^2} 
\)

Граничные условия: \( E = 0 \) при 
\(
	x = 0, a_1; y = 0, a_2; z = 0, a_3 
\)

Представим решение уравнения \( [1] \) в виде:
\[
	E = A\sin\left( k_x x + \alpha_x \right)
		\sin\left( k_y y + \alpha_y \right)
		\sin\left( k_z z + \alpha_z \right) \eqno{2}
\]

Подставляя его в уравнение \( [1] \), получаем:
\[
	\frac{\omega^2}{c^2} = k^2_x + k^2_y + k^2_z \eqno{3}
\]

Подставляя граничные условия в уравнение \( [2] \), получаем:
\[
	\left\{ \begin{array}{ll}
		\sin\alpha_x = 0 \quad\Rightarrow\quad 
			\alpha_x = 0 \\
		\sin\alpha_y = 0 \quad\Rightarrow\quad 
			\alpha_y = 0 \\
		\sin\alpha_z = 0 \quad\Rightarrow\quad 
			\alpha_z = 0 \\
	\end{array} \right. \quad
	\left\{ \begin{array}{ll}
		\sin k_x a_1 = 0 \quad\Rightarrow\quad 
			k_x = \cfrac{n_1\pi}{a_1} \\
		\sin k_y a_2 = 0 \quad\Rightarrow\quad 
			k_y = \cfrac{n_2\pi}{a_2} \\
		\sin k_z a_3 = 0 \quad\Rightarrow\quad 
			k_z = \cfrac{n_3\pi}{a_3} \\
	\end{array} \right.
\]

Подставляя значения \( k_x, k_y, k_z \) в уравнение \( [3] \) 
найдём собственную частоту:
\[
	\omega^2_{n_1 n_2 n_3} = c^2 \pi^2 \left[ 
		\left( \cfrac{n_1}{a_1} \right)^2 + 
		\left( \cfrac{n_2}{a_2} \right)^2 + 
		\left( \cfrac{n_3}{a_3} \right)^2 \right]
\]

Наименьшая частота будет при условии \( n_1 = n_2 = n_3 = 1 \)
\[
	\omega_{111} = \pm c\pi
		\frac{\sqrt{a^2_1 + a^2_2 + a^2_3}}{a_1 a_2 a_3}
\]

\end{document}
