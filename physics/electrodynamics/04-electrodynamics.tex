\input{../../.preambles/01-semester_work}
\input{../../.preambles/10-russian}
\input{../../.preambles/20-math}
\input{../../.preambles/30-physics}
\begin{document}
\maketitlepage{Факультет электроники и вычислительной техники}{физики}
{Электродинамика}{4}{}{студент группы Ф-369\\Голубев~А.~В.}
{}{доцент Грецов~М.~В.}{}{}

\newpage
\emph{Задача №3.32}: В диэлектрике имеется постоянное магнитное поле 
\( H_0 \). Исследовать монохроматическую волну, которая распространяется:
а) вдоль и б) поперёк магнитного поля. Диэлектрик считать прозрачным и 
немагнитным. 

\emph{Решение:}

\emph{Ответ:}

\newpage

\emph{Задача №3.51}: В однородном, прозрачном и немагнитном диэлектрике 
имеется постоянное магнитное поле \( H_0 \), направленное перпендикулярно 
к его поверхности. На эту поверхность в направлении магнитного поля \( H_0 \) 
падает в вакууме плоская монохроматическая и линейно поляризованная волна, 
интенсивность которой \( J_0 \). Определить интенсивность и состояние 
поляризации отраженной и прошедшей волн.

\emph{Решение:}

\emph{Ответ:}

\newpage

\emph{Задача №3.73}: Показать, что для распространяющейся в волноводе 
\( H \)--волны коэффициент поглощения
\[
	\alpha = \frac{c\kappa^2\theta'}{2k\omega\mu}
	\frac{\oint\limits_{l} 
		\left\{ |H_z|^2 + \frac{k^2}{\kappa^4}|\nabla_2 H_z|^2 \right\}dl
	}{\oint\limits_{S} |H_z|^2 dS}
\]

\emph{Решение:}

\emph{Ответ:}

\newpage

\emph{Задача №3.74}: Определить собственные электромагнитные колебания 
в полом (\(\eps = \mu = 1\)) резонаторе, имеющих форму прямоугольного 
параллелепипеда с идеально проводящими стенками, ребра которого равны 
\( a_1, a_2 \) и \( a_3 \). Найти наименьшую собственную частоту.

\emph{Решение:}

\emph{Ответ:}
\end{document}