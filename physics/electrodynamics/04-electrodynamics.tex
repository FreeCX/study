\input{../../.preambles/01-semester_work}
\input{../../.preambles/10-russian}
\input{../../.preambles/20-math}
\input{../../.preambles/30-physics}

\begin{document}
\maketitlepage{Факультет электроники и вычислительной техники}{физики}
{Электродинамика}{4}{}{студент группы Ф-369\\Голубев~А.~В.}
{}{доцент Грецов~М.~В.}{}{}

\newpage
\emph{Задача №3.32}: В диэлектрике имеется постоянное магнитное поле 
\( H_0 \). Исследовать монохроматическую волну, которая распространяется:
а) вдоль и б) поперёк магнитного поля. Диэлектрик считать прозрачным и 
немагнитным. 

\emph{Решение:}

\emph{Ответ:}

\newpage

\emph{Задача №3.51}: В однородном, прозрачном и немагнитном диэлектрике 
имеется постоянное магнитное поле \( H_0 \), направленное перпендикулярно 
к его поверхности. На эту поверхность в направлении магнитного поля \( H_0 \) 
падает в вакууме плоская монохроматическая и линейно поляризованная волна, 
интенсивность которой \( J_0 \). Определить интенсивность и состояние 
поляризации отраженной и прошедшей волн.

\emph{Решение:}

\emph{Ответ:}

\newpage

\emph{Задача №3.73}: Показать, что для распространяющейся в волноводе 
\( H \)--волны коэффициент поглощения
\[
	\alpha = \frac{c\kappa^2\theta'}{2k\omega\mu}
	\frac{\oint\limits_{l} 
		\left\{ |H_z|^2 + \frac{k^2}{\kappa^4}|\nabla_2 H_z|^2 \right\}dl
	}{\oint\limits_{S} |H_z|^2 dS}
\]

\emph{Решение:}

\emph{Ответ:}

\newpage

\emph{Задача №3.74}: Определить собственные электромагнитные колебания 
в полом (\(\eps = \mu = 1\)) резонаторе, имеющих форму прямоугольного 
параллелепипеда с идеально проводящими стенками, ребра которого равны 
\( a_1, a_2 \) и \( a_3 \). Найти наименьшую собственную частоту.

\emph{Решение:}

Воспользуемся волновым уравнение:
\[
	\nabla E + \frac{\omega^2}{c^2} E = 0 \eqno{1}
\]

где \( \nabla^2 = \ppder{}{x} + \ppder{}{y} + \ppder{}{z} \)

Граничные условия: \( E = 0 \) при 
\(
	x = 0, a_1; y = 0, a_2; z = 0, a_3 
\)

Представим решение уравнения \( [1] \) в виде:
\[
	E = A\sin\left( k_x x + \alpha_x \right)
		\sin\left( k_y y + \alpha_y \right)
		\sin\left( k_z z + \alpha_z \right) \eqno{2}
\]

Подставляя его в уравнение \( [1] \), получаем:
\[
	\frac{\omega^2}{c^2} = k^2_x + k^2_y + k^2_z \eqno{3}
\]

Подставляя граничные условия в уравнение \( [2] \), получаем:
\[
	\left\{ \begin{array}{ll}
		\sin\alpha_x = 0 \quad\Rightarrow\quad 
			\alpha_x = 0 \\
		\sin\alpha_y = 0 \quad\Rightarrow\quad 
			\alpha_y = 0 \\
		\sin\alpha_z = 0 \quad\Rightarrow\quad 
			\alpha_z = 0 \\
	\end{array} \right. \quad
	\left\{ \begin{array}{ll}
		\sin k_x a_1 = 0 \quad\Rightarrow\quad 
			k_x = \cfrac{n_1\pi}{a_1} \\
		\sin k_y a_2 = 0 \quad\Rightarrow\quad 
			k_y = \cfrac{n_2\pi}{a_2} \\
		\sin k_z a_3 = 0 \quad\Rightarrow\quad 
			k_z = \cfrac{n_3\pi}{a_3} \\
	\end{array} \right.
\]

Подставляя значения \( k_x, k_y, k_z \) в уравнение \( [3] \) 
найдём собственную частоту:
\[
	\omega^2_{n_1 n_2 n_3} = c^2 \pi^2 \left[ 
		\left( \cfrac{n_1}{a_1} \right)^2 + 
		\left( \cfrac{n_2}{a_2} \right)^2 + 
		\left( \cfrac{n_3}{a_3} \right)^2 \right]
\]

Наименьшая частоту будет при условии \( n_1 = n_2 = n_3 = 1 \)
\[
	\omega_{111} = \pm c\pi
		\frac{\sqrt{a^2_1 + a^2_2 + a^2_3}}{a_1 a_2 a_3}
\]

\end{document}