\documentclass[14pt,final,titlepage,pscyr]{hedwork}
\usepackage[russian]{babel}
\usepackage[utf8]{inputenc}
\usepackage[derivative]{hedmaths}
\usepackage{graphicx}

\graphicspath{{img/}}

\faculty{Факультет электроники и вычислительной техники}
\department{<<Физика>>}
\subject{Термодинамика и статистическая физика}
\variant{6}
\student[m]{студент группы Ф-469\\Голубев А. В.}
\teacher[m]{доцент Крючков С. В.}

\begin{document}
\maketitle
\emph{Задача №37}: Вычислите работу, совершенную киломолем газа при 
изотермическом расширении от объёма \( V_1 \) до объёма \( V_2 \), если 
состояние газа описывается: а) уравнением Клапейрона--Менделеева; б) уравнением 
Ван-дер-Ваальса.

\emph{Решение:}
а) имеем выражение \( pV = \nu RT = const \), тогда можно записать 
\( pV = p_1 V_1 \) откуда выражаем \( p = p_1 V_1 / V \) и подставляем в 
уравнение для работы \( dA = pdV \). Интегрируя выражение от \( V_1 \) до 
\( V_2 \) получаем:
\begin{equation}
	A = \int\limits_{V_1}^{V_2} pdV = \int\limits_{V_1}^{V_2} p_1 V_1 
		\frac{dV}{V} = \nu RT ln\frac{V_2}{V_1}
	\label{eq:mendel}
\end{equation}
б) имеем уравнение Ван-дер-Ваальса
\[
	\left( p + \frac{a\nu^2}{V^2} \right)\left( V - b\nu \right) = \nu RT
\]
откуда выражаем давление и подставляем в формулу для энергии и интергируем
\begin{equation}
	A = \int\limits_{V_1}^{V_2} pdV = \int\limits_{V_1}^{V_2} 
		\frac{\nu RT}{V-b\nu} dV - \int\limits_{V_1}^{V_2} 
		\frac{a\nu^2}{V^2} dV = \nu RT ln\frac{V_2 - b\nu}{V_1 - b\nu} - 
		a\nu^2\left( \frac{1}{V_1} - \frac{1}{V_2}\right)
	\label{eq:van-der}
\end{equation}
при \( a,b \rightarrow 0 \) уравнение \eqref{eq:van-der} переходит в 
\eqref{eq:mendel}
\newpage

\emph{Задача №71}: Стержень длиной \( l \) растягивается под действием силы 
\( f \). Найдите разность теплоёмкостей при постоянной длине и постоянном 
натяжении \( C_l - C_f \). Считать, что деформация является упругой и длина 
стержня зависит от натяжения и абсолютной температуры.

\emph{Решение:} Согласно первому началу термодинамики 
\begin{equation}
	\delta Q = dU + \delta A = dU - fdl
	\label{eq:71_first}
\end{equation}
Используя в качестве переменных \( l \) и \( T \), запишем полный дифференциал 
для \( dU \) и подставим в уравнение \eqref{eq:71_first}
\[
	\delta Q = \left( \pder{U}{T} \right)_l dT + \left[ 
		\left( \pder{U}{l} \right)_T - f\right]dl
\]
Получаем, что
\[
	C_l = \left( \pder{U}{T} \right)_l
\]
Исходя из условия, что \( l \) является функцией \( l = l(T,f) \), имеем
\[
	dl = \left( \pder{l}{T} \right)_f dT + \left( \pder{l}{f} \right)_T df
\]
Подставляя полученное в \eqref{eq:71_first}:
\[
	\delta Q = \left[ \left( \pder{U}{T} \right)_l + 
		\kappa\left( \pder{l}{T} \right)_f \right]dT + 
		\kappa\left( \pder{l}{f} \right)_T 
\]
где \( \kappa = \left( \pder{U}{l} \right)_T - f \). Тогда 
\begin{gather}
	C_f = \left( \pder{U}{T} \right)_l + \kappa\left( \pder{l}{T} \right)_f
	\nonumber \\ \text{ или } \nonumber \\ 
	C_f - C_l = \kappa\left( \pder{l}{T} \right)_f = 
		\left[ \left( \pder{U}{l} \right)_T - f\right]
		\left( \pder{l}{T} \right)_f \nonumber
\end{gather}

\newpage

\emph{Задача №101}: Эмпирическая формула, дающая зависимость молярной 
теплоёмкости \( C_p \) от температуры, для углекислоты в интервале между 
\( t_1 = -75^\circ C \) и \( t_1 = 20^\circ C \) имеет вид:
\[
	C_p = (8.71 + 6.6\cdot10^{-3}t - 2.2\cdot10^{-6}t^2) 
		\text{кал/(град}\cdot\text{моль)}
\]
Предполагая справедливым для \( CO_2 \) соотношения:
\[
	C_p - C_V = R \text{ и } \left( \pder{U}{V} \right)_T = 0
\]
найдите увеличение внутренней энергии при нагревании углекислого газа от 
\( t_1 \) до \( t_2 \).

\emph{Решение:} Рассмотрим внутреннюю энергию как функцию температуры и объёма: 
\( U = U(T,V) \). Запишем полный дифференциал 
\[
	dU = \left( \pder{U}{T} \right)_V dT + 
		\left( \pder{U}{V} \right)_T dV
\]
По определению молярной теплоёмкости \( C_V = \left( \pder{U}{T} \right)_V \)
получаем 
\begin{align*}
	\Delta U = \int\limits_{t_1}^{t_2} (C_p - R)dt = 8.71\cdot(t_2 - t_1) + 
		6.6\cdot10^{-3}\frac{1}{2}\cdot(t^2_2-t^2_1) - \\ -
		2.2\cdot10^{-6}\frac{1}{3}\cdot(t^3_2-t^3_1) - R\cdot(t_2-t_1) \approx 
		621 \left( \frac{\text{кал}}{\text{моль}} \right)
\end{align*}
\newpage

\emph{Задача №182}: Получите уравнение адиабаты газа, уравнение состояния 
которого имеет вид:
\[
	p = p_0\left( 1 + \alpha T - \beta V \right), C_p = const, 
	(p_0, \alpha, \beta \text{ -- постоянные})
\]

\emph{Решение:}
Имеем первое начало термодинамики 
\begin{equation}
	\delta Q = dU + pdV
	\label{eq:182_first}
\end{equation}
Запишем полный дифференциал функции \( U = U(T,V) \)
\begin{equation}
	dU = \left( \pder{U}{V} \right)_T dV + \left( \pder{U}{T} \right)_V dT
	\label{eq:diff_U}
\end{equation}
Запишем первое слагаемое в виде и подставим значение функции \( p \)
\[
	\left( \pder{U}{T} \right)_V = T\left( \pder{p}{T} \right)_V - p = 
		\alpha p_0 T - p_0 \cdot (1+\alpha T-\beta V) = p_0\cdot(1-\beta V)
\]
Тогда \eqref{eq:diff_U} можно записать с учётом 
\( \left( \pder{U}{T} \right)_V = C_V \) и выше сказанного:
\[
	dU = p_0\cdot(1-\beta V)dV + C_V dT
\]
Подставляя в \eqref{eq:182_first} получаем:
\[
	\delta Q = p_0\cdot(1-\beta V)dV + C_V dT + 
		p_0\cdot(1+\alpha T-\beta V)dV = C_V dT + \alpha p_0 T dV + p_0 dV
\]
Переходя к энтропии \( dS = \delta Q / T \) можно последнее уравнение 
переписать в виде:
\begin{gather}
	TdS = C_V dT + \alpha p_0 T dV + p_0 dV \nonumber \\
	dS = C_V \frac{dT}{T} + \alpha p_0 dV + \frac{p_0}{T} dV \Rightarrow 
	S = C_V\ln T + \alpha p_0 V + \frac{p_0 V}{T}
\end{gather}
И окончательно \( C_V \ln T + \alpha p_0 V = const \)
\newpage

\emph{Задача №369}: Найти среднюю потенциальную энергию молекулы идеального 
газа, находящегося в центрифуге радиуса \( R \), вращающейся с постоянной 
угловой скоростью \( \omega \).

\emph{Решение:} В системе координат, связанной с центрифугой молекула газа 
находится в поле центробежной силы инерции
\[
	F_\text{ин} = m\omega^2 r, \text{ где } r \text{ -- расстояние до 
		молекулы от оси вращения}
\]
Вычислим потенциальную энергию молекулы в таком поле:
\[
	m\omega^2 r = -\frac{dU}{dr} \Rightarrow 
	U = -\frac{m\omega^2 r^2}{2}
\]
Согласно распределению Больцмана вероятность обнаружить молекулы в точке с 
цилиндрическими координатами \( r, \phi, z \):
\[
	dW(r,\phi,z) = B\exp\left(-\frac{U}{k_0 T}\right) dV = B
		\exp\left(\frac{m\omega^2 r^2}{2k_0 T}\right) r dr d\phi dz
\]
Отсюда получаем функцию распределения по \( r \):
\[
	dW(r) = A\exp\left(\frac{m\omega^2 r^2}{2k_0 T}\right) r dr
\]
Используя условие нормировки для \( dW(r) \) найдём значение константы \( A \):
\[
	\int dW(r) = A\int\limits_{0}^{R} 
		\exp\left( \frac{m\omega^2 r^2}{2k_0 T} \right) r dr = 1
\]
Получаем
\[
	A = \frac{m\omega^2}{k_0 T}\left( 
		\exp\left[\frac{m\omega^2 R^2}{2k_0 T}\right] - 1 \right)
\]
Среднее значение потенциальной энергии можно определить в виде:
\begin{gather}
	<U> = \int U(r) w(r) dV = \int\limits_{0}^{R} U(r) dW(r) = 
	\nonumber \\ =
		-\frac{m^2 \omega^4}{2k_0 T}\left( 
		\exp\left[\frac{m\omega^2 R^2}{2k_0 T}\right] - 1 \right)
		\int\limits_{0}^{R} r^3 \exp\left[ 
		\frac{m\omega^2 r^2}{2k_0 T} \right] dr = \nonumber \\ =
		-k_0 T \frac{1 + 
			\left[\frac{m\omega^2 R^2}{2k_0 T} - 1 \right]
			\exp\left( \frac{m\omega^2 R^2}{2k_0 T} \right)}
			{\exp\left[\frac{m\omega^2 R^2}{2k_0 T}\right] - 1}
	\nonumber
\end{gather}
\newpage

\emph{Задача №386}: Найти выражение свободной энергии \( \mathcal{F} \) и 
энтропии \( S \) идеального газа при одномерном движении.

\emph{Решение:} Состояние идеального газа можно определить заданием всех точек 
системы. Так статистическая сумма \( Z \) переходит в интегральный виде и 
запишется в виде, с учётом одномерного движения:
\[
	Z = \frac{1}{N!} \int \Omega(E) \exp\left( -\frac{E}{kT} \right), 
	\text{ где } \Omega(E) = \frac{dpdq}{(2\pi\hbar)^N} \text{ из }
	\Delta p \Delta q = 2\pi\hbar
\]
Запишем энергию для \( i \)-ой частицы в виде:
\[
	E(p,q) = \frac{p_{ix}^2}{2m}
\]
С учётом всего вышеперечисленного, при условии независимости одной части от 
другой, система преобразуется к виду
\[
	Z_i = \frac{1}{N!}\frac{1}{(2\pi\hbar)^2} \prod\limits_{i=1}^{N}
		\left\{ \int dx_i \int dp_{xi} 
			\exp\left( -\frac{p_{ix}^2}{2mkT} \right) 
		\right\}
\]
В силу независимости частиц, можно статистическую сумму переписать в виде:
\[
	Z = \frac{1}{N!}\frac{L^N}{(2\pi\hbar)^N} 
		\left\{ 
			\int\limits_{-\infty}^{+\infty} 
			\exp\left( -\frac{p_{x}^2}{2mkT} \right) dp_x
		\right\}^N = 
		\frac{L^N}{N!} 
		\left\{
			\frac{\sqrt{2\pi mkT}}{2\pi\hbar} 
		\right\}^N
\]
где \( L \) -- линейные размеры области в направлении движения. Для нахождения 
свободной энергии воспользуемся известным равенством:
\[
	F = -kT\ln Z
\]
\begin{gather}
	\ln Z = N\ln L + \frac{N}{2}\ln T + \frac{N}{2}\ln(2\pi m k) - \ln N\! - 
		N\ln(2\pi\hbar) \nonumber \\
	F = -kT\ln Z = -\frac{NkT}{2}\ln T - NkT\ln L - \frac{NkT}{2}\ln(2\pi km) + 
		kT\ln N'! + NkT\ln(2\pi\hbar) \nonumber
\end{gather}
Для нахождения энтропии можно воспользоваться следующим выражением
\[
	S = -\frac{\partial F}{\partial T}
\]
\[
	S = Nk\ln L + \frac{Nk}{2}\ln T + \frac{Nk}{2}\ln(2\pi km) - k\ln N\! - 
		Nk\ln(2\pi\hbar) + \frac{Nk}{2}
\]

\newpage

\emph{Задача №420}: Найти свободную энергию и энтропию для системы из \( N \) 
независимых линейных осцилляторов.

\emph{Решение:} Рассмотрим систему состоящую из \( N \) гармонических 
осцилляторов без взаимодействия. Определим свободную энергию \( F_i \) для 
\( i \)-го осциллятора, имеем
\[
	F = \sum\limits_{i=1}^{N} F_i \text{ и соответственно стат.сумма. }
	Z_i = \sum\limits_{n} \exp\left( -\frac{E_n^i}{kT} \right)
\]
Здесь статистическая сумма записана без статистического веса, так как нет 
вырождения по энергии. \( E_n^i \) согласно квантовой механике для 
\( i \)-го осциллятора имеет вид:
\[
	E^i_n = \hbar\omega_i \left( n + \frac{1}{2} \right), n = 0, 1, \ldots
\]
Тогда
\[
	Z_i = \sum\limits_n \exp\left( -\frac{h\omega_i
		\left( n + \frac{1}{2} \right)}{kT} \right)
\]
с учётом
\[
	\frac{1}{1-\exp\left( -\cfrac{\hbar\omega_i}{kT} \right)} = 1 + 
		\exp\left( -\frac{\hbar\omega_i}{kT} \right) + 
		\exp\left( -\frac{2\hbar\omega_i}{kT} \right) + \ldots
\]
сумму можно переписать в виде:
\[
	Z_i = \frac{\exp\left( -\cfrac{\hbar\omega_i}{2kT} \right)}
		{1-\exp\left( -\cfrac{\hbar\omega_i}{kT} \right)}
\]
Для нахождения свободной энергии для \( i \)-го осциллятора воспользуемся 
известным равенством:
\begin{equation}
	F_i = -kT\ln Z_i
	\label{eq:free_energy}
\end{equation}
для этого найдём значение \( \ln Z_i \):
\[
	\ln Z_i = -\frac{\hbar\omega_i}{2kT} - \ln\left( 1-\exp\left\{ 
		-\frac{\hbar\omega_i}{kT}\right\}\right)
\]
подставляя в уравнение \eqref{eq:free_energy} и собирая свободную энергию 
по всем осцилляторам получим:
\[
	F = \sum\limits_{i=1}^{N} F_i = \frac{N\hbar\omega}{2} + 
		NkT\ln\left( 1-\exp\left\{ -\frac{\hbar\omega}{kT}\right\}\right)
\]
Для нахождения энтропии можно воспользоваться следующим выражением
\[
	S = -\frac{\partial F}{\partial T}
\]
\[
	S = Nk\left[ \frac{\hbar\omega}{kT}
		\frac{1}{\exp\left\{-\cfrac{\hbar\omega}{kT}\right\}-1} - 
		\ln\left( 1-\exp\left\{ -\frac{\hbar\omega}{kT}\right\}\right) 
	\right]
\]

\newpage

\emph{Задача №480}: Пользуясь распределением Гиббса для системы с переменным 
числом частиц, выразить \( \overline{(\Delta N)^2} \) через 
\( \left( \partial N / \partial \eta \right)_{T,V} \) (где \( \eta \) -- 
химический потенциал).

\emph{Решение:} Распределение Гиббса для системы с переменным числом частиц 
имеет вид:
\[
	W = \exp\left( \frac{\Omega + N\eta - \eps_n N_n}{\Theta} \right)
\]
где \( \eps_n \) -- энергия одной частицы на \( n \)-ом уровне, \( \Omega \) 
определяется из условия нормировки:
\begin{equation}
	\exp\left( \frac{\Omega}{\theta} \right) 
		\sum\limits_N \exp\left( \frac{N\eta}{\Theta} \right)
		\sum\limits_n \exp\left( -\frac{\eps_n N_n}{\Theta} \right) = 1 
	\label{eq:480_2}
\end{equation}
Среднее значение для числа частиц определяется равенством:
\begin{equation}
	\overline{N} = \exp\left( \frac{\Omega}{\Theta} \right)
		\sum\limits_N N\exp\left( \frac{N\eta}{\Theta} \right)
		\sum\limits_n \exp\left( -\frac{\eps_n N_n}{\Theta} \right)
	\label{eq:480_3}
\end{equation}
Дифференцируя \eqref{eq:480_3} по \( \eta \), с учётом зависимости 
\( \Omega \) от \( \eta \), получим:
\begin{gather}
	\left( \pder{\overline{N}}{\eta} \right)_{T,V} = \frac{1}{\Theta}
		\exp\left( \frac{\Omega}{\Theta} \right)
		\sum\limits_N \left( N^2 + N\pder{\Omega}{\eta} \right)
		\exp\left( \frac{N\eta}{\Theta} \right)
		\sum\limits_n \exp\left( -\frac{\eps_n N_n}{\Theta} \right) = 
		\nonumber \\
		= \Big[ \text{ с учётом определения \eqref{eq:480_3} } \Big] =
		\frac{1}{\Theta} \left( \overline{N^2} + 
			\overline{N}\pder{\Omega}{\eta} \right)
	\label{eq:480_4}
\end{gather}
Из \eqref{eq:480_2} путём дифференцирую по \( \eta \) аналогично найдём:
\begin{equation}
	\pder{\Omega}{\eta} = -\overline{N}
	\label{eq:480_5}
\end{equation}
Тогда подставляя \eqref{eq:480_5} в \eqref{eq:480_4} получаем:
\[
	\left( \pder{\overline{N}}{\eta} \right)_{T,V} = 
		\frac{1}{\Theta} \left( \overline{N^2} - \overline{N}^2 \right) = 
		\frac{1}{\Theta} \overline{\Delta N^2} \quad \Rightarrow \quad
	\overline{(\Delta N)^2} = \Theta 
		\left( \pder{\overline{N}}{\eta} \right)_{T,V} 
\]

\end{document}
