\documentclass[14pt,final,titlepage,pscyr]{hedwork}
\usepackage[russian]{babel}
\usepackage[utf8]{inputenc}
\usepackage{hedmaths}
\usepackage{graphicx}

\graphicspath{{img/}}

\faculty{Факультет электроники и вычислительной техники}
\department{<<Физика>>}
\subject{Статистическая радиофизика и обработка сигналов}
\variant{6}
\student[m]{студент группы Ф-469\\Голубев А. В.}
\teacher[m]{доцент Поляков И. В.}

\begin{document}
\maketitle
\emph{Задача №6:} Определить непосредственным интегрированием \\
\( 
    m_k(\xi) = <x^k> = \int\limits_{-\infty}^{\infty} x^k w(x) dx
\) начальные моменты непрерывных случайных величин, имеющих закон распределения
Накагами (\( m \)-распределения)
\[
    w(x) = \frac{2m^m x^{2m-1}}{m_2^m \Gamma(m)}
        \exp\left( -\frac{m}{m_2}x^2 \right), x \geq 0, m \geq \frac{1}{2}
\]

\emph{Решение:}
\begin{gather}
    <x^k> = \int\limits_{0}^{+\infty} x^k 
        \frac{2m^m}{m_2^m \Gamma(m)}
        \exp\left( -\frac{m}{m_2}x^2 \right) dx = \nonumber \\
    = \frac{2m^m x^{2m-1}}{m_2^m \Gamma(m)}\left.\left\{ 
        -\frac{1}{2} x^{2m-k}\left[ 
            \frac{mx^2}{m_2}
        \right]^{\frac{1}{2}(k-2m)}
        \gamma\left( m-\frac{k}{2}, \frac{mx^2}{m_2} \right)
    \right\}\right|_{0}^{+\infty} \nonumber
\end{gather}
где \( \gamma \) -- неполная \( \Gamma \)-функция. Окончательно получаем:
\[
    <x^k> = \left( \frac{m_2}{m} \right)^{k/2} 
        \frac{\Gamma\left(m+\frac{1}{2}\right)}{\Gamma(m)}
\]

\emph{Задача №16:} Найти плотность вероятности и начальные моменты случайной 
величины \( \eta = e^\xi \), где \( \xi \) -- случайная гауссовская величина 
с математическим ожиданием \( a \) и дисперсией \( \sigma^2 \).

\emph{Решение:} Плотность вероятности случайной величины \( \eta = e^\xi \)
\[
    w(\eta) = \frac{d\eta}{d\xi} = e^{\xi} = e^{\ln\eta}
\]
Найдём моменты случайной величины \( \eta = e^\xi \)
\begin{gather}
    \mu_1 = \int e^\xi \xi d\xi = \xi e^\xi - \int e^\xi d\xi = 
        e^\xi( \xi - 1 ) = a \nonumber \\
    \mu_2 = \int e^\xi \xi^2 d\xi = \xi^2 e^\xi - 2\mu_1 = \sigma^2 
        \nonumber \\
    \mu_3 = \xi^3 e^\xi - 3\mu_2 \text{ и т.д.} \nonumber
\end{gather}
Обобщая формулу для любого \( n \), получим:
\[
    \mu_n = e^\xi \left[ 
        \xi^n - \sum\limits_{i=1}^{n-1}(n-i+1)!\xi^i - n! 
    \right]
\]

\newpage

\emph{Задача №33:} На вход идеальной дифференцирующей цепи подан стационарный 
гауссовский случайный сигнал \( x(t) \) с нулевым математическим ожиданием и 
корреляционной функцией 
\( R_x(\tau) = D_x e^{-\alpha|\tau|}(1 + \alpha|\tau|) \). Определить 
корреляционную функцию \( R_y(\tau) \) сигнала \( y = \cfrac{dx}{dt} \) на 
выходе.

\emph{Решение:} Корреляционную функцию на выходе можно записать в виде:
\begin{equation}
    R_y(\tau) = -\frac{d^2 R_x(\tau)}{d\tau^2}
    \label{eq:01}
\end{equation}
Рассмотрим один из интервалов \( \tau > 0 \)
\begin{gather}
    R_x(\tau) = D_x e^{-\alpha\tau}(1+\alpha\tau) \nonumber \\
    R'_x(\tau) = -D_x e^{-\alpha\tau}(\alpha-\alpha(1+\alpha\tau)) = 
    -\alpha^2\tau D_x e^{-\alpha\tau} \nonumber \\
    R''_x(\tau) = -\alpha^2 D_x e^{-\alpha\tau} + \alpha^3 \tau 
        e^{-\alpha\tau} = -\alpha^2 D_x e^{-\alpha\tau} ( 1 - \alpha\tau ) 
    \nonumber
\end{gather}
Для второго интервала получается аналогичные уравнения с точность до знака 
перед \( \tau \), поэтому можно записать ответ воспользовавшись \eqref{eq:01}
\[
    R_y(\tau) = \alpha^2 D_x e^{-\alpha|\tau|} ( 1 - \alpha|\tau| )
\]

\emph{Задача №43:} Случайны стационарный процесс \( \xi(t) \) с корреляционной 
функцией 
\( 
    R_\xi(\tau) = \sigma_\xi^2 e^{-\alpha|\tau|} \left( 
        \cos\omega_0\tau + \cfrac{a}{\omega_0}\sin\omega_0|\tau| 
    \right)
\) записан в виде \\ 
\( \xi(t) = A_c(t)\cos\omega_0 t - A_s(t)\sin\omega_0 t \). 
Вычислить спектральную плотность \( S_c(\omega), S_s(\omega) \), а так же 
взаимную спектральную плотность \( S_{cs}(\omega) \).

\emph{Решение:} Так как процесс является стационарным, то справедливы 
следующие уравнения:
\begin{gather}
    R_c(\tau) = R_s(\tau) = \sigma_\xi^2 e^{-\alpha|\tau|} \nonumber \\
    R_{cs}(\tau) = \sigma_\xi^2 e^{-\alpha|\tau|} \cdot 
        \mathrm{sgn}(\tau) \nonumber
\end{gather}
\begin{gather}
    S_c(\omega) = S_s(\omega) = \cfrac{1}{2\pi} \int\limits_{-\infty}^{+\infty}
        R_c(\tau) e^{-j\omega\tau} d\tau = \cfrac{\sigma_\xi^2}{\pi}
        \cfrac{\alpha}{\alpha^2 + \omega^2} \nonumber \\
    S_{cs}(\omega) = \cfrac{1}{2\pi} \int\limits_{-\infty}^{+\infty}
        R_{cs}(\tau) e^{-j\omega\tau} d\tau = \cfrac{\sigma_\xi^2}{2\pi}
        \cfrac{a}{\omega_0} \left\{ 
            -\int\limits_{-\infty}^{0} e^{+(\alpha+j\omega)\tau} d\tau
            +\int\limits_{0}^{+\infty} e^{-(\alpha-j\omega)\tau} d\tau
        \right\} = \nonumber \\
        \cfrac{\sigma_\xi^2}{2\pi}\cfrac{a}{\omega_0} \left[ 
            \cfrac{1}{\alpha-j\omega} - \cfrac{1}{\alpha+j\omega}
        \right] = \cfrac{\sigma_\xi^2}{2\pi}\cfrac{a}{\omega_0} \left[ 
            \cfrac{2j\omega}{\alpha^2 + \omega^2}
        \right] \nonumber
\end{gather}

\end{document}
