\input{../../.preambles/01-semester_work}
\input{../../.preambles/10-russian}
\input{../../.preambles/20-math}
\input{../../.preambles/30-physics}

\begin{document}

\maketitlepagewithvariant{Факультет электроники и вычислительной техники}{физики}
{Физика атома}{студент группы Ф-369\\Голубев~А.~В.}
{Ерёмин~А.~В.}{№1}{8}

\newpage

\emph{Задача 6.238: }
Найти уравнение адиабатического процесса (в переменных \( V \), \( T \) ), 
проводимого с тепловым излучением, имея ввиду, что между давлением и плотностью 
энергии теплового излучения существует связь \( P = u/3 \).

\emph{Решение:}

В данной задаче будем считать излучение чернотельным. Запишем объёмную
плотность энергии чернотельного излучения:
	\[ u = \frac{4\sigma T^4}{c} \]
Энергия теплового излучения будет выражаться, через формулу:
	\[ U = uV = \frac{4\sigma VT^4}{c} \]
где \( u \) -- плотность энергии излучения, \( V \) -- объём полости, 
T -- термодинамическая температура стенок этой полости, 
\( \sigma \) -- постоянная Стефана-Больцмана, \\ \( c \) -- скорость света. \\
Для адиабатического процесса
	\[ dU + PdV = 0 \]
где p -- давление излучения, dV -- приращение объёма полости. \\
По условию задачи:
	\[ P = \frac{u}{3} = \frac{4\sigma T^4}{3c} \]
Подставляя полученные формулы для \( U \) и \( P \) в \( dU + PdV = 0 \), получим:
	\[ d(VT^4) + \frac{1}{3}T^4 dV = 0 \]
откуда получаем уравнение:
	\[ 3\frac{dT}{T} + \frac{dV}{V} = 0 \Rightarrow dln(VT^3) = 0 \]
Окончательно получаем:
	\[ VT^3 = const \] 

\emph{Ответ:} \( VT^3 = const \)

\newpage

%--------------------------------------------------------------------------------

\emph{Задача 1.29: }
Лазер излучает в импульсе длительностью \( \tau = 0.13 \) мс узкий пучок света 
с энергией \( E = 10 \) Дж. Найти среднее за время \( \tau \) давление такого 
пучка света, если его сфокусировать в пятнышко диаметром \( d \) -- 10 мкм 
на поверхности, перпендикулярной пучку, с коэффициентом отражения \( \rho = 0.50 \).

\emph{Решение:}

Среднее давление определяется формулой:
	\[ P = \frac{F}{S} \]
где \( F = \cfrac{\Delta p}{\Delta t} \), 
	\( \Delta p \) -- изменение импульса за время \( \Delta t \).
Из закона сохранения импульса для одного фотона следует:
	\[ 
		\vec{p_1} = \vec{p_2} + \vec{p_3} \Rightarrow 
		\vec{p_3} = \vec{p_1} - \vec{p_2}
	\]
-- импульс, переданный фотоном за время \( \tau \). Учитывая,
что импульс отраженного фотона равен \( p_2 = \rho p_1 \),
получим:
	\[ p_3 = p_1 + \rho p_1 = p_1(1+\rho) \]
Импульс, переданный всеми фотонами равен изменению импульса 
\( \Delta p \) за время \( \tau \), то есть
	\[ \Delta p = np_3 = np_1(1+\rho) \]
где \( n \) -- число фотонов, \( p_1 = \hbar\omega/c \) --
импульс падающего фотона. Таким образом, сила
	\[ 
		F = \frac{n\hbar\omega}{c\tau}(1+\rho) 
		  = \frac{E}{c\tau}(1+\rho)
	\]
Среднее давление
	\[ P = \frac{4E}{cd^2\tau\pi}(1+\rho) = 5\text{ МПа} \]

\emph{Ответ:} \( P = \) 5 МПа

\newpage

%--------------------------------------------------------------------------------

\emph{Задача 2.19: }
Дифференциальное сечение рассеяния \( \alpha \)-частиц кулоновским полем неподвижного 
ядра \( d\sigma/d\Omega = 7.0\cdot10^{-22} \text{ см}^2/\text{ср} \)
для угла \( \theta_0 = 30^{\circ} \). Вычислить среднее сечение рассеяния 
\( \alpha \)-частиц в интервале углов \( \theta > \theta_0 \).

\emph{Решение:}

Запишем формулу Резерфорда:
\[ 
	\frac{d\sigma}{d\Omega} = 
	\left(\frac{ke^2z}{mv^2}\right)^2\cdot\frac{1}{\sin^4\frac{\theta}{2}}
\]
Преобразуем формулу, сделав замену \( C = \left(\cfrac{ke^2z}{mv^2}\right)^2 \):
\[ 
	C = \frac{d\sigma}{d\Omega}\cdot\sin^4\frac{\theta}{2}
\]
В случае угла \( \theta_0 \), мы можем определить эту константу.\\
Проинтегрируем формулу Резерфорда от \( \theta_0 \) до \( \pi \):
\[
	\Delta\sigma = C\int_{\theta_0}^{\pi}\frac{2\pi\sin\theta d\theta}
	{\sin^4\frac{\theta}{2}} = \pi C\ctg^2\frac{\theta_0}{2}
\]
Подставляя значение константы \( C \) при \( \theta_0 \), получим:
\[
	\Delta\sigma = \frac{d\sigma}{d\Omega}\cdot\pi\cdot\sin^4\frac{\theta_0}{2}
	\cdot\ctg^2\frac{\theta_0}{2}
\]

\emph{Ответ:}
\[
	\Delta\sigma = \frac{d\sigma}{d\Omega}\cdot\pi\cdot\sin^4\frac{\theta_0}{2}
	\cdot\ctg^2\frac{\theta_0}{2}
\]

\newpage

%--------------------------------------------------------------------------------

\emph{Задача 2.36: }
Энергия связи электрона в атоме \textbf{He} равна \( E_0 = 24.6 \) эВ. Найти энергию,
необходимую для удалению обоих электронов из этого атома.

\emph{Решение:}

Запишем обобщённую формулу Бальмера:
	\[ \omega = z^2 R(\frac{1}{n^2}-\frac{1}{m^2}) \]
где \( R \) -- постоянная Ридберга.\\
Энергия, необходимая для удаления двух электронов из атома будет складываться из 
энергии связи и энергии ионизации, т.е.:
	\[ E = E_0 + \hbar\omega \]
Подставляем в последнее уравнение, формулу Бальмера, 
с учётом того, что атом находится в основном состоянии \( m \rightarrow \infty \) и
\( z \) атома \textbf{He} равен \( 2 \), получаем:
	\[ E = E_0 + 4\hbar R \]

\emph{Ответ:} \( E = E_0 + 4\hbar R \)

\newpage

%--------------------------------------------------------------------------------

\emph{Задача 3.51:}
Частица массой \( m \) находится в некотором одномерном \\
потенциальном поле \( U(x) \) в стационарном состоянии, для которого волновая 
функция имеет вид: \( \psi(x) = Aexp(-\alpha x^2) \), где \( A \) и 
\( \alpha \) -- заданные постоянные. Имея ввиду, что \( U(x) = 0 \) при 
\( x = 0 \), найти \( U(x) \) и энергию \( E \) частицы.

\emph{Решение:}

Одномерный случай уравнения Шрёдингера в стационарном виде: 
	\[ 
		\frac{\partial^2\psi}{\partial x^2} + 
		\frac{2m}{\hbar^2}(E-U(x))\psi = 0\quad\quad(1)
	\]
Найдём значение второй производной функции \( \psi(x) \):
	\[ \psi'_{x} = -2A\alpha x exp(-\alpha x^2) \]
	\[ \psi''_{xx} = 2A\alpha exp(-\alpha x^2)(2\alpha x^2 - 1) \]
и подставим в уравнение (1):
	\[ 2A\alpha exp(-\alpha x^2)(2\alpha x^2 - 1) 
	   + \frac{2m}{\hbar^2}(E - U(x))Aexp(-\alpha x^2) = 0 
	\]
или после небольшого преобразования, получаем формулу (2):
	\[ 2\alpha(2\alpha x^2 - 1) + \frac{2m}{\hbar^2}(E - U(x)) = 0 \]
Найдем значение \( E \) при \( U(0) = 0 \) (\( x = 0 \)). \\
Подставляем в уравнение значение \( U \) и \( x \), получаем 
	\[ E = \frac{\alpha\hbar^2}{m} \]
Теперь найдём функцию \( U(x) \), подстановкой значения \( E \) в уравнение (2):
	\[ U(x) = \frac{2\alpha^2\hbar^2}{m}x^2 \]

\emph{Ответ:} \( U(x) = \cfrac{2\alpha^2\hbar^2}{m}x^2 \), 
	\(E = \cfrac{\alpha\hbar^2}{m} \)

\newpage

%--------------------------------------------------------------------------------

\emph{Задача 5.17:}
Сколько различных типов термов возможно у двухэлектронной системы, состоящей из
\( d- \) и \( f- \)электронов?

\emph{Решение:}\\
Для d-электрона: \( l_1 = 2;\quad s_1 = \frac{1}{2} \) \\
Для f-электрона: \( l_2 = 3;\quad s_2 = \frac{1}{2} \) \\
	\[ S = s_1 + s_2, s_1 - s_2 = 1, 0 \]
	\[ L = l_1 + l_2, ..., |l_1 - l_2| = 5, 4, 3, 2, 1 \]
	\[ J = L+S, L, L-S \]
Для синглета: \( J = \{ L+0, L, L-0 \} \rightarrow \) 5 линий
\[ J = 5, 4, 3, 2, 1 \]
Для триплета: \( J = \{ L+1, L, L-1 \} \rightarrow \) 15 линий
\[ J = 6, 5, 4, 5, 4, 3, 4, 3, 2, 3, 2, 1, 2, 1, 0 \]

\emph{Ответ:} 20 линий (5 синглетных + 15 триплетных)

\newpage

%--------------------------------------------------------------------------------

\emph{Задача 5.206:}
Некоторый атом находится в состоянии, для которого \\\( S = 2 \), полный 
механический момент \( M = \hbar\sqrt{2} \), а магнитный момент равен нулю. 
Написать спектральный символ соответствующего терма.

\emph{Решение:}

Найдём значение квантового числа \( J \) по формуле:
	\[ M = \hbar\sqrt{2} = \hbar\sqrt{J(J+1)} \]
откуда \( J = 1 \). Запишем уравнение для магнитного момента:
	\[ \mu = g\mu_{\text{Б}}\sqrt{J(J+1)} = 0\]
откуда получаем, что фактора Ланде \( g = 0 \). \\
Найдём значение L используя формулу для определения фактора Ланде:
	\[ 
		0 = 1 + \frac{J(J+1)+S(S+1)-L(L+1)}{2J(J+1)} =
		1 + \frac{1\cdot2+2\cdot3-L(L+1)}{2\cdot2} 
	\]
и сделав математические преобразования:
	\[ L^2+L - 12 =0 \]
решая квадратное уравнение получаем значение L = 3. \\
Для определения символа терма используем формулу 
\( ^{2S+1}\large{\text{L}}_{J} \),
где \(\large{\text{L}} = \{s, p, d, f, ...\} \). \\
Окончательно получаем: \( ^{5}\text{F}_{1} \).\\

\emph{Ответ:} \( ^{5}\text{F}_{1} \)

\end{document}