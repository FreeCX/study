\documentclass[a4paper, 14pt]{extarticle}
\usepackage[utf8]{inputenc}
\usepackage[paper=a4paper, top=1cm, right=1cm, bottom=1.5cm, left=2cm]{geometry}
\usepackage{setspace}
\usepackage[russian]{babel}
\usepackage[T2A]{fontenc}
\usepackage{indentfirst}
\usepackage{pscyr}
\onehalfspacing

\usepackage{graphicx}

\parindent=1.25cm

\usepackage{titlesec}

\usepackage{caption}
\captionsetup{labelsep=period}


\titleformat{\section}
    {\normalsize\bfseries}
    {\thesection}
    {1em}{}

\titleformat{\subsection}
    {\normalsize\bfseries}
    {\thesubsection}
    {1em}{}

% Настройка вертикальных и горизонтальных отступов
\titlespacing*{\chapter}{0pt}{-30pt}{8pt}
\titlespacing*{\section}{\parindent}{*4}{*4}
\titlespacing*{\subsection}{\parindent}{*4}{*4}

\usepackage[square, numbers, sort&compress]{natbib}
\makeatletter
\bibliographystyle{unsrt}
\renewcommand{\@biblabel}[1]{#1.} 
\renewcommand{\theenumi}{\arabic{enumi}}
\renewcommand{\labelenumi}{\arabic{enumi}.}
\renewcommand{\theenumii}{.\arabic{enumii}}
\renewcommand{\labelenumii}{\arabic{enumi}.\arabic{enumii}.}
\renewcommand{\theenumiii}{.\arabic{enumiii}}
\renewcommand{\labelenumiii}{\arabic{enumi}.\arabic{enumii}.\arabic{enumiii}.}
\renewcommand{\baselinestretch}{1.5}
\makeatother


\newcommand{\maketitlepage}[6]{
    \begin{titlepage}
        \singlespacing
        \newpage
        \begin{center}
            Министерство образования и науки Российской Федерации \\
            Федеральное государственное бюджетное образовательное \\
            учреждение высшего профессионального образования \\
            <<Волгоградский государственный технический университет>> \\
            #1 \\
            Кафедра #2
        \end{center}


        \vspace{14em}

        \begin{center}
            \large Реферат по дисциплине
            \\ #3
        \end{center}

        \vspace{5em}

        \begin{flushright}
            \begin{minipage}{.3\textwidth}
                Выполнил:\\#4
                \vspace{1em}\\
                Проверил:\\#5
                \\
                \\ Оценка \underline{\ \ \ \ \ \ \ \ \ \ \ \ \ \ \ \ }
            \end{minipage}
        \end{flushright}

        \vspace{\fill}

        \begin{center}
            Волгоград, 2013
        \end{center}

    \end{titlepage}
    \setcounter{page}{2}
}
\input{../../.preambles/10-russian}

\begin{document}
\maketitlepage{Факультет электроники и вычислительной техники}
{<<Электронно-вычислительные машины и системы>>}
{Микроэлектроника и схемотехника \\ <<Типы памяти>>}
{студент группы Ф-369\\Голубев~А.~В.}{Черных~Д.~А.}{}

На текущий момент развития технологий мы настолко привыкли к тому, что в 
память современных цифровых устройствах реализована на полупроводниковых 
элементах, что и не задумываемся как в будущем, всё может измениться, и 
конденсаторы с транзисторами, составляющие основу ячеек современной 
оперативной и флэш-памяти, уступят своё место. 

Современная полупроводниковая технология - это компромисс, навязанный нам 
производителем микроэлектроники. Наверное, нет ничего хуже, чем формировать 
значение двоичной единицы, загнав толпу таких энергетичных созданий, как 
электроны, в ловушку конденсаторов или транзисторных затворов. Мало того, 
эти электроны несмотря ни на какие затворы стараются утечь из ячейки, что 
в модулях оперативной памяти периодической перезаписи ячеек, так, выбегая 
из неё на свобуду, они норовят нагреть всё вокруг себя своей энергией. 
Если рассматривать текущих фаворитов рынка флеш памяти, то они используют 
большой импульс силы, для того чтобы загнать электроны под затвор 
транзистора-ячейки, что частично её разрушает. То есть эта технология 
ограничена количеством циклов перезаписи, тем самым ставя под вопрос 
надёжность SSD технологии.

Тем временем, со времён разработки первых цифровых ЭВМ инженеры 
использовали магнитное взаимодействие. Так было введено в использование 
прототипы памяти с произвольным доступом. Идея была проста: магнитное поле 
хранит бит информации, принцип электромагнитной индукции извлекает этот 
бит в виде импульса индукционного тока. 

Было проведено много экспериментов, на поиск более эффективных 
материалов, способные хранить информацию в виде остаточной намагниченности 
и способы их преобразования в поток электронов.

Результатом этих исследований стала память на магнитных сердечниках, где 
ячейкой хранения выступало кольцо из магнитно-твёрдого вещество феррита, в 
химической основе которого лежат разные соединения оксида железа.

Очень важная особенность феррита является практически прямоугольная 
петля магнитного гистирезиса. Её верхняя граница соответствует остаточной 
намагниченности кольца, которое используется в качестве логической 
единицы, граница противоположной остаточной намагниченности соответствует 
логическому нулю. 

Фактически, модуль такой памяти представлял собой полотно из черырёх 
переплетенных между собой проводов, ответственных за возбуждение 
магнитного поля разной направленности, считывание данных и запрета (в 
случае записи в ячейку логического нуля). 

Ферритовые кольца располагались в перекрестье этих проводов, образовывая 
подобие высокотехнологичной кольчуги. И главной проблемой являлась 
сложность плетения этой кольчуги, не учитывая поддержание определенной 
температуры (обычно высокой). 

Очевидно, что для памяти большого объёма 
нужно больше ячеек, что подразумевает штамповку большого числа колец и 
сложную процедуру их вплетения в провода. При этом делать такую ферритовую 
память в виде гигантского гобелена было и технически и экономически 
нецелесообразно. В процессе плетения модулей и в процессе сборки 
ферритовых кубов вкрадывались ошибки (работа была практически ручная), что 
приводило к увеличению времени отладки и устранения неполадок.

В поисках компромиссного решения инженеры решили заменит кольца 
ферритовыми пластинами. В таких пластинах идея ферритового кольца была 
возведена в абсолют. По сути, вся поверхность пластины была 
ферритовым кольцом с множеством отверстий, сквозь которые продевались 
управляющие провода. Процесс изготовления памяти на ферритовых 
пластинах был несколько проще. Но, всё-таки, он являлся тем же методом 
плетения памяти-кольчуги. 

Благодаря проблеме трудоёмкости разработки памяти на ферритовых кольцах 
у сотрудника лаборатории Bell Labs Эндрю Бобека появилась возможность 
проявить свой изобретательский талант.

Бобек решил кардинально изменить направление исследований и предложить 
альтернативу памяти на ферритовых кольцах. Первым вопросом по которому 
нужно задуматься: <<обязательно ли использовать в ферритоподобные 
материалы?>>. Так как не только у них подходящая реализация памяти и 
петли гистерезиса. В технике давно были известны магнито-мягкие сплавы 
обладающие подходящими свойствами. В первую очередь сплав железа с 
никелем (пермаллой), железо с кобальтом (пермендюр) и железа с кремнием 
(трансформаторная сталь).

Начав экспериментировать с пермаллоем Бобек дошёл до идеи навивать фольгу 
из этого материала на несущий провод под необходимым для правильного 
намагничивания углом в сорок пять градусов. Он назвал такой провод 
твистор-кабелем. 

Навив подобным образом ленту пермаллоя на достаточно длинный провод, его 
можно будет свернуть так, чтобы создать зигзагообразную 
матрицу параллельны твистор-кабелей. Потом эту матрицу можно запаковать, 
например, в полиэтиленовую плёнку, и массив пермаллоевых псевдоколец 
продетых через один из несущих проводов уже есть. Второй провод Бобек 
предложил заменить медной шиной, на который укладывался запакованная в 
полиэтилен матрица твистор-каблей. На пересечениях шины и твистор-кабеля 
раполагались небольшие постоянные магниты, поддерживаюшие необходимое 
магнитное поле. 

Этим предложением Бобек фактически решил проблему создания сколь угодно 
больших по объёму массивов памяти. Так как длинную полиэтиленовую ленту, с 
впаянными в неё твисторами, можно компактно свернуть гармошкой, перемежая 
слои медными шинами. 

Уникальная особенность твистор-кабеля являлась возможность 
чтения или записи целой строки пермаллоевых псевдоколец, находящихся на 
параллельных твистор-кабелях, проходящих над одной шиной. Это существенно 
упрощало конструкцию модуля твистор памяти по сравнению с памятью на 
кольца, лишая её дополнительных проводов запрета. 

Правда, без фирритовых колец в твистор памяти не обошлось. Закрепленные 
на каждой из медных шин, они играют роль соленойда, передающего 
индукционный ток на адресные кабели, идущие к центральной щине ЭВМ.

Идея твистор памяти настолько впечатлила руководстов Bell Labs, что на 
её комерциализацию были выделены внушительные силы и средства. Но в этот 
момент появление полупроводниковой памяти, которая была ни сколь не хуже 
по потребительским качаствам, а в производстве стоила в разы дешевле, а 
также сремительное развитие интегральных микросхем значительно повлияло на 
развитие твистор памяти. Конечно она применялась в ряде проектов AT\&T 
почти до середины восьмидесятых годов, но это была агония, чем прогресс и 
она ушла в история как и память на магнитных сердечниках.

Впрочем, один положительный момент от разработки твистор памяти всё же 
имелся. Исследуя магнитострикцинный эффект в сочетаниях плёнок 
пермаллоая с ортоферритами (ферриты на основе редкоземельных элементов), 
Бобек подметил одну их особенность, связанную с намагничиванием. 
Особенность, которая привела к разработке удивительной пузырьковой 
памяти. [1]

После неудачи с твистор памятью Бобек не прекратил исследовать 
магнитную природу вещества. Дальнейшее развитие технологии началось с 
череды опытов, которые он проводил с пермаллоем в сочетании с 
ферромагнитными материалами на основе редкоземельных элементов. Бобек, в 
частности, экспериментировал с гадолиний галлиевым гранатом, используя его 
в качестве подложки для тонкого листа пермаллоя. Он выяснил, что в 
полученном сэндвиче при отсутствии магнитного поля области намагничивания 
располагаются в виде доменов разнообразной формы. Ничего нового в этом не 
было. Разбиение магнитного поля ферромагнетиков на макроскопические области 
(домены), обладающие спонтаннй намагниченностью, была предсказана ещё в 
1907 году французским физиком Пьером Вейссом. Бобек пошёл в своих 
исследованиях дальше и посмотрел, как будут вести себя такие домены в 
магнитном поле, имеющем перпендикулярное направление областям 
намагниченнности пермаллоя. К его удивлению с увеличением силы магнитного 
поля домены собирались в компактные области. Он назвал их <<пузырьками>>.

Индукционно воздействуя на пузырьки электрическим током, инженер заставил 
их двигаться по поверхности листа пермаллоя. Так же была замечена и 
другая особенность. Участки пермаллоя особой формы способны были 
отклонить движение пузырьков в предсказуемоом направлении. Экспериментируя 
с формой таких участков, Бобек нашёл оптимальную для управления 
пузырьками-доменами форму, похожую на шеврон.

Именно тогда и сформировалась идея пузырьковой памяти, в которой 
носителями логической единицы были домены спонтанной намагниченности в 
листе пермаллоя -- пузырьки. Поскольку Бобек научился двигать пузырьки по 
поверхности пермаллоя, он придумал остроумное решение по чтению 
информации в своём новом образце памяти. Если в тардиционных магнитных 
накопителях головки чтения/записи двигались над поверхностью магнитного 
слоя, отыскивая нужный участок, или, в случае магнитной ленты, последняя 
межанически протягивалась вдоль неподвижный головок, то в новой памяти 
Бобека вообще не было движущихся компонентов. Неподвижные головки 
чтения ожидали, пока магнитный пузырек к ним <<приедет>> самостоятельно, 
управляемый электрическим полем. Отклонить его в нужном направлении 
помогала система пермаллоевых <<шевронов>>.

Электрический заряд над особым участком листа пермаллоя, называемым 
генератором, непрерывно создавал магнитные пузырьки -- логические 
единицы, которые начинали двигаться по основному кольцу. Таким образом 
формировался непрерывный поток логических единиц. Кодирование информации 
происходило с помощью аннигилятора пузырьков, который <<выбивал>> в 
потоке логических единиц дыры -- логические нули. Двигаясь по основному 
кольцу поток пузырьков достигал нескольких вторичных колец-хранилищ, в 
которых часть пузырьков, перемежающихся нулями оставалась на хранение, 
постоянно циркулируя.

Было предложено и остроумное решение по считыванию информации из уже 
заполненных колец-хранилищ. <<Выгнав>> пузырьки из нужного вторичного 
кольца, контроллер электрической обмотки двигал их по главному кольцу до 
так называемого дупликатора -- системы <<шевронов>>, разделяющих 
пузырёк на два клона. Один из этих клонов по главному кольцу снова 
возращался в своё вторичное кольцо-хранилище, а второй двигался к 
детектору, содержащему обмотки, в которых наводился индукционный ток, 
передаваемый по адресной шине ЭВМ как логическая единица.

Идея была настолько простой и изящной, что после того как Бобек получил 
на неё патент, право на использование эффекта пузырьковой памятм 
приобрели почти все ключевые игроки компьютерных комплектующих того 
времени и даже исследовательские лаборатории таких солидных контор, как 
NASA.

К уникальной особенности пузырьковой памяти -- полнейшему отсутствию 
движущихся частей, добавилось ещё одно немаловажное свойство -- 
противостояние электромагнитному импульсу или жёсткому космическому 
излучению, которые фатально воздействует на память полупроводниковую. 
Именно поэтому пузырьковой памятью, в первую очередь заинтересовались 
военные и разработчики космических аппаратов. 

Основным недостатком пузырькового памяти была низкая скорость 
чтения/записи, составлявшая от дести до пятидесяти миллисекунд. Составить 
конкуренцию оперативной памяти пузырьки не могли, зато с тогдашними 
жёсткими дисками они серьёзно конкурировали. И проиграли только тогда, 
когда технология производства последних в сочетании с повышением 
скорости чтения/записи в них стали оптимальны для массового рынка.

Но пузырьковая память Бобека не стала сразу историей. Предложенный им 
способ направленного перемещение магнитных доменов в слое пермаллоая не 
давал покоя исследователям, старавшимся улучшить потребительские 
характеристики такого перспективного вида памяьти. И инженерам из 
лаборатории IBM Research, возглавляемым Стюартом Перкиным это удалось.

Их перспективный вид памятим, которую они красноречиво именуют 
Racetrack Memory является удивительной комбинацие идей Бобека и 
современных нанотехнологий.

Как и в случае пузырьковой памяти Бобека, в Racetrack Memory магнитные 
домены-единицы движутся внутри пермаллоя, но изготовлен он в виде 
тончайшего нанопроводника. На этот изогнутый подковой провод подаётся ток, 
заставляющий домены мчаться мимо головок чтения/записи, расположенный в 
основании подковы. Меняя магнитную полярность, исследователи заставили 
двигаться записанную информацию вдоль проводника, обеспечивая 
невероятную скорость чтения и записи -- единицы наносекунд.

Модуль Racetrack Memory будет предсавлять собой массив таких 
нанопроводников, каждый из которых сможет хранить определенное количество 
бит информации в виде магнитных пузырьков-доменом. 

В настоящее время Racetrack Memory всё ещё исследовательская разработка, 
о которой, однако, говорят как о вполне коммерческой перспективе в области 
систем хранения данных с произвольным доступом. [2]

На текущий момент разрабатываются новые типы памяти в которых не будет 
многих недостатков их предшественников. Например одно из <<горячих>> 
направлений магнитнорезистивная память, которая на текущий момент 
является одним из перспективных направлений в исследовании и развитии.

Ячейки MRAM сравнимы по быстродействию с SRAM -- памятью, которая 
используется в кэше процессора, по плотности ячеек -- с DRAM. Является 
энергонезависимой и гораздо экономичнее и долговечнее флэш-памяти.

Информация в ячейках MRAM хранится в двух ферромагнитных слоях, 
разделённых тонким слоем диэлектрика. Один из слоёв -- постоянных магнит, 
направление магнитного поля второго слоя может меняться. Для изменение 
магнитного поля через ячейку пропускают ток. От взаимной ориентации полей 
в этих слоях зависит электрическое сопротивление ячейки. Измеряя это 
сопротивление, можно считать хранящийся в ячейке бит. [3]

----

Магниторезистивная память (MRAM) -- одно из перспективных направлений 
новых типов памяти. В ближайшем будущем она может превзойти все 
существующие виды памяти по всем характеристикам. Ячейки MRAM сравнимы по 
быстродействию с SRAM -- памятью, которая используется в кэше процессора, 
по плотности ячеек -- c DRAM. Память MRAM энергонезависима и она гораздо 
экономичнее и долговечнее флэш-памяти.

Информация в ячейках MRAM хранится в двух ферромагнитных слоях, 
разделённых тонким слоем диэлектрика. Один из слоёв -- постоянный магнит, 
направление магнитного поля второго слоя может меняться. Для изменения 
магнитного поля через ячейку пропускают ток. От взаимной ориентации полей в 
этих слоях зависит электрическое сопротивление ячейки. Измеряя это 
сопротивление, можно считать хранящийся в ячейке бит. На сегодня самый 
совершенный способ записи используют эффект переноса спинового момента -- 
<<поляризованные>> электроны, проходя через слой ферромагнетика, помогают 
вращать магнитное поле в определённом направлении, в результате чего 
требуется гораздо меньший ток, и ячейки могут быть меньшего размера.

Исследователи Калифорнийского университате в Лос-Анджелесе (UCLA) нашли 
способ переключать магнитное поле с помощью электрического напряжения, а 
не тока, благодаря чему можно получить выигрыш в энергопотреблении от 
10 до 1000 раз, соответственно уменьшить нагрев ячеек в процессе записи 
и увеличить плотность их размещения на кристалле в 5 раз. Они называют 
это вид памяти MeRAM (magnetoelectrix random access memory). [3]


Также существует одно из перспективных типов памяти PRAM (Phase-change 
memory). Она основывается на уникальном поведении халькогенида, который 
при нагреве может <<переключаться>> между двумя состояниями: 
кристаллическим и аморфным. PRAM -- одна из новых технологий памяти, 
созданная в попытке превзойти в области энергонезависимой памяти почти 
универсальную флеш-память. [4]

\pagebreak

Список литературы:
\begin{enumerate}
	\item http://old.computerra.ru/vision/621983/
	\item http://old.computerra.ru/vision/622225/
	\item http://www.habrahabr.ru/post/163749
    \item http://ru.wikipedia.org/wiki/Память\_с\_изменением\_фазового\_состояния
\end{enumerate} 

\end{document}
