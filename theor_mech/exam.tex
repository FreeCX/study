\documentclass[a4paper, 12pt]{report}
\usepackage[utf8]{inputenc}
\usepackage[russian]{babel}
\usepackage{indentfirst}
\usepackage{pscyr}
\usepackage[margin = 2cm]{geometry}
\pagestyle{plain}

\begin{document}

  \emph{Вопросы по теоретической механике (физики).}\\

	1. Способы задания движения материальной точки. Векторный и
	координатный способы задания движения. Определение траектории
	движения, скорости, ускорения.\\

	2. Общее уравнение динамики (принцип Д'Аламбера-Лагранжа).\\

	3. Общее уравнение динамики в обобщенных координатах. Уравнения
	Лагранжа 2-го рода. Примеры применения уравнений Лагранжа в
	потенциальном поле сил.\\

	4. Параллельное векторное поле. Символы Кристофеля первого и второго
	рода. Определение символов Кристофеля через фундаментальный
	метрический тензор.\\

	5. Понятие об абсолютной и ковариантной производных.\\

	6. Естественный способ задания движения. Определение скорости и
	ускорения точки. Нормальное касательное ускорения. Их физический
	смысл. Определение касательного и нормального ускорений при
	координатном способе задания движения.\\

	7. Уравнения Лагранжа 1-го рода. Понятие об избыточных координатах.
	Неопределенные множители Лагранжа. Полная система уравнений,
	описывающая движение механической системы, с избыточными
	координатами. Примеры применения.\\

	8. Переменные Лагранжа и Эйлера для описания движения частицы
	сплошной среды. Вектор смещения. Тензор смещения как сумма
	тензора деформации и тензора поворота.\\

	9. Основные законы динамики материальной точки. Законы Ньютона.
	Дифференциальные уравнения движения материальной точки. Две
	основные задачи динамики точки.\\

	10. Понятие об изохронной и полной вариациях функции. Действие по 
	Гамильтону. Принцип наименьшего действия Гамильтона-Остроградского.\\

	11. Физический смысл дивергенции вектора смещения. Понятие о тензоре
	скоростей деформации.\\

	12. Основные законы (теоремы) механики для материальной точки. Закон
	сохранения импульса. Теорема об изменении количества движения.\\

	13. Принцип наименьшего действия Монертюн-Лагранжа. Действие по
	Лагранжу.\\

	14. Принцип наименьшего действия Монертюн-Лагранжа в форме Якоби.\\

	15. Кинетический момент твердого тела относительно неподвижной точки.
	Кинетическая энергия твердого тела, совершающего сферическое
	движение. Динамические уравнения Эйлера.\\

	16. Основные законы (теоремы) механики для материальной точки.
	Теорема об изменении момента импульса, (момента количества
	движения, кинетического момента). Закон сохранения момента
	импульса.\\

	17. Функция Гамильтона. Канонические уравнения Гамильтона. Примеры
	применения уравнений Гамильтона.\\

	18. Основные законы (теоремы) механики для материальной точки. Терема
	об изменении кинетической энергии. Закон сохранения механической
	энергии.\\

	19. Циклические координаты и циклические интегралы. Связь циклических интегралов 
	с законами сохранения.\\

	20. Структура кинетической энергии в обобщенных координатах. Матрица
	инерционных коэффициентов уравнений Лагранжа в случае
	стационарных связей.\\

	21. Полная система уравнений, описывающая движение частицы
	сплошной среды. Уравнение состояния. Простейшие модели сплошных
	сред: линейное упругое тело, идеальная жидкость.\\

	22. Определение скорости и ускорения точки в полярной системе
	координат. Дифференциальные уравнения движения материальной
	точки в цилиндрической системе координат. Количество движения,
	кинетический момент и кинетическая энергия материальной точки в
	цилиндрической системе координат.\\

	23. Кинематика твердого тела. Равновесие твердого тела.\\

	24. Кинематика твердого тела. Поступательное движение.\\

	25. Кинематика твердого тела. Вращательное движение
	вокруг неподвижной оси.\\

	26. Кинематика твердого тела. Плоское движение. Определение
	кинематических характеристик движения: траектории движения
	скорости и ускорения произвольной точки тела. Понятие о мгновенном
	центре скоростей.\\

	27. Исследование движения материальной точки в центрально-
	симметричном поле сил. Первые интегралы уравнений движения.
	Понятие об эффективной потенциальной энергии. Интегрирование
	уравнений движения.\\

	28. Принцип возможных перемещений (принцип Лагранжа). 
	Доказательство необходимости и достаточности. Примеры применения
	принципа возможных перемещений.\\

	29. Массово-геометрические характеристики твердого тела. Моменты
	инерции относительно точки, оси. Центробежные моменты инерции.
	Теорема Гюйгенса-Штейнера. Вычисление моментов инерции тела
	относительно оси заданного направления.\\

	30. Дифференциальное уравнение поступательного движения частицы
	сплошной среды. Ковариантная форма уравнения движения. Движение
	частицы сплошной среды относительно ее центра масс. Закон парности
	касательных напряжении.\\

	31. Движение материальной точки в неинерциальной системе координат.
	Абсолютное, переносное и относительное движение. Понятие об
	абсолютной (полной) и местной (локальной) производных. Формула
	Бура. Определение абсолютной скорости и ускорения материальной
	точки. Кориолисово ускорение.\\

	32. Сферическое движение твердого тела. Уравнения движения. Углы
	Эйлера. Определение скорости произвольной точки твердого тела.
	Кинематические уравнения Эйлера. Уравнение мгновенной оси 
	вращения.\\

	33. Преобразование координат. Понятие тензора. Ковариантные и
	контрвариантные компоненты вектора, тензора. Криволинейные
	координаты. Фундаментальный метрический тензор.\\

	34. Динамика относительного движения материальной точки. Переносная
	и кориолисова силы инерции. Принцип относительности Галилея.
	Частные случаи движения.\\

	35. Обобщенные силы. Способы вычисления обобщенных сил. Понятие об
	идеальных связях. Вычисление обобщенных сил в случае движения
	механической системы в поле потенциальных сил.\\

	36. Параллельное векторное поле. Символы Кристофеля первого и второго
	рода. Определение символов Кристофеля через фундаментальный
	метрический тензор. Понятие об абсолютной и ковариантной
	производных.\\

	37. Понятие об обобщенных координатах. Понятие о связях. Классификация связей. 
	Возможные (виртуальные) перемещения. Число степеней свободы механической системы.\\

	38. Кинетическая энергия механической системы. Теорема Кенига.
	Теорема об изменении кинетической энергии механической системы.
	Закон сохранения полной механической энергии.\\

	39. Уравнение неразрывности. Вид уравнения неразрывности в декартовой
	и криволинейных координатах.\\

	40. Понятие механической системы. Основные характеристики
	механической системы: масса, центр масс. Внутренние силы и их
	свойства. Дифференциальные уравнения движения механической
	системы. Теорема о движении центра масс механической системы.
	Закон сохранения движения центра масс.\\

	41. Определение траектории движения материальной точки в движущемся
	центрально-симметричном поле сил. Дифференциальное уравнение
	траектории движения. Уравнение Бине. Решение уравнения Бине. Виды
	траекторий. Влияние начальных условий на вид траектории.\\

	42. Сферическое движение твердого тела. Уравнения движения. Углы
	Эйлера. Определение скорости произвольной точки твердого тела.
	Кинематические уравнения Эйлера. Уравнение мгновенной оси
	вращения.\\

	43. Механическая система (изменяемая и неизменяемая). Масса системы.
	Центр масс и его координаты. Статические моменты массы системы
	относительно полюса и плоскости. Статические моменты массы
	относительно центра масс и плоскостей, проходящих через центр масс.\\

	44. Классификация сил, действующих на механическую систему: силы
	внутренние и внешние, задаваемые силы. Главный вектор и главный
	момент внутренних сил. Дифференциальные уравнения движения
	механической системы.\\

	45. Количество движения точки и системы. Элементарный и полный
	импульс силы. Теорема об изменении количества движения точки в
	дифференциальной и конечной формах. Теорема импульсов.\\

	46. Количество движения системы и способы его вычисления. Теорема об
	изменении количества движения системы в дифференциальной и
	конечной формах. Законы сохранения количества движения системы.\\

	47. Теорема о движении центра масс механической системы. Следствия из
	теоремы. Дифференциальные уравнения поступательного движения
	твердого тела.\\

	48. Кинетический момент точки и системы относительно центра и оси.
	Кинетический момент вращающегося твердого тела относительно оси
	вращения. Теорема об изменении кинетического момента для точки.\\

	49. Теорема об изменении кинетического момента для механической
	системы. Законы сохранения кинетических моментов.
	Дифференциальное уравнение вращательного движения твердого тела
	вокруг неподвижной оси. Физический маятник.\\

	50. Кинетический момент системы в абсолютном движении. Теорема об
	изменении кинетического момента системы в относительном
	движении по отношению к центру масс. Дифференциальные уравнения
	плоского движения твердого тела.\\

	51. Элементарная работа силы и ее аналитическое выражение . Работа
	силы на конечном пути. Работа равнодействующей. Мощность.\\

	52. Примеры вычисления работы силы. Работа силы тяжести, линейной
	силы упругости и силы тяготения.\\

	53. Работа и мощность сил. приложенных к твердому телу при
	поступательном, вращательном, сферическом и свободном движении
	твердого тела. Работа внутренних сил твердого тела.\\

	54. Кинетическая энергия материальной точки и системы. Теорема Кенига.
	Кинетическая энергия твердого тела в поступательном, вращательном
	и плоском движениях.\\

	55. Теорема об изменении кинетической энергии материальной точки и
	системы в дифференциальной и конечной формах. Случай абсолютно
	твердого тела.\\

\end{document}
