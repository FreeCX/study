\chapter{Изучение нормативных документов по РБ} \label{chap1}
\section{ОСПОРБ-99/2010} \label{sect1_1}
\subsection{Область применения} \label{subsect1_1_1}
    Основные санитарные правила и нормативы обеспечения радиационной 
    безопасности (далее -- Правила) устанавливают требования по защите 
    людей от вредного радиационного воздействия при всех условиях 
    облучения от ИИИ, на которые распространяется действие СанПиН 
    2.6.1.2523-09 НРБ-99/2009. [2]

    Основные санитарные правила, которые в целях обеспечения радиационной 
    безопасности при работе с радионуклидными источниками регламентируют:
    \begin{itemize}
    	\item[-] требования к проектированию и размещению радиационных 
    		объектов;
		\item[-] организацию работ с источниками излучения;
		\item[-] требования к администрации, персоналу и гражданам по 
			обеспечению радиационной безопасности;
		\item[-] поставку, учет, хранение и перевозки источников 
			ионизирующего излучения;
		\item[-] требования к контролю за радиационной безопасностью;
		\item[-] работу с закрытыми источниками и устройствами, генерирующими 
			ионизирующее излучение;
		\item[-] работу с открытыми источниками излучения (радиоактивными 
			веществами);
		\item[-] обращение с радиоактивными отходами;
		\item[-] радиационный контроль при работе с техногенными источниками 
			ионизирующего излучения;
		\item[-] методы и средства индивидуальной защиты и личной гигиены;
		\item[-] радиационную безопасность пациентов и населения при 
			медицинском облучении, при воздействии природных источников 
			излучения, при радиационных авариях;
		\item[-] медицинское обеспечение радиационной безопасности.
    \end{itemize}

\subsection{Общие требования к радиационному контролю} \label{subsect1_1_2}
	Радиационный контроль является частью производственного контроля и должен 
	охватывать все основные виды воздействия ионизирующего излучения на человека.

	Целью радиационного контроля является получение информации об индивидуальных 
	и коллективных дозах облучения персонала, пациентов и населения, а также 
	показателях, характеризующих радиационную обстановку.

	Объектами радиационного контроля являются:
	\begin{itemize}
		\item[-] персонал групп А и Б при воздействии на них ионизирующего 
			излучения в производственных условиях;
		\item[-] пациенты при выполнении медицинских рентгенорадиологических 
			процедур;
		\item[-] население при воздействии на него природных и техногенных 
			источников излучения;
		\item[-] среда обитания человека.
	\end{itemize}

	Радиационная безопасность на объекте и вокруг него обеспечивается за счет:
	\begin{itemize}
		\item[-] качества проекта радиационного объекта;
		\item[-] обоснованного выбора района и площадки для размещения 
			радиационного объекта;
		\item[-] физической защиты источников излучения;
		\item[-] зонирования территории вокруг наиболее опасных объектов и 
			внутри них; 
		\item[-] условий эксплуатации технологических систем;
		\item[-] санитарно-эпидемиологической оценки и лицензирования 
			деятельности с источниками излучения;
		\item[-] санитарно-эпидемиологической оценки изделий и технологий;
		\item[-] наличия системы радиационного контроля;
		\item[-] планирования и проведения мероприятий по обеспечению 
			радиационной безопасности персонала и населения при нормальной 
			работе объекта, его реконструкции и выводе из эксплуатации;
		\item[-] повышения радиационно-гигиенической грамотности персонала и 
			населения.
	\end{itemize}

\subsection{Требования к персоналу радиационного объекта} 
\label{subsect1_1_3}
	Персоналу группы А следует:
	\begin{itemize}
		\item[-] знать и строго выполнять требования по обеспечению радиационной 
			безопасности, установленные санитарными нормами и правилами;
		\item[-] использовать в предусмотренных случаях средства индивидуальной 
			защиты;
		\item[-] выполнять установленные требования по предупреждению радиационной 
			аварии и правила поведения в случае ее возникновения;
		\item[-] своевременно проходить периодические медицинские осмотры и 
			выполнять рекомендации медицинской комиссии;
		\item[-] обо всех обнаруженных неисправностях в работе установок, приборов 
			и аппаратов, являющихся источниками излучения, немедленно ставить 
			в известность руководителя (цеха, участка, лаборатории) и службу 
			радиационной безопасности (лицо, ответственное за радиационную 
			безопасность);
		\item[-] выполнять указания работников службы радиационной безопасности, 
			касающиеся обеспечения радиационной безопасности при выполнении работ.
	\end{itemize}

	Персонал группы Б должен знать свои действия в случае радиационной аварии.

\subsection{Организация работ с источниками излучения} \label{subsect1_1_4}
	Обращение с источниками излучения, предусмотренное статьей 27 
	Федерального закона от 30.03.1999 N 52-ФЗ <<О санитарно-эпидемиологическом 
	благополучии населения>> в различных областях промышленности, науки, 
	медицины, образования, сельского хозяйства, торговли, разрешается только 
	при наличии санитарно-эпидемиологического заключения на эти источники. [2]

	К работе с источниками излучения допускаются лица не моложе 18 лет, не 
	имеющие медицинских противопоказаний, отнесенные приказом руководителя к 
	категории персонала группы А, прошедшие обучение по правилам работы с 
	источником излучения и по радиационной безопасности, прошедшие инструктаж 
	по радиационной безопасности. [2]

	На определенные виды деятельности допускается персонал группы А при наличии 
	у них разрешений, выдаваемых органами государственного регулирования 
	безопасности. Перечень специалистов указанного персонала, а также 
	предъявляемые к ним квалификационные требования определяются Правительством 
	Российской Федерации. [2]

\subsection{Хранение и транспортирование источников излучения} 
\label{subsect1_1_5}
	Источники излучения, не находящиеся в работе, должны храниться в 
	специально отведенных местах или в оборудованных хранилищах, 
	обеспечивающих их сохранность и исключающих доступ к ним посторонних лиц. 
	Активность радионуклидов, находящихся в хранилище, не должна превышать 
	установленных в технической документации допустимых. Мощность эквивалентной 
	дозы на наружной поверхности такого хранилища или его ограждения, 
	исключающего доступ посторонних лиц, не должна превышать 1,0 мкЗв/ч.

	Транспортные средства, специально предназначенные для перевозки 
	радионуклидных источников за пределами радиационного объекта, должны 
	соответствовать требованиям СанПиН 2.6.1.1281-03 <<Санитарные правила по 
	радиационной безопасности персонала и населения при транспортировании 
	радиоактивных материалов (веществ)>>. [2]

	Уровни радиоактивного загрязнения поверхности транспортных средств не 
	должны превышать значений, приведенных в таблице 8.10 НРБ-99/2009.

\subsection{Нормы радиационного фона} \label{subsect1_1_6}
	Степень радиационной безопасности населения характеризуют следующие 
	значения эффективных доз облучения от всех основных природных источников 
	излучения:
	\begin{itemize}
		\item[-] менее 5 мЗв/год - приемлемый уровень облучения населения 
			от природных источников излучения;
		\item[-] свыше 5 до 10 мЗв/год - облучение населения является 
			повышенным;
		\item[-] более 10 мЗв/год - облучение населения является высоким.
	\end{itemize}

	Мероприятия по снижению уровней облучения природными источниками излучения 
	должны осуществляться в первоочередном порядке для групп населения, 
	подвергающихся облучению в дозах более 10 мЗв/год.

	Среднегодовые значения ЭРОА изотопов радона в помещениях эксплуатируемых 
	производственных зданий и сооружений не должны превышать 300 Бк/м\(^3\), а 
	мощность эквивалентной дозы гамма-излучения -- 0,6 мкЗв/ч. При 
	невозможности снизить ЭРОА изотопов радона ниже 300 Бк/м\(^3\) и/или мощности 
	эквивалентной дозы гамма-излучения ниже 0,6 мкЗв/ч, то решается вопрос о 
	перепрофилировании здания или части его помещений.
	
	Для возведения зданий и сооружений производственного назначения должны 
	применяться строительные материалы и изделия с эффективной удельной 
	активностью природных радионуклидов не более 740 Бк/кг.
	
	Обращение в производственных условиях с сырьем, материалами и изделиями 
	с эффективной удельной активностью природных радионуклидов до 740 Бк/кг, 
	а также с производственными отходами с эффективной удельной активностью 
	природных радионуклидов до 1500 Бк/кг допускается без ограничений по 
	радиационному фактору.
	
	Производственные отходы с эффективной удельной активностью природных 
	радионуклидов до 1500 Бк/кг могут направляться для захоронения в места 
	захоронения промышленных отходов без ограничений по радиационному фактору.
	
	Производственные отходы с эффективной удельной активностью природных 
	радионуклидов свыше 1,5 до 10 кБк/кг направляются для захоронения на 
	специально выделенные участки в места захоронения промышленных отходов. 
	При этом доза облучения критической группы населения за счет захоронения 
	таких отходов не должна превышать 0,1 мЗв/год. Порядок, условия и способы 
	захоронения таких производственных отходов устанавливаются органами 
	местного самоуправления. [2]

\clearpage

\section{НРБ-99/2009} \label{sect2_1}
\subsection{Область применения} \label{subsect1_2_1}
	Нормы радиационной безопасности (НРБ) -- применяются для обеспечения 
	безопасности человека во всех условиях воздействия на него 
	ионизирующего излучения искусственного или природного происхождения.

	Суммарная доза от всех видов облучения используется для оценки 
	радиационной обстановки и ожидаемых медицинских последствий, а также 
	для обоснования защитных мероприятий и оценки их эффективности.
	
	Требования и нормативы, установленные Нормами, являются обязательными 
	для всех юридических и физических лиц, независимо от их подчиненности и 
	формы собственности, в результате деятельности которых возможно облучение 
	людей, а также для администраций субъектов Российской Федерации, местных 
	органов власти, граждан Российской Федерации, иностранных граждан и лиц без 
	гражданства, проживающих на территории Российской Федерации. [1]

	Нормы распространяются на следующие  источники ионизирующего излучения:
	\begin{itemize}
		\item[-] техногенные источники за счёт нормальной эксплуатации 
			техногенных источников излучения;
		\item[-] техногенные источники в результате радиационной аварии;
		\item[-] природные  источники;
		\item[-] медицинские источники.
	\end{itemize}

	Требования Норм не распространяются на источники излучения, создающие при 
	любых условиях обращения с ними: 
	\begin{itemize}
		\item[-] индивидуальную годовую эффективную дозу не более 10 мкЗв;
		\item[-] коллективную эффективную годовую дозу не более 1 чел.-Зв, 
			либо когда при коллективной дозе более 1 чел.-Зв оценка по принципу 
			оптимизации показывает нецелесообразность снижения коллективной дозы;
		\item[-] индивидуальную годовую эквивалентную дозу в коже не более 50 
			мЗв и в хрусталике глаза не более 15 мЗв.
	\end{itemize}
	
	Требования Норм не распространяются также на космическое излучение на 
	поверхности Земли и внутреннее облучение человека, создаваемое природным 
	калием, на которые практически невозможно влиять.

\subsection{Общие положения} \label{subsect1_2_2}
	Для  обеспечения радиационной безопасности при нормальной эксплуатации 
	источников излучения необходимо руководствоваться следующими основными 
	принципами:
	\begin{itemize}
		\item[-] не превышение допустимых пределов индивидуальных доз облучения 
			граждан от всех источников  излучения;
		\item[-] запрещение всех видов деятельности по использованию 
			источников излучения, при которых полученная для человека и 
			общества польза не превышает риск возможного вреда, 
			причиненного дополнительным облучением;
		\item[-] поддержание на возможно низком и достижимом уровне с учетом 
			экономических и социальных факторов индивидуальных доз облучения и 
			числа облучаемых лиц при использовании любого источника излучения.
	\end{itemize}

	Для обеспечения условий, при которых радиационное воздействие будет ниже 
	допустимого, с учётом достигнутого в организации уровня радиационной 
	безопасности, администрацией организации дополнительно устанавливаются 
	контрольные уровни.

	Основные пределы доз, как и все остальные допустимые уровни облучения 
	персонала группы Б равны \( 1/4 \) значений для персонала группы А.

	Эффективная доза для группы А: 20 мЗв в год в среднем за любые 
	последовательные 5 лет, но не более 50 мЗв в год, а для населения:
	1 мЗв в год в среднем за любые последовательные 5 лет, но не более 
	5 мЗв в год.

	Эквивалентная доза за год в хрусталике глаза, коже, кистях и стопах 
	для группы А составляет: 150, 500 и 500 мЗв. Для населения эквивалентная 
	доза составляет одну сотую от заявленного для персонала группы А.

	В отношении всех источников облучения населения следует принимать 
	меры как по снижению дозы облучения у отдельных лиц, так и по 
	уменьшению числа лиц, подвергающихся облучению, в соответствии с 
	принципом оптимизации. [1]

\section{Методики радиационного контроля}
	Существует много методик радиационного контроля и каждая из них отвечает 
	своим определенным требованиям к проведению мероприятий по РБ. Сделаем 
	краткий обзор используемых методик на предприятии ЗАО 
	<<Титан-Изотоп>> [7-14].

	Методика [7] устанавливает правила организации дозиметрического 
    контроля металлолома, порядок и способы выполнения измерений, а 
    также правила оценки результата контроля.

	Методика [8] представляет правила организации контроля, требование к 
    объектам содержащие естественные и природные радионуклиды, а также 
    дозиметрический контроль материалов и изделий и их классификацию.

	Методика [9] представляет правила дозиметрического контроля помещений 
	жилых и общественных зданий, а также обеспечивает оценку радиационного 
	состояния названных объектов с точки зрения радиационной безопасности 
	населения. 

	Методика [10] представляет правила дозиметрического контроля 
    производственных помещений, а также обеспечивает оценку с точки 
    зрения радиационной безопасности персонала.

	Методика [11] устанавливает нормы дозиметрического контроля для 
    территорий, а также их организацию и выполнение.

	Методика [12] представляет результаты дозиметрического контроля, 
    служащие основой для оценки пригодности участков к строительству.

	Методика [13] предназначена для обеспечения контроля удельной 
    активности грунта (почвы) на территориях различного вида, выполняемого 
    лабораторией радиационного контроля ЗАО <<Титан-Изотоп>>.
	
	Методика [14] предназначена для обеспечения радиометрического контроля 
    производственных отходов с повышенным содержанием естественных 
    р/нуклидов.

\clearpage
