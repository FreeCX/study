\section{Введение}

Проблема техники в ее отношении к культуре и истории впервые возникает в 
XIX в. XVIII столетие ставило вопрос о смысле и ценности культуры главным 
образом скептически, с равным отчаянию сомнением -- тот вопрос, который вел к 
дальнейшим, все более мелким вопросам, а тем самым создал предпосылки для 
того, чтобы сегодня, в XX в., стала заметной проблематичность всей мировой 
истории.

Тогда, в век Робинзона и Руссо, английских парков и пастушеской поэзии, в с
амом <<первобытном>> человеке видели некую овечку, мирную и добродетельную, 
лишь впоследствии испорченную культурой. Технику вообще не замечали и во 
всяком случае считали ее -- в сравнении с рассуждениями о морали -- чем-то не 
заслуживающим внимания.

Но со времен Наполеона колоссально разросшаяся машинная техника Западной 
Европы, с ее фабричными городами, железными дорогами и пароходами, заставила, 
наконец, со всей серьезностью поставить эту проблему. Что означает техника? 
Каков ее смысл в рамках истории, какова ее ценность для человеческой жизни, 
каков ее нравственный или метафизический уровень? На это были даны 
бесчисленные ответы, но по существу они сводятся к двум.

Чтобы понять сущность техники, нужно исходить не из машинной техники, по 
крайней мере не поддаваться искушению видеть цель техники в создании машин и 
инструментов.

В действительности техника принадлежит древнейшим временам. Она не является и 
какой-то исторической особенностью, будучи чем-то чудовищно всеобщим. Она 
простирается за пределы человека, назад, к жизни животных, а именно, всех 
животных. В отличие от растений, к жизненному типу животных принадлежит 
свободное передвижение в пространстве, относительная самопроизвольность и 
независимость от всей остальной природы, а тем самым и необходимость себя ей 
противопоставлять, чтобы наделять свой вид смыслом, содержанием, и 
превосходством. \cite{bib:06}

Чтобы лучше понять явление техники как цивилизационного процесса начнём с 
понятия цивилизации, а далее более подробно рассмотрим технику и её влияние 
на формирование человеческих благ.

\newpage

\section{Понятие цивилизация}

Понятие <<цивилизация>> первоначально появилось во французском языке в 
середине XVIII в. в русле теории прогресса и употреблялось только в 
единственном числе. В частности, просветители называли цивилизацией идеальное 
общество, основанное на разуме и справедливости.

Основные подходы к интерпретации понятия цивилизации. В содержательно-
методологическом плане можно выделить несколько подходов в интерпретации 
понятия <<цивилизация>>: культурологический, социологический, 
этнопсихологический, географический.

В рамках культурологического подхода М. Вебер, а за ним А. Тойнби 
рассматривали цивилизацию как особый социокультурный феномен, ограниченный 
определенными пространственно-временными рамками, основу которого составляет 
религия. А. Тойнби указывал также и на четко выраженные параметры 
технологического развития. 

Ряд ученых, оставаясь в пределах культурологического подхода, выносят 
цивилизацию за пределы ее социокультурного основания. Они адресуют понятие 
<<цивилизация>> только тем социокультурным образованиям, которые обладают 
творческой способностью вырабатывать универсальные символы. В частности, 
В. Каволис связывает понятие <<цивилизация>> со способностью данной культурной 
системы к коммуникации, усвоению и толкованию всеобщих идиом и значений. 
Поэтому он предлагает положить в основу определения конкретной цивилизации 
не ее социокультурный код, как это обычно делается, а принцип 
<<соотнесенности>> присущих ей всеобщих представлений и значений. При этом, с 
одной стороны, В. Каволис подчеркивает, что отдельные цивилизации вырабатывают 
собственные оценки этих универсалий (например, свободы, прав человека, власти 
и т.д.) и выражают таковые через призмы своих ценностей и исторического опыта. 
С другой стороны, он признает наличие глобальной конфигурации этих 
универсально-символических форм и даже глобального сознания.

Многие исследователи сходятся в том, что цивилизация представляет собой 
внешний по отношению к человеку мир, воздействующий на него и противостоящий 
ему, в то время как культура является внутренним достоянием человека, 
раскрывая меру его развития и являясь символом его духовного богатства. Такое 
утверждение вполне уместно только в рамках деятельностного подхода к пониманию 
культуры как внутренней интенции жизнедеятельности человека, ее духовного 
кода. Однако при ценностном подходе к культуре, при котором она 
интерпретируется как нормативно-ценностное пространство бытия той или иной 
социальной общности, такое противопоставление культуры и цивилизации выглядит 
необоснованным. Вместе с тем надо согласиться, что цивилизация по отношению к 
человеку является внешним миром. Однако при этом не следует противопоставлять 
понятия <<цивилизация>> и <<культура>>, ибо культура как надындивидуальная 
реальность также является внешней.

<<Эпоха техники>> приобрела всеохватывающий характер в XVIII и получила 
чрезвычайно быстрое развитие в XX столетии. Эта эпоха знаменует собой время 
наступления духовного единства человечества, мировой история, не как идеи, а 
как реальности. Ситуация единства мировой истории была создана Европой, 
которая благодаря географическим открытиям, достижениям науки и техники, к 
концу XX в. обрела власть над миром, усвоившим европейскую технику, но 
сохранившим в своих устремлениях уникальные культурные различия.

Другой разновидностью всемирно-исторической интерпретации понятия 
<<цивилизация>> является своеобразная историческая концепция Д. Уилкинса. Он 
считает, что существует единая <<Центральная цивилизация>>, зародившаяся при 
слиянии египетской и месопотамской цивилизаций и пережившая все другие 14 
цивилизаций. Современный мир -- это, следовательно, лишь стадия исторически 
непрерывной <<Центральной цивилизации>>.\cite{net:01}

\section{Техника и её влияние}

\subsection{К первоосновам техники}

Как и почему существует в мире столь странное явление, столь абсолютный акт, 
который представляет техника, техническая деятельность? Для получения 
серьёзного ответа на этот вопрос нужно решительно и немедля заглянуть в 
подстерегающие нас на этом пути глубины.

Лишь тогда мы увидим, как одно сущее (то есть человек, если он желает 
существовать) вынуждено пребывать в другом -- в мире или природе. И это 
пребывание одного в другом -- человека в мире -должно отвечать одному из трех 
требований.

\begin{enumerate}
    \item Природа представляет пребывающему в ней человеку только сплошные 
        удобства в чистом виде. Это значит, что человеческое бытие целиком и 
        полностью совпадает с бытием мира, и человек оказывается целиком 
        творением природы. Таковы камни, растения и, по-видимому, животные. Но 
        в таком случае человек не имел бы потребностей (ни о чем бы не 
        заботился и ни в чем бы не нуждался). Его желания никак не отличались 
        бы от их исполнения. Ибо человеку всегда хотелось бы одного: чтобы все 
        в мире существовало так, как оно существует. А стоило бы ему что-нибудь 
        себе еще пожелать, и оно ipso facto тут же являлось бы, как в сказках 
        про волшебную палочку. Подобное существо не могло бы воспринимать мир 
        как отличное от себя, поскольку данный мир не оказывал бы ему никакого 
        сопротивления. Путешествовать, странствовать по белу свету означало бы 
        путешествовать, странствовать внутри самого себя.

    \item Однако могло бы произойти и обратное: мир создавал бы человеку одни 
        трудности, иначе говоря, бытие человека и бытие мира находились бы в 
        абсолютном противоборстве. В таком случае у человека не было бы 
        никакой возможности укрыться в мире, он не мог бы находиться, 
        пребывать в нем ни секунды. Но тогда так называемой <<человеческой 
        жизни>> не было бы вовсе, как не было бы и техники.

    \item Третья возможность -- это, собственно, сама реальность. Человек, 
        поскольку ему необходимо быть в мире, сталкивается с тем, что мир 
        сплошь и рядом опутывает его   плотной замысловатой сетью, 
        предоставляя удобства и в то же время чиня препятствия. По сути, мир 
        из них и состоит. Скажем, земля, почва поддерживает человека своей 
        упругостью, твердостью, позволяя ему прилечь для отдыха или и убежать 
        от опасности. Тот, кто тонул в морской пучине или срывался с крыши, 
        вполне оценил надежную твердость земли. Но земля -- это еще и 
        расстояние, и как часто она разделяет жаждущего и родник! Иной раз 
        земля вдруг круто поднимается вверх откосом, который предстоит 
        одолеть. Вот, пожалуй, радикальнейший из, феноменов: наше 
        существование в мире окружено удобствами и трудностями. И именно это 
        придает особый онтологический характер реальности, называемой 
        человеческой жизнью, бытием человека.
\end{enumerate}

Если бы на жизненном пути вообще не встречалось удобств, то пребывание 
человека в мире было бы невозможным; иначе говоря, он вообще не существовал 
бы, а следовательно, не было бы и проблемы. Но поскольку удобства, которыми 
удается воспользоваться, все же встречаются в жизни, то и возможность жизни 
реализуется. Однако эта возможность всегда под угрозой -- ибо человек 
встречает и трудности, и помехи. Отсюда вывод: человеческое существование, 
пребывание в мире вовсе не означает пассивного присутствия; наоборот, оно 
неизбежно предполагает борьбу с трудностями и неудобствами, препятствующими 
нам надежно укрыться в мире. Любому камню существование всегда предсказано в 
изначальном, готовом виде, ему не нужно бороться, чтобы быть тем, что он есть: 
камнем среди природы. Для человека существование всегда подразумевает борьбу с 
окружающими трудностями; иными словами, в каждый миг человек вынужден 
создавать себя самого. Можно сказать и по-другому: существование дано человеку 
как абстрактная возможность. Но реальность ее человеку приходится завоевывать 
самому; в каждый жизненный миг не только экономически, но и метафизически 
человек обречен зарабатывать себе на жизнь.

Человек одновременно и естествен, и сверхъестествен. Это своего рода 
онтологический кентавр, одна половина которого вросла в природу, а другая -- 
выходит за пределы, то есть ей трансцендентна. Данте мог бы сказать, что 
человек находится в природе, как лодка, вытащенная на берег, когда одна 
половина лежит на песке, а другая -- в воде. Природное, или естественное, 
человеческое начало осуществляется само по себе -- здесь нет проблемы. И 
именно поэтому человек не считает природное подлинным бытием. Наоборот, 
сверхъестественное, надприродное в человеке никак не может считаться 
осуществленным, итоговым -- он всегда в стремлении к бытию, в жизненном 
проекте. Это и есть наше подлинное бытие, наша личность, наше <<Я>>. Ни в 
коем случае нельзя истолковывать эту сверхъестественную и антиестественную 
часть человека в духе былого спиритуализма. Здесь нет и речи об ангелах или 
так называемом духе -- смутной и странной идеи, преисполненной магических 
отголосков.

То, что люди называют жизнью, не что иное, как неудержимое стремление воплотить 
определенный проект или программу существования. И <<Я>>, личность каждого -- 
это не что иное, как воображаемая программа. Итак, человек -- это 
прежде всего нечто, не имеющее телесной или духовной реальности; человек -- 
это программа как таковая и, следовательно, то, чего еще нет, и то, что 
стремится быть. Но здесь возникает одно возражение: а возможна ли программа 
без предваряющего размышления, без предзаданной мысли, а значит, ума, души -- 
неважно, как это называть. Сделаем одно замечение: совершенно неважно, 
возникает ли программа, проект, к примеру, стать крупным финансистом 
обязательно в форме идеи. Суть в том, что <<быть>> такой программой отнюдь не 
значит быть подобной <<идеей>>. Ведь я преспокойно могу мыслить такую идею, 
будучи тем не менее весьма далек от того, чтобы быть подобным проектом.

Все, называемое природой, обстоятельствами, миром, изначально представляет 
собой в чистом виде систему удобств и трудностей, которые находит упомянутый 
человек-программа. И если подумать, то все три слова -- только 
истолкования, данные человеком тому, с чем он изначально встречается и что 
являет лишь сложную совокупность удобств и трудностей. В первую очередь именно 
<<природа>> и <<мир>> выступают двумя понятиями, которые характеризуют нечто 
наличное или же существующее само по себе, независимо от нас. Так обстоит 
дело с понятием <<вещь>>, которое означает нечто, имеющее определенное и 
постоянное бытие, отдельное от человека, самостоятельное. Но, повторяю, все 
это лишь реакция нашей мысли на то, что мы изначально обнаруживаем вокруг 
своего <<Я>>. А то, что мы изначально обнаруживаем, на самом деле лишено 
бытия, отдельного, независимого от нас, ибо оно исчерпывает свое содержание, 
существуя как удобство или неудобство для нас. Следовательно, в той мере, в 
какой оно существует относительно нашего намерения. Ведь нечто может быть 
удобством или трудностью только в связи с подобным усилием. Только в 
зависимости от нас, от наших чаяний, сообщающих нам самим подлинность, 
существуют те или иные -- большие или меньшие -- удобства и трудности. Из них 
и состоит наше окружение в его изначальном и чистом виде. Вот мир оказывается 
разным в каждую эпоху и для каждого отдельного человека. На нашу личную 
программу, на се динамичность, подчиняющую обстоятельства, последние отвечают, 
формируя свой иной облик, предстающий как особые удобства и трудности. Вне 
сомнений, мир не одинаков для торговца и для поэта; и там, где один 
спотыкается на каждом шагу, другой чувствует себя как рыба в воде; и то, что 
одному глубоко омерзительно, другому доставляет высшую радость. Конечно, миры 
обоих имеют множество общих черт -- тех самых, которые вообще свойственны 
человеку как представителю известного рода. Но именно потому, что человеческое 
бытие -- не данность, а лишь исходная, воображаемая возможность, род людской 
отличается такой неустойчивостью и изменчивостью, которые не идут ни в какое 
сравнение с различиями, характерными для животных. Словом, вопреки тому, что 
твердили ревнители равенства на протяжении двух прошлых веков и что за ними 
повторяют теперешние архаисты, люди бесконечно разны. \cite{bib:04}

\subsection{Понятие и определение техники}

Под техникой понимается система созданных средств и орудий производства, а 
также приемы и операции, умение и искусство осуществления трудового процесса. 
В технике человечество аккумулировало свой многовековой опыт, приемы, методы 
познания и преобразования природы, воплотило все достижения человеческой 
культуры. В формах и функциях технических средств своеобразно отразились формы 
и способы воздействия человека на природу. Будучи продолжением и многократным 
усилением органов человеческого тела (рук, ног, пальцев, зубов, глаз и других 
органов чувств, а ныне и мозга, например компьютеры), определенные технические 
устройства в свою очередь диктуют человеку приемы и способы их применения: из 
лука стреляют, а с помощью комбайна осуществляют сложные сельскохозяйственные 
операции, молотком забивают гвозди, а с помощью гвоздодера их вытаскивают. 
Техника возникает, когда для достижения цели вводятся промежуточные средства. 
Таким образом, техника как <<производительные органы общественного человека>> 
есть результат человеческого труда и развития знания и одновременно их 
средство. \cite{net:02}

\textit{Техника как средство.} Техника возникает, когда для достижения цели 
вводятся промежуточные средства. Непосредственная деятельность, подобно 
дыханию, движению, принятию пищи, еще не называется техникой. Лишь в том 
случае, если эти процессы совершаются неверно, и для того, чтобы выполнять 
их правильно принимаются преднамеренные действия, говорят о технике дыхания 
и т. п.

\textit{Смысл техники.} Власть над природой обретает смысл лишь при наличии 
целей, поставленных человеком, таких, как облегчение жизни, сокращение 
каждодневных усилий, затрачиваемых на обеспечение физического существования, 
увеличение досуга и удобств. Смысл техники состоит в освобождении от власти 
природы. Ее назначение -- освободить человека как животное существо от 
подчинения природе с ее бедствиями, угрозами и оковами. Поэтому принцип 
техники заключается в целенаправленном манипулировании материалами и силами 
природы для реализации назначения человека.

\textit{Виды техники.} Мы различаем технику, производящую энергию, и технику, 
производящую продукты. Так, например, рабочую силу человек получает с помощью 
прирученных им животных, ветряных и водяных мельниц. Техника, производящая 
продукты, делает возможными такие занятия, как прядение, ткачество, гончарное, 
строительное дело, а также применение медицинских средств лечения.

\textit{Искажения.} Если смысл техники состоит в преобразовании среды для 
целей человеческого существования, то об искажении можно говорить во всех тех 
случаях, когда орудия и действия перестают быть опосредствующими звеньями и 
становятся самоцелью, где забывают о конечной цели, и целью, абсолютной по 
своему значению, становятся средства.\cite{bib:03}

\textit{Цель и функция техники} -- преобразовывать природу и мир человека в 
соответствии с целями, сформулированными людьми на основе их нужд и желаний. 
Лишь редко люди могут выжить без своей преобразующей деятельности. 
Следовательно, техника -- это необходимая часть человеческого существования на 
протяжении всей истории.

Техника не есть цель сама по себе. Она имеет ценность только как средство. 
Конечно, можно рассматривать технику как самостоятельный феномен, но эта 
самостоятельность относительна: техника органически вписана в контекст 
социального бытия и сознания, составляя основу цивилизации, она находится в 
потоке текущего исторического времени и постоянно прогрессирует.

Сама по себе техника не хороша и не дурна. Все зависит от того, что из нее 
сделает человек, чему она служит, в какие условия он ее ставит, и вопрос в 
том, что за человек подчинит ее себе, каким проявит он себя с ее помощью. 
Техника не зависит от того, что может, быть ею достигнуто.

В качестве самостоятельной сущности -- это бесплодная сила, парализующий по 
своим конечным результатам триумф средства над целью. Как бы могущественно 
сильна ни была техника в своих созидательных и разрушительных возможностях, 
она в принципе всегда -- и прежде в веках, и в сколь угодно отдаленном 
грядущем -- есть средство, орудие, подчиненное разуму и воле человека. А 
разум, по Г. Гегелю, <<столь же хитер, сколь могуществен>>.\cite{net:02}

\subsection{Современная техника}

В настоящее время мы все осознаем, что находимся на переломном рубеже истории, 
живем в период, который уже сто лет тому назад сравнивали с закатом античного 
мира, а затем все глубже стали ощущать его громадное значение не только для 
Европы и западной культуры, но и для всего мира. Это -- век техники со всеми 
ее последствиями, которые, по-видимому, не оставят ничего из всего того, что 
на протяжении тысячелетий человек обрел в области труда, жизни, мышления, в 
области символики.

Немецкие философы-идеалисты -- Фихте, Гегель и Шеллинг -- интерпретировали 
свое время как эпоху глубочайшего поворота в истории, исходя из идеи 
христианского осевого времени, которое, по их мнению, только и ведет к 
последнему рубежу, к завершению!. Это не более чем дерзостное высокомерие, 
порожденное духовным самообманом. Теперь, проводя сравнение, мы можем с 
уверенностью сказать: настоящее -- не второе осевое время. Более того, резко 
контрастируя с ним, настоящее являет собой катастрофичное обеднение в области 
духовной жизни, человечности, любви и творческой энергии; и только одно -- 
успехи науки и техники -- действительно составляет его величие в сравнении со 
всем предыдущим. 

Мы понимаем, как счастливы должны быть первооткрыватели и изобретатели, но 
вместе с тем видим, что они лишь функционеры в цепи по существу анонимного 
творческого процесса, внутри которого одно звено переходит в другое и 
участники которого действуют не как люди и не в величии единой всеохватывающей 
души. Невзирая на высокий уровень творческих находок, терпеливого, упорного 
труда, смелости теоретических поисков и планов, все это в целом подчас 
производит такое впечатление, будто самый дух втягивается в технический 
процесс, который подчиняет себе даже науку, и от поколения к поколению все 
более решительно. Отсюда и поразительная ограниченность многих естественников 
вне их специальной области, беспомощность стольких техников вне их 
непосредственных задач, которые для них, но отнюдь не сами по себе, являются 
столь важными; отсюда и скрытая неудовлетворенность, господствующая в этом 
все более теряющем всякую человечность мире.\cite{bib:03}

Наука как двигатель прогресса дало нам много выдающихся открытий. Без 
передовых научных открытий в физике у нас не было бы компьютера, телефона, 
микроволновой печи и многих других полезных в современном мире устройств и 
приборов. Постепенное развитие роботов и искусственного интеллекта должно 
облегчить жизнь многим людям. Печать протезов и биологических тканей помогает 
многим людям восстановить утраченные возможности, да и возможно расширить их. 
Я не могу сказать, что это не понесёт за собой никаких разрушительных 
последствий, всегда есть такая вероятность (любой фантаст может предоставить 
много интересных идей насчёт технической сингулярности), но это даёт большой 
спектр новых возможностей для широкого круга людей в реализации их целей и 
невероятных технических идей за которыми будующее. Конечно эволюция 
технологических средств несёт и негативный характер: становится всё меньше 
читающих людей, что влечёт за собой менее развитое поколение детей, повышается 
планка технической грамотности, что затрудняет некоторым людям вести обычную 
жизнь; но не может же быть только хорошее, это же не утопическая система.

\subsection{Человек и техника}

Человек всегда был связан с техникой; он производит и использует или 
потребляет продукты техники. Но в то же время человек -- продукт своей 
технической деятельности посредством коммуникации и обмена трудом и работой.

Исторический процесс развития техники включает три основных этапа: орудия 
ручного труда, машины, автоматы. Техника в своем развитии сейчас, пожалуй, 
начинает приближаться к человеческому уровню, двигаясь от аналогии с 
физическим трудом и его организацией к аналогиям с ментальными свойствами 
человека. Пока мы достигли зоологической стадии техники, которая 
действительно значительно разработана.

Чем менее материальной, физической или наглядной является техническая имитация 
человека, тем сложнее овладеть техникой и контролировать ее. Так как все, что 
сделано человеком, происходит от его человечности, техника всегда является 
средством для самореализации и познания самого себя. По словам А. Хунига, 
техника во все исторические моменты выражает людей и идею человечности данного 
времени. Это становится ясным в результате разработок в современной технике, 
особенно в таких отраслях, как микробиология и информатика. Новые открытия и 
изобретения в этой области могут привести к новому знанию о человеке и 
человеческом мире.

Создав такое орудие труда, как компьютер -- кибернетическую систему, 
моделирующую различные виды мыслительной деятельности, оперирующую сложными 
видами информации, человек произвел свой интеллектуально-
информационный аналог, создал псевдосубъекта. Конечно, компьютерная система -- 
прежде всего орудие труда. Человек активно воздействует на него, постигая при 
взаимодействии с ним его возможности, изменяет, совершенствует его -- это одна 
сторона взаимодействия, которая условно может быть названа объектовой. В то же 
время современный компьютер -- уже не простое орудие. Хотя и не в полной мере 
и не в совершенном виде он представляет собой функциональный аналог 
мыслительной деятельности. Человек, взаимодействуя с ним, испытывает на себе 
его влияние -- это другая, условно говоря, гуманитарная сторона взаимодействия.

По мнению Н. Винера, проблема совместного функционирования, взаимной 
коммуникации человека и машины является одной из узловых проблем кибернетики. 
Производство персональных компьютеров достигло десятков миллионов в год, и в 
сферу взаимодействия с компьютерами вовлечены значительные массы людей во всем 
мире. Поэтому проблема взаимодействия человека с компьютером из проблемы 
кибернетики, психологии и других специальных дисциплин в ближайшие годы может 
перерасти в глобальную, общечеловеческую.\cite{net:02}

\subsection{Опасность техники}

В чем главная опасность, которую несет с собою машина для человека, опасность 
уже вполне обнаружившаяся? Я не думаю, чтобы это была опасность главным 
образом для духа и духовной жизни. Машина и техника наносят страшные поражения 
душевной жизни человека, и, прежде всего жизни эмоциональной, человеческим 
чувствам. Душевно-эмоциональная стихия угасает в современной цивилизации. 
Так можно сказать, что старая культура была опасна для человеческого тела, она 
оставляла его в небрежении, часто его изнеживала и расслабляла. Машинная, 
техническая цивилизация опасна, прежде всего, для души. Сердце с трудом 
выносит прикосновение холодного металла, оно не может жить в металлической 
среде. Для нашей эпохи характерны процессы разрушения сердца как ядра души. 
У самых больших французских писателей нашей эпохи, напр., Пруста и Жида, 
нельзя уже найти сердца как целостного органа душевной жизни человека. Все 
разложилось на элемент интеллектуальный и на чувственные ощущения. Кейзерлинг 
совершенно прав, когда он говорит о разрушении эмоционального порядка в 
современной технической цивилизации и хочет восстановления этого порядка. 
Техника наносит страшные удары гуманизму, гуманистическому миросозерцанию, 
гуманистическому идеалу человека и культуры. Машина по природе своей 
антигуманистична. Техническое понимание науки совершенно противоположно 
гуманистическому пониманию науки и вступает в конфликт с гуманистическим 
пониманием полноты человечности. Это все тот же вопрос об отношении к душе. 
Техника менее опасна для духа, хотя это на первый взгляд может удивить. В 
действительности можно сказать, что мы живем в эпоху техники и духа, не в 
эпоху душевности. Религиозный смысл современной техники именно в том, что она 
все ставит под знак духовного вопроса, а потому может привести и к 
одухотворению. Она требует напряжения духовности.

Техника перестает быть нейтральной, она давно уже не нейтральна, не 
безразлична для духа и вопросов духа. Да и ничто, в конце концов, не может 
быть нейтральным, нейтральным могло что-то казаться лишь до известного времени 
и лишь на поверхностный взгляд. Техника убийственно действует на душу, но она 
вместе с тем вызывает сильную реакцию духа. Если душа, предоставленная себе, 
оказалась слабой и беззащитной перед возрастающей властью техники, то дух 
может оказаться достаточно сильным. Техника делает человека космиургом. По 
сравнению с орудиями, которые современная техника дает в руки человека, 
прежние его орудия кажутся игрушечными. Это особенно видно на технике войны. 
Разрушительная сила прежних орудий войны была очень ограничена, все было очень 
локализовано. Старыми пушками, ружьями и саблями нельзя было истребить большой 
массы человечества, уничтожить большие города, подвергнуть опасности самое 
существование культуры. Между тем как новая техника дает эту возможность. И 
во всем техника дает в руки человека страшную силу, которая может стать 
истребительной. Скоро мирные ученые смогут производить потрясения не только 
исторического, но и космического характера. Небольшая кучка людей, обладающая 
секретом технических изобретений, сможет тиранически держать в своей власти 
все человечество. Это вполне можно себе представить. Эту возможность 
предвидел Ренан. Но когда человеку дается сила, которой он может управлять 
миром и может истребить значительную часть человечества и культуры, тогда все 
делается зависящим от духовного и нравственного состояния человека, от того, 
во имя чего он будет употреблять эту силу, какого он духа. Вопрос техники 
неизбежно делается духовным вопросом, в конце концов, религиозным вопросом. 
От этого зависит судьба человечества. Чудеса техники, всегда двойственной по 
своей природе, требуют небывалого напряжения духовности, неизмеримо большего, 
чем прежние культурные эпохи. Духовность человека не может уже, быть 
органически-растительной. И мы стоим перед требованием нового героизма, и 
внутреннего, и внешнего. Героизм человека, связанный в прошлом с войной, 
кончается, его уже почти не было в последней войне. Но техника требует от 
человека нового героизма, и мы постоянно читаем и слышим о его проявлениях. 
Таков героизм ученых, которые принуждены выйти из своих кабинетов и 
лабораторий. Полет в стратосферу или опускание на дно океана требует, конечно, 
настоящего героизма. Героизма требуют все смелые полеты аэропланов, борьба с 
воздушными бурями. Проявления человеческого героизма начинают связываться со 
сферами космическими. Но силы духа требует техника, прежде всего для того, 
чтобы человек не был ею порабощен и уничтожен. В известном смысле можно 
сказать, что речь идет о жизни и смерти. Иногда представляется такая страшная 
утопия. Настанет время, когда будут совершенные машины, которыми человек мог 
бы управлять миром, но человека больше не будет. Машины сами будут 
действовать в совершенстве и достигать максимальных результатов. Последние 
люди сами превратятся в машины, но затем и они исчезнут за ненадобностью и 
невозможностью для них органического дыхания и кровообращения. Фабрики будут 
производить товары с большой быстротой и совершенством. Автомобили и аэропланы 
будут летать. По всему миру будут звучать музыка и пение, будут 
воспроизводиться речи прежних людей. Природа будет покорена технике. Новая 
действительность, созданная техникой, останется в космической жизни. Но 
человека не будет, не будет органической жизни. Этот страшный кошмар иногда 
снится. От напряжения силы духа зависит, избежит ли человек этой участи. 
Исключительная власть технизации и машинизации влечет именно к этому пределу, 
к небытию в техническом совершенстве. Невозможно допустить автономию техники, 
предоставить ей полную свободу действия, она должна быть подчинена духу и 
духовным ценностям жизни, как, впрочем, и все. Но дух человеческий справится с 
грандиозной задачей в том  лишь случае, если он не будет изолирован, и не 
будет опираться лишь на себя, если он будет соединен с Богом. Только тогда 
сохранится в человеке образ и подобие Божие, т. е. сохранится человек. В этом 
обнаруживается различие эсхатологии христианской и эсхатологии технической. 
\cite{bib:01}

\newpage

\section{Заключение}

Мы, конечно, не должны закрывать глаза на то, что с возрастанием мощности 
техники увеличивается как благополучие людей, так и опасность для их 
существования. Следует вспомнить, что Т. Адорно в статье <<О технике и 
гуманизме>> на вопрос: <<приносит современная техника в конечном счете пользу 
или вред человечеству?>>, отвечает, что <<это зависит не от техников и даже не 
от самой техники, а от того, как она используется обществом. Это использование 
не является делом доброй или злой воли, а зависит от объективных структур 
общества в целом. В обществе, устроенном соответственно человеческому 
достоинству, техника не только была бы освобождающим фактором, но и обрела бы 
сама себя. Если сегодня техники иногда испытывают страх перед тем, что может 
произойти с их изобретениями, то ведь лучшей реакцией на этот страх была бы 
попытка как-то содействовать установлению общества, отвечающего человеческому 
достоинству>>.

В заключении работы хочу сказать следующее: технологический прогресс 
невозможно остановить и не важно является он хороший или плохой (хотя лучше 
всего его нужно считать нейтральным), нужно правильно направлять его течение и 
надеяться на лучшее. Никто не может сказать как повлияет на развитие 
цивилизации то или иное открытие, но не сделав его мы также не можем сказать к 
чему это приведёт. В любом случае найдётся другой человек, который всё равно 
его сделает и использует, но в этом случае нельзя быть точно уверенным, что он 
использует своё творение правильно. Мы же всего лишь люди.

\newpage