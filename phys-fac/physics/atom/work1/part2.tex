\section{Тепловое излучение.}

\begin{enumerate}
% 2.1 --------------
\item На поверхность с поглощательной способностью \( \alpha = 0,\!5 \),
находящуюся в равновесии с излучением, падает поток лучистой энергии
\( \Phi \). Какой поток \( \Phi \) распространяется от поверхности по всем
направлениям в пределах телесного угла \( 2\pi \)? За счёт чего образуется этот
поток?

\emph{Решение:}

\newpage

% 2.2 --------------
\item При переходе от температуры \( T_1 \) к температуре \( T_2 \) площадь,
ограниченная графиком функции распределения плотности энергии равновесного
излучения по длинам волн, увеличивается в 16~раз. Как изменяется при этом длина
волны \( \lambda_m \), на которую приходится максимум испускательной способности
абсолютно чёрного тела?

\emph{Решение:}

\newpage

% 2.3 --------------
\item Энергетическая светимость абсолютно чёрного тела
\( R^* = 250\text{~кВт/м}^2 \). На какую длину волны \( \lambda_m \) приходится
максимум испускательной способности этого тела?

\emph{Решение:}

\newpage

% 2.4 --------------
\item Найти среднюю энергию \( \midnum{\eps} \) квантового осциллятора при
температуре \( T \). Частота осциллятора равна \( \omega \). Вычислить среднюю
энергию \( \midnum{\eps} \) квантового осциллятора для:
\begin{enumerate}
    \item частоты \( \omega_1 \), отвечающей условию \( \hbar\omega_1 = kT \);
    \item частоты \( \omega_2 = 0,\!1\, \omega_1 \);
    \item частоты \( \omega_3 = 10\, \omega_1 \).
\end{enumerate}
Выразить \( \midnum{\eps} \) через \( kT \). Сравнить найденные значения со
средней энергией  \( \midnum{\eps}_\emph{кл} \) классического осциллятора.

\emph{Решение:}

\newpage

% 2.5 --------------
\item Найти:
\begin{enumerate}
    \item температурную зависимость частоты \( \omega_m \), на которую
    приходится максимум функции \( f(\omega, T) \) определяющей испускательную
    способность абсолютно чёрного тела;
    \item значение произведения \( \lambda_m\omega_m \), где \( \lambda_m \) --
    длина волны, отвечающая максимуму функции \( \phi(\omega, T) \). Сравнить
    это значение с \( 2\pi c \).
\end{enumerate}

\emph{Решение:}

\newpage

% 2.6 --------------
\item Поверхность Солнца близка по своим свойствам к абсолютно чёрному телу.
Максимум испускательной способности приходится на длину волны \( \lambda_m =
0,\!50 \)~мкм (в излучении Солнца, прошедшем через атмосферу и достигшем
поверхности Земли, максимум приходится на \( \lambda = 0,\!55 \)~мкм).
Определить:
\begin{enumerate}
    \item температуру \( T \) солнечной поверхности;
    \item энергию \( E \), излучаемую Солнцем за 1 секунду за счёт излучения;
    \item массу \( m \), теряемую Солнцем в 1 секунду за счёт излучения;
    \item примерное время \( \tau \), за которое масса Солнца уменьшилась бы за
    счёт излучения на 1\%, если бы температура Солнца оставалась постоянной.
\end{enumerate}

\emph{Решение:}

\newpage

% 2.7 --------------
\item Полагая, что Солнце обладает свойствами абсолютно чёрного тела,
определить интенсивность \( I \) солнечного излучения вблизи Земли за пределами
её атмосферы (эта интенсивность называется солнечной постоянной). Температура
солнечной поверхности \( T = 5785 \)~К.

\emph{Решение:}

\newpage

% 2.8 --------------
\item На корпусе космической лаборатории, летящей вокруг Солнца по круговой
орбите, радиус которой \( R \) равен среднему расстоянию от Земли до Солнца,
установлено устройство, моделирующее абсолютно чёрное тело. Наружная поверхность
оболочки этого устройства является идеально отражающей. Отверстие в оболочке всё
время обращено к Солнцу. Пренебрегая теплообменом через крепление устройства к
корпусу лаборатории, определить равновесную температуру \( T \), которая установится
внутри устройства. Температуру солнечной поверхности \( T_C \) принять равной 5800~К.

\emph{Решение:}

\newpage

% 2.9 --------------
\item Начальная температура теплового излучения \( T = 2000 \)~К. На сколько
кельвинов изменилась эта температура, если наиболее вероятная длина волны в его
спектре увеличилась на \( \Delta\lambda = 0,\!25 \)~мкм?

\emph{Решение:}

\newpage

% 2.10 -------------
\item Найти наиболее вероятную длину волны в спектре теплового излучения с
энергетической светимостью \( M = 5,\!7\text{~Вт/см}^2 \).

\emph{Решение:}

\newpage

% 2.11 -------------
\item Зная, что давление теплового излучения \( p = u/3 \), где \( u \) --
плотность энергии излучения найти:
\begin{enumerate}
    \item давление теплового излучения во внутренних областях Солнца, где
    температура \( T \approx 1,6\cdot 10^7 \)~К;
    \item температуру полностью ионизированной водородной плазмы плотностью
    \( \rho = 0,\!10\text{~г/см}^3 \), при которой давление излучения равно
    кинетическому давлению частиц плазмы (при высоких температурах вещества
    подчиняются уравнению состояния для идеальных газов).
\end{enumerate}

\emph{Решение:}

\newpage

% 2.12 -------------
\item Медный шарик радиусом \( r = 10,\!0 \)~мм с абсолютно черной поверхностью
поместили в откачанный сосуд, температура стенок которого поддерживается близкой
к абсолютному нулю. Начальная температура шарика \( T_0 = 300 \)~К. Через
сколько времени его температура уменьшится в \( n = 1,\!50 \)~раза? Удельная
теплоёмкость меди \( c = 0,\!38 \)~Дж/(г\( \cdot \)К).

\emph{Решение:}

\newpage

% 2.13 -------------
\item Вычислить с помощью формулы Планка мощность излучения единицы поверхности
абсолютно черного тела в интервале длин волн, отличающихся не более чем на
\( \eta = 0,\!50\% \) от наиболее вероятной длины волны при \( T = 2000 \)~К.

\emph{Решение:}

\newpage

% 2.14 -------------
\item Имеется два абсолютно чёрных источника теплового излучения. Температура
одного из них \( T_1 = 2500 \)~К. Найти температуру другого источника, если
длина волны, отвечающая максимуму его испускательной способности, на
\( \Delta\lambda = 0,\!50 \)~мкм больше длины волны, соответствующей максимуму
испускательной способности первого источника.

\emph{Решение:}

\newpage

% 2.15 -------------
\item Температура поверхности Солнца \( T_0 = 5500 \)~К. Считая, что
поглощательная способность Солнца и Земли равна единице и что Земля находится в
состоянии теплового равновесия, оценить её температуру.

\emph{Решение:}

\newpage

% 2.16 -------------
\item Полость объёмом \( V = 1,\!0 \)~л заполнена тепловым излучением при
температуре \( T = 1000 \)~К. Найти:
\begin{enumerate}
    \item теплоёмкость \( C_V \);
    \item энтропию \( S \) этого излучения.
\end{enumerate}

\emph{Решение:}

\newpage

% 2.17 -------------
\item Найти уравнение адиабатического процесса (в переменных \( V \), \( T \)),
проводимого с тепловым излучением, имея в виду, что между давлением и плотностью
энергии теплового излучения существует связь \( p = u/3 \).

\emph{Решение:}

\end{enumerate}
