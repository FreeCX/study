\chapter{Генетический триггер Жакоба-Моно}
	
В 1962 году Жакобом и Моно была предложена биохимическая схема 
регуляции белкового синтеза.Ими же была предложена триггерная схема 
перекрёстной регуляции двух генов. Рассмотрим математическую модель, 
соответствующей этой схеме и опишем последовательно стадии регуляции 
синтеза белков.

\chapter{Синтез m-РНК}

Синтез m-РНК описывается системой уравнений:
\[
    \left\{ \begin{array}{ll}
        \cfrac{dI}{dt} = \cfrac{1}{\tau_0}W_2 - \cfrac{1}{\tau_1}I \\
        \cfrac{dW_1}{dt} = -a_2 W_1 + \cfrac{1}{\tau_0}W_2 -
            f(r)W_1 + a_3 W_3 \\
        \cfrac{dW_2}{dt} = a_2 W_1 - \cfrac{1}{\tau_0}W_2 \\
        \cfrac{dW_3}{dt} = -a_3 W_3 + f(r)W_1
    \end{array}\right. \eqno{1}
\]

Здесь \( W_1 \) -- вероятность застать оперон и весь участок гена 
свободным, \( W_2 \) -- вероятность застать оперон и весь участок гена в
рабочем состоянии, \( W_3 \) -- вероятность того, что оперон 
зарепрессирован. Член \( W_2 / \tau_0 \) описывает синтез m-РНК; 
\( \tau_0 \) -- время синтеза (величина порядка \( \approx 1-3 \) сек). 
Член \( I / \tau_1 \) описывет распад РНК под действием рибонуклеаз. 
\( I \) -- усредненная по объёму клетки концентрация m-РНК; 
\( \tau_1 \) -- время жизни m-РНК (величина порядка \( \approx \) 10 мин).
Член \( a_2 W_1 \) описывает взаимодейтсвие свободного оперона с 
РНК-полимеразой. Члены \( f(r) W_1 \) и \( a_3 W_3 \) описывают репрессию 
и дерепресси. 

Упростим систему \( (1) \). Учтём, что сумма вероятностей всех возможных 
состояний равна единице
\[ 
    W_1 + W_2 + W_3 = 1
\]

В результате чего система уравнений \( (1) \) упрощается до трёх.

Исследуем стационарное состояние, в котором концентрации равны
\[
    \begin{array}{ll}
        \bar{W}_1 = \cfrac{a_3}{\tau a_2 a_3 + f(r) + a_3}; \quad
        \bar{W}_2 = \cfrac{\tau_0 a_2 a_3}{\tau_0 a_2 a_3 + f(r) + a_3} \\
        \bar{W}_3 = \cfrac{f(r)}{\tau_0 a_2 a_3 + f(r) + a_3}; \quad
        \bar{I} = \cfrac{\tau_1 a_2 a_3}{\tau_0 a_2 a_3 + f(r) + a_3}
    \end{array}
    \eqno{2}
\]

И вводя безразмерные переменные
\[
    u = \frac{W_2}{\bar{W}_2};\quad
    v = \frac{W_3}{\bar{W}_3};\quad
    z = \frac{I}{\bar{I}};\quad
    t' = \frac{t}{\tau_1}
\]

преобразуем систему к виду:
\[
    \left\{ \begin{array}{ll}
        \cfrac{dz}{dt'} = u - z \\
        \cfrac{du}{dt'} = \cfrac{\tau_1}{\tau_0}
            \left[ 
                \cfrac{f(r)}{a_2} + 1 + a_2 \tau_0 - 
                \left( 1 + a_2 \tau_0 \right)u - \cfrac{f(r)}{a_2}v 
            \right] \\
        \cfrac{dv}{dt'} = \cfrac{\tau_1}{\tau_0}
            \left\{
                    a_2 a_3 \tau_0^2 + \tau_0\left[ f(r) + a_3 \right]
                    \left( 1 - v \right) - a_2 a_3 \tau_0^2 u
            \right\} 
    \end{array} \right. \eqno{3}
\]

Величина \( \tau_1 / \tau_0 \) имеет порядок \( 10^2 \div 10^3 \gg 1 \), 
остальные комбинации коэффициентов порядка единицы. Таким образом, 
переменные \( u \) и \( v \) изменяются быстро по сравнению с величиной 
\( z \), то есть два последних уравнений представляют собой присоединенную 
систему. Используя теорему Тихонова систему \( (3) \) можно заменить 
одним уравнением:
\[
    \frac{dz}{dt'} = 1 - z
\]

И возвращаясь к размерным переменным, получим:
\[
    \frac{dI}{dt} = \frac{a_2 a_3}{\tau_0 a_2 a_3 + f(r) + a_3} -
    \frac{1}{\tau_1}I \eqno{4}
\]

\chapter{Образование активного репрессора}

Примем, что неактивная форма продуцируется с постоянной скоростью 
\( v \) и распадается со скоростью \( \kappa_p r_p \). Кроме того, примем, 
что при участии корепрессора \( R \) репрессор может обратимо переходить 
в активную форму \( r_a \). Тогда реакция может быть представлена в 
виде:
\[
    nr_p + mR \xrightleftarrow[k_{-}]{k_{+}} (r_p)_n R_m \equiv r_a
\]

Здесь \( n \) -- число субъединиц в активной форме репрессора, 
\( m \) -- число молекул корепрессора входящих в комплекс. Уравнения 
этих процессов можно записать в виде:
\[
    \left\{ \begin{array}{ll}
        \cfrac{dr_p}{dt} = v - \kappa_p r_p - nk_{+} r^n_p R^m + 
            nk_{-} r_a \\
        \cfrac{dr_a}{dt} = - k_{-} r_a + k_{+} r^n_p R^m - 
            \kappa_a r_a
    \end{array} \right. \eqno{5}
\]

Так как процессы ассоциации и диссоциации протекают существеноо быстрее 
синтеза или протеолиза, можно воспользоваться теоремой Тихонова и 
заменить систему \( (5) \):
\[
    \left\{ \begin{array}{ll}
        \cfrac{a(r_p + nr_a)}{dt} = v - \kappa_p r_p - n\kappa_a r_a \\
        \cfrac{r^n_p R^m}{r_a} = K_r = \cfrac{k_{-}}{k_{+}}
    \end{array} \right. \eqno{6}
\]

Рассмотрим случай, когда и скорость синтеза, и скорость распада репрессоров 
очень малы
\[
    \frac{d}{dt}\left( r_p + nr_a \right) \approx 0 
\]

Рассходы энергии и материалов на синтез репрессора при этом минимальны, а 
эффективность и скорость регуляции достаточно высоки. Введём величину 
\( r_0 \) -- суммарное число субъединиц репрессора. В этом случае 
положим
\[
    r_p = r_0 - r_a n;\quad
    K_r r_a = (r_0 - r_a n)^n R^m
\]

Отсюда легко находятся основные черты зависимости \( r_a \) от \( R \): 
при \( R \rightarrow \infty \) концентрация \( r_a \rightarrow r_0 \ n \), 
то есть имеет место насыщение; при малых \( R \) концентрация активного 
репрессора \( r_a \approx r^n_o R^m \ K_r \). Далее для упрощения будем 
пологать, что \( r_a \sim R^m \) и \( f(r) = k_0 R^m \).

\chapter{Синтез фермента E и катализируемого продукта P}

Рассмотрим модель синтаза фермента \( E \) и катализируемого им продукта 
\( P \). Уравнения запишутся в виде:
\[
    \left\{ \begin{array}{ll}
        \cfrac{dE}{dt} = \cfrac{1}{\tau_E}I - \frac{1}{\tau_2}E \\
        \cfrac{dP}{dt} = E\cfrac{k_{+2}S}{K_S + S} - qP \\
        \cfrac{dI}{dt} = Q(r) - \frac{1}{\tau_1}I
    \end{array} \right. \eqno{7}
\]
где \( \tau_E \) -- время синтеза молекулы фермента \( E \). Член 
\( E/tau_2 \) описывает распад фермента за счёт протеолиза, член
\( k_{+2}ES / (K_S + S) \) -- синтез продукта \( P \) из субстрата \( S \) 
с участием фермента \( E \), член \( qP \) -- отток продукта \( P \).

Последнее уравнение описывет синтез m-РНК. Тогда согласно уравнению 
\( (4) \) и \( f(r) = k_0 R^m \) скорость синтеза будет:
\[
    Q(r) = \frac{a_2 a_3}{\tau_0 a_2 a_3 + a_3 + k_0 R^m}
\]

В стационарных условиях
\[
    \bar{E} = \frac{\tau_2}{\tau_E}\bar{I};\quad
    \bar{P} = \frac{k_{+2}S}{q(K_S + S)\bar{E}};\quad
    \bar{I} = Q\tau_1
\]
и введя обозначения
\[
    x = \frac{E}{\bar{E}};\quad
    y = \frac{P}{\bar{P}};\quad
    z = \frac{I}{\bar{I}};\quad
    t'' = tq
\]
перепишем систему \( (7) \) в виде:
\[
    \left\{ \begin{array}{ll}
        \cfrac{dx}{dt''} = \cfrac{1}{\tau_2 q}\left( z - x \right) \\
        \cfrac{dy}{dt''} = x - y \\
        \cfrac{dz}{dt''} = \cfrac{1}{\tau_1 q}\left( 1 - z \right)
    \end{array} \right. \eqno{8}
\]

Величины \( (\tau_2 q)^{-1} \) и \( (\tau_1 q)^{-1} \) можно считать 
достаточно большими, а первое и третье уравнение -- присоединенными, и 
используя теорему Тихонова приходим к простой модели из одного уравнения:
\[
    \frac{dy}{dt''} = 1 - y
\]

Возвращаясь к прежним обозначениям:
\[
    \frac{dP}{dt} = \frac{k_{+2}S}{K_S + S}\frac{\tau_1\tau_2}{\tau_E}
        \frac{a_2 a_3}{\tau_0 a_2 a_3 + a_3 + k_0 R^m} - qP \eqno{9}
\]

В случае когда \( \tau_2 q \sim 1 \) оставляем два уравнения:
\[
    \left\{ \begin{array}{ll}
        \cfrac{dx}{dt''} = \cfrac{1}{\tau_2 q}\left( 1 - x \right) \\
        \cfrac{dy}{dt''} = x - y
    \end{array} \right. \eqno{10}
\]

или в размерном виде:
\[
    \left\{ \begin{array}{ll}
        \cfrac{dE}{dt} = \cfrac{\tau_1 a_2 a_3}
            {\tau_E\left( \tau_0 a_2 a_3 + a_3 + k_o R^m \right)}
            - \cfrac{1}{\tau_2}E \\
        \cfrac{dP}{dt} = \cfrac{k_{+2}S}{K_S + S} E - qP
    \end{array} \right. \eqno{11}
\]

\chapter{Триггерная схема Жакоба-Моно}

Модели описываемые уравнениями \( (9) \) и \( (11) \) не являются 
замкнутыми, поскольку не указано, как связаны концентрация активного 
репрессора \( r_a \) и субстратар \( S \) с концентрацией продукта \( P \). 
Кроме того, они описывают поведение одного гена, в то время важно учесть 
влияние работы одного гена на другой. В этом случае величины \( P \) и 
\( S \) могут быть связаны с продуктами деятельности другого гена. В 
качестве простейшего варианта Жакоб и Моно предложили триггерную схему. 
В ней корепрессором первой системы является продукт второй системы 
\( P_2 \) и корепрессором второй -- \( P_1 \).

Рассмотрим случай, когда концентрация субстратов \( S_1 \) и \( S_2 \) 
постоянны. Модель должна состоять из уравнений, описывающих синтез 
\( P_1 \) и \( P_2 \); при этом согласно схеме положим:
\[
    R_1 \equiv P_2;\quad R_2 \equiv P_1
\]

Принимая, что синтез продуктов описывается уравнением \( (9) \), запишем 
модель в виде:
\[
    \left\{ \begin{array}{ll}
        \cfrac{dP_1}{dt} = \cfrac{A_1}{B_1 + P_2^m} - q_1 P_1 \\
        \cfrac{dP_2}{dt} = \cfrac{A_2}{B_2 + P_1^m} - q_2 P_2
    \end{array} \right. \eqno{12}
\]

Здесь величины \( A \) и \( B \) выражаются через параметры своих систем 
следующим образом:


\pagebreak

\chapter{Список использованной литературы}
    \begin{enumerate}
        \item Ризниченко, Г.Ю. Лекции по математическим моделям в 
            биологии. Ч.1. -- Ижевск: НИЦ
            <<Регулярная и хаотическая динамика>>, 2002, 232 с.
        \item Романовский, Ю.М. Математическое моделирование в биофизике
            -- Москва: <<Наука>>, 1975, 335 с.
    \end{enumerate}
