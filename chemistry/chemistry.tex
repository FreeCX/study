\input{../../.preambles/01-semester_work}
\input{../../.preambles/10-russian}
\input{../../.preambles/20-math}
\input{../../.preambles/30-physics}

\begin{document}

\maketitlepagewithvariant{Химико-технологический факультет}
{общая и неорганическая химия}{Общая и неорганическая химия}
{студент группы Ф-369\\Голубев~А.~В.}{старший преподаватель Гаджиева~Н.~Х.}
{}{6}

\newpage

\emph{1. На нейтрализацию 7.33 г фосфорной кислоты пошло 4.44 г 
гидроксида натрия \( NaOH \). Вычислите эквивалентную массу 
\( H_3 PO_4 \)}

\emph{Решение:}

\emph{Ответ: } \\\\

%--------------------------------------------------------------------------

\emph{2. Определите эквивалент и эквивалентную массу \( Ca(HCO_3)_2 \), 
\( Al(OH)_2Cl \) в реакциях:}
\begin{enumerate}
    \item \( Ca(HCO_3)_2 + 2HCl = CaCl_2 + H_2CO_3 \)
    \item \( Al(OH)_2Cl + KOH = Al(OH)_3 + KCl \)
\end{enumerate}

\emph{Решение:}

\emph{Ответ: } \\\\

%--------------------------------------------------------------------------

\emph{3. Какой объём (н.у.) занимают \( 3.01\cdot10^{21} \) молекул 
газа? Определите молекулярную массу газа, если вычисленный объём газа 
имеет массу 0.22 г.}

\emph{Решение:}

Для нахождения объёма газа воспользуемся формулами, выражающие связь 
между количеством вещества, массой, объёмом и количеством молекул:
\[
	\mu = \frac{m}{M} = \frac{N}{N_A} = \frac{V}{V_m}
\]
где \( N_A \) -- постоянная Авогадро, \( V_m \) -- молярный объём газа.
\[
	V = V_m \frac{N}{N_A} = 22.41 \text{ л} \cdot 
	\frac{3.01\cdot10^{21} \text{ шт}}{6.02\cdot10^{23} \text{ моль}^{-1}} =
	0.11 \text{ л}
\]

Воспользуемся первым уравнением в виде:
\[
	M = m\frac{N_A}{N} = 0.22 \text{ г} \cdot  
	\frac{6.02\cdot10^{23} \text{ моль}^{-1}}{3.01\cdot10^{21}} =
	44 \text{ г/моль}
\]

\emph{Ответ: } \( V = 0.11 \text{ л} \), 
\( M = 44 \text{ г/моль} \) \\\\

%--------------------------------------------------------------------------

\emph{4. Какой объем воды надо прибавить к 200 мл 48\% (по массе) 
раствора азотной кислоты (\(\rho = 1.3 \) г/мл), чтобы получить 
20\% (по массе) раствор?}

\emph{Решение:}

Массовая доля раствора при \( \omega \)
\[
	\omega = \frac{m}{V\cdot\rho} \eqno{1}
\]

При добавлении воды формула запишется в виде:
\[
	\omega_n = \frac{m}{V\cdot\rho + 
	V_\text{воды}\cdot\rho_\text{воды}} \eqno{2}
\] 

Подставляя (1) в (2) и преобразуя относительно \( V_\text{воды} \) получаем:
\[
	V_\text{воды} = \frac{V\rho}{\rho_\text{воды}}\left( 
		\frac{\omega}{\omega_n} - 1 \right) = 
	\left( \frac{0.48}{0.20} - 1 \right)\cdot 200 \cdot 
	\frac{1.3}{1} = 364
\]

\emph{Ответ: } \( V_\text{воды} = 364 \) г/мл \\\\

%--------------------------------------------------------------------------

\emph{5. В 800 мл раствора содержится 20.0 г \( H_2 SO_4 \). Вычислите 
молярную и нормальную концентрации раствора.}

\emph{Решение:}

Нормальная концентрация выражается формулой: 
\[ 
	C_\text{н} = \frac{m_\text{р.в.}}{M_\text{Э}\cdot V} 
\]

где \( M_\text{Э}(H_2 SO_4) = \cfrac{1}{2}\cdot 98 = 49 \) (г/моль).
\[
	C_\text{н} = \frac{20}{49\cdot0.8} = 0.51 \text{ (моль/л)} 	
\]

Молярная концентрация:
\[
	C_\text{м} = \frac{m_\text{р.в.}}{M\cdot V} 
\]

где \( M(H_2 SO_4) = 98.079 \) (г/моль).
\[
	C_\text{м} = \frac{20}{98.079\cdot0.8} = 0.255 \text{ (моль/л)} 
\]

\emph{Ответ: } \( C_\text{н} = 0.51 \) (моль/л); \( C_\text{м} = 0.255 \) 
(моль/л) \\\\

%--------------------------------------------------------------------------

\emph{6. В 200 мл раствора \( KOH \) содержится 10 г \( KOH \). 
Вычислите титр этого раствора.}

\emph{Решение:}

Воспользуемся формулой для определения титр раствора:
\[
	T = \frac{m}{V}
\]

Подставляем значения получаем:
\[
	T = \frac{10}{200} = 0.05 \quad(\text{г/мл})
\]

\emph{Ответ: } \( T = 0.05 \) г/мл \\\\

%--------------------------------------------------------------------------

\emph{7. Напишите электронную формулу элемента ртути \( Hg \). Укажите 
валентность в нормальном и возбужденном состояниях. Рассчитайте значение 
суммарного спина.}

\emph{Решение:}
\[
	Hg: 1s^2 2s^2 2p^6 3s^2 3p^6 4s^2 3d^{10} 4p^6 5s^2 4d^{10} 5p^6
        4f^{14} 5d^{10} 6s^2
\]

\emph{Ответ: } \\\\

%--------------------------------------------------------------------------

\emph{8. Определите тип гибридизации в молекуле \( SiBr_4 \) и укажите 
форму молекулы.}

\emph{Решение:}

Рассмотрим электронную конфигурацию молекулы:
\[ Si: 3s^2 3p^2 \]
\[ Br: 3d^10 4s^2 4p^5 \]

\emph{Ответ: } \\\\

%--------------------------------------------------------------------------

\emph{9. Как изменится скорость реакции 
\( 2NO_{(\text{г})} + O_{2(\text{г})} = 2NO_{2(\text{г})} \), если:}
\begin{enumerate}
    \item увеличить давление в системе в 3 раза
    \item уменьшить концентрации всех веществ в 2 раза
\end{enumerate}

\emph{Решение:}

\emph{Ответ: } \\\\

%--------------------------------------------------------------------------

\emph{10. Определите, при какой температуре будет кристаллизоваться 40\% 
(по массе) раствор этилового спирта \( C_2H_5OH \).}

\emph{Решение:}

\emph{Ответ: } \\\\

%--------------------------------------------------------------------------

\emph{11. Давление пара раствора, содержащего 22.1 г \( CaCl_2 \) в 
1000 г воды при \( 20 ^{\circ}С \) равно 2.32 кПа. Вычислите степень 
электрической диссоциации.}

\emph{Решение:}

\emph{Ответ: } \\\\

%--------------------------------------------------------------------------

\emph{12. Рассчитайте \( pH \) раствора, если концентрация ионов 
\( OH^{-} \) равна \( 10^{-5} \) моль/л.}

\emph{Решение:}

\emph{Ответ: } \\\\

%--------------------------------------------------------------------------

\emph{13. Определите \( \Delta H^{0}_{298} \) образования \( PH_3\), 
из уравнения 
\(
    2PH_{3(\text{г})} + 4O_{2(\text{г})} = 
    P_2 O_{5(\text{к})} + 3H_2 O_{(\text{ж})};\quad
    \Delta H = -2360 
\) кДж/моль.}

\emph{Решение:}

\emph{Ответ: } \\\\

%--------------------------------------------------------------------------

\emph{14. Напишите уравнение гидролиза в молекулярном и 
ионномолекулярном виде следующих солей: 
\( Ba(NO_2)_2, Na_2 SO_3, Cr_2 S_3 \).}

\emph{Решение:}

\emph{Ответ: } \\\\

%--------------------------------------------------------------------------

\emph{15. Какие процессы будут протекать в гальваническом элементе 
\( Mn/MnCl_2(0.001 M)//Sn(NO_3)_2(0.1 M)/Sn \)? 
Определите ЭДС гальванического элемента.}

\emph{Решение:}

\emph{Ответ: } \\\\

%--------------------------------------------------------------------------

\emph{16. Используя метод электронного баланса, расставьте коэффициенты 
в уравнении реакции, укажите окислитель и восстановитель:
\( V + HNO_3 + HCl = VO_2 Cl + NO + H_2 O \) }

\emph{Решение:}

\[ 
	\stackrel{}{V} + \stackrel{}{H}\stackrel{}{N}\stackrel{}{O}_3 + 
	\stackrel{}{H}\stackrel{}{Cl} = \stackrel{}{V}\stackrel{}{O}_2
	\stackrel{}{Cl} + 
	\stackrel{}{N}\stackrel{}{O} + \stackrel{}{H}_2\stackrel{}{O} 
\]
\[ 3V + 5HNO_3 + 3HCl = 3VO_2 Cl + 5NO + 4H_2 O \]

\emph{Ответ: } \\\\

%--------------------------------------------------------------------------

\emph{17. Составьте схему процессов, происходящих на электродах при 
электролизе расплава \( CaCl_2 \)}

\emph{Решение:}

\emph{Ответ: } \\\\

%--------------------------------------------------------------------------

\emph{18. Какая масса хлорида натрия разложилась при электролизе раствора 
этой соли, если на аноде выделилось 67.2 \( \text{м}^3 \) хлора, 
измеренного при нормальных условиях?}

\emph{Решение:}

\emph{Ответ: } \\\\

\end{document}
