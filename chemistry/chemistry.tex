\input{../.preambles/01-semester_work}
\input{../.preambles/10-russian}
\input{../.preambles/20-math}
\input{../.preambles/30-physics}

\begin{document}
\maketitlepage{Химико-технологический факультет}{общая и неорганическая химия}
{Общая и неорганическая химия}{}{6}{студент группы Ф-369\\Голубев~А.~В.}
{}{старший преподаватель \\Гаджиева~Н.~Х.}{}{}

\pagebreak

\emph{1. На нейтрализацию 7.33 г фосфорной кислоты пошло 4.44 г 
гидроксида натрия \( NaOH \). Вычислите эквивалентную массу 
\( H_3 PO_4 \)}

\emph{Решение:}

Реакция имеет вид:
\[
	HPO_3 + NaOH \rightarrow H_3 PO_4 + H_2 O
\]
\[
	\frac{m(HPO_3)}{m(NaOH)} = \frac{M_\text{э1}(H_3 PO_4)}{M_\text{э2}(H_2 O)}
\]

Отсюда получаем:
\[
	M_\text{э1}(H_3 PO_4) = \frac{m(HPO_3}{m(NaOH)}\cdot M_\text{э2}(H_2 O) =
	\frac{7.33}{4.44}\cdot\frac{16}{2} \approx 13.21 \text{ г/моль}
\]

\pagebreak
%--------------------------------------------------------------------------

\emph{2. Определите эквивалент и эквивалентную массу \( Ca(HCO_3)_2 \), 
\( Al(OH)_2Cl \) в реакциях:}
\begin{enumerate}
    \item \( Ca(HCO_3)_2 + 2HCl = CaCl_2 + H_2CO_3 \)
    \item \( Al(OH)_2Cl + KOH = Al(OH)_3 + KCl \)
\end{enumerate}

\emph{Решение:}

Эквивалент в первой реакции:
\[
	\text{Э}=\text{Э}(Ca(HCO_3)_2) = \frac{1}{2}
\]

тогда эквивалентная масса:
\[
	M_\text{Э} = M(Ca(HCO_3)_2) \cdot \text{Э} = \cfrac{162.12}{2} = 
	81.06 \text{ г/моль}
\]

Во второй реакции
\[
	\text{Э}=\frac{1}{2}\text{Э}(Al(OH)_2 Cl) = 1
\]
\[
	M_\text{Э} = M(Ca(HCO_3)_2) \cdot \text{Э} = 95.5 \text{ г/моль}
\]

\pagebreak
%--------------------------------------------------------------------------

\emph{3. Какой объём (н.у.) занимают \( 3.01\cdot10^{21} \) молекул 
газа? Определите молекулярную массу газа, если вычисленный объём газа 
имеет массу 0.22 г.}

\emph{Решение:}

Для нахождения объёма газа воспользуемся формулами, выражающие связь 
между количеством вещества, массой, объёмом и количеством молекул:
\[
	\mu = \frac{m}{M} = \frac{N}{N_A} = \frac{V}{V_m}
\]
где \( N_A \) -- постоянная Авогадро, \( V_m \) -- молярный объём газа.
\[
	V = V_m \frac{N}{N_A} = 22.41 \text{ л} \cdot 
	\frac{3.01\cdot10^{21} \text{ шт}}{6.02\cdot10^{23} \text{ моль}^{-1}} =
	0.11 \text{ л}
\]

Воспользуемся первым уравнением в виде:
\[
	M = m\frac{N_A}{N} = 0.22 \text{ г} \cdot  
	\frac{6.02\cdot10^{23} \text{ моль}^{-1}}{3.01\cdot10^{21}} =
	44 \text{ г/моль}
\]

\pagebreak
%--------------------------------------------------------------------------

\emph{4. Какой объем воды надо прибавить к 200 мл 48\% (по массе) 
раствора азотной кислоты (\(\rho = 1.3 \) г/мл), чтобы получить 
20\% (по массе) раствор?}

\emph{Решение:}

Массовая доля раствора при \( \omega \)
\[
	\omega = \frac{m}{V\cdot\rho} \eqno{1}
\]

При добавлении воды формула запишется в виде:
\[
	\omega_n = \frac{m}{V\cdot\rho + 
	V_\text{воды}\cdot\rho_\text{воды}} \eqno{2}
\] 

Подставляя (1) в (2) и преобразуя относительно \( V_\text{воды} \) получаем:
\[
	V_\text{воды} = \frac{V\rho}{\rho_\text{воды}}\left( 
		\frac{\omega}{\omega_n} - 1 \right) = 
	\left( \frac{0.48}{0.20} - 1 \right)\cdot 200 \cdot 
	\frac{1.3}{1} = 364 \text{ (г/мл)}
\]

\pagebreak
%--------------------------------------------------------------------------

\emph{5. В 800 мл раствора содержится 20.0 г \( H_2 SO_4 \). Вычислите 
молярную и нормальную концентрации раствора.}

\emph{Решение:}

Нормальная концентрация выражается формулой: 
\[ 
	C_\text{н} = \frac{m_\text{р.в.}}{M_\text{Э}\cdot V} 
\]

где \( M_\text{Э}(H_2 SO_4) = \cfrac{1}{2}\cdot 98 = 49 \) (г/моль).
\[
	C_\text{н} = \frac{20}{49\cdot0.8} = 0.51 \text{ (моль/л)} 	
\]

Молярная концентрация:
\[
	C_\text{м} = \frac{m_\text{р.в.}}{M\cdot V} 
\]

где \( M(H_2 SO_4) = 98.079 \) (г/моль).
\[
	C_\text{м} = \frac{20}{98.079\cdot0.8} = 0.255 \text{ (моль/л)} 
\]

\pagebreak
%--------------------------------------------------------------------------

\emph{6. В 200 мл раствора \( KOH \) содержится 10 г \( KOH \). 
Вычислите титр этого раствора.}

\emph{Решение:}

Воспользуемся формулой для определения титр раствора:
\[
	T = \frac{m}{V}
\]

Подставляем значения получаем:
\[
	T = \frac{10}{200} = 0.05 \quad(\text{г/мл})
\]

\pagebreak
%--------------------------------------------------------------------------

\emph{7. Напишите электронную формулу элемента ртути \( Hg \). Укажите 
валентность в нормальном и возбужденном состояниях. Рассчитайте значение 
суммарного спина.}

\emph{Решение:}
\[
	Hg: 1s^2 2s^2 2p^6 3s^2 3p^6 4s^2 3d^{10} 4p^6 5s^2 4d^{10} 5p^6
        4f^{14} 5d^{10} 6s^2
\]
Элемент ртути в нормальном состоянии: 
\(
	\vert\uparrow\downarrow\vert\quad\quad\vert\quad\quad\vert\quad\quad\vert
\) \\
Элемент ртути в возбужденном состоянии:
\(
	\vert\uparrow\quad\vert\uparrow\quad\vert\quad\quad\vert\quad\quad\vert
\) \\
Спин атома ртути \( S = \pm \cfrac{1}{2} \). Суммарный спин равен 1.

\pagebreak
%--------------------------------------------------------------------------

\emph{8. Определите тип гибридизации в молекуле \( SiBr_4 \) и укажите 
форму молекулы.}

\emph{Решение:}

Рассмотрим электронную конфигурацию молекулы:
\[ Si: 3s^2 3p^2 \]
\[ Br: 3d^10 4s^2 4p^5 \]

При соединении Si и Br получаем бинарное неоргоническое соединение с 
типом гибридизации: \( sp3 \) и формой молекулы: тетраэдер.

\pagebreak
%--------------------------------------------------------------------------

\emph{9. Как изменится скорость реакции 
\( 2NO_{(\text{г})} + O_{2(\text{г})} = 2NO_{2(\text{г})} \), если:}
\begin{enumerate}
    \item увеличить давление в системе в 3 раза
    \item уменьшить концентрации всех веществ в 2 раза
\end{enumerate}

\emph{Решение:}
\[
	v = k \cdot [NO]^2 \cdot [O_2]
\]

При увеличении давления в системе в 3 раза получим увеличение скорости 
реакции в 27 раз. \\

При уменьшении концентрации всех веществ в 2 раза получим уменьшение 
скорости реакции в 4 раза.

\pagebreak
%--------------------------------------------------------------------------

\emph{10. Определите, при какой температуре будет кристаллизоваться 40\% 
(по массе) раствор этилового спирта \( C_2H_5OH \).}

\emph{Решение:}

Определим сколько нужно раствора этилового спирта \( m_1 \) добавить в 
килограмм воды \( m_2 \) для получения 40\% раствора:
\[
	K = \frac{m_1}{m_1 + m_2} \Rightarrow
	m_1 = \frac{K}{1-K}m_2 = \frac{0.4}{0.6}\cdot 1000 = 666.6 
	\text{ (г)}
\]
Молярная масса этилового спирта 
\( 12 \cdot 2 + 1 \cdot 5 + 16 \cdot 1 + 1 = 46 \) (г/моль). Поделим 
массу этилового спирта на молярную массу получим количество молей на 
килограмм растворителя \( 14.48 \) (моль). Умножив на 
криоскопическую константу ( для воды \( 1.86 \) ) и вычтем из 
температуры замерзания воды получим температуру кристаллизации 
этилового спирта \( \approx -27 ^\circ C\).

\pagebreak
%--------------------------------------------------------------------------

\emph{11. Давление пара раствора, содержащего 22.1 г \( CaCl_2 \) в 
1000 г воды при \( 20 ^{\circ}С \) равно 2.32 кПа. Вычислите степень 
электрической диссоциации.}

\emph{Решение:}

\pagebreak
%--------------------------------------------------------------------------

\emph{12. Рассчитайте \( pH \) раствора, если концентрация ионов 
\( OH^{-} \) равна \( 10^{-5} \) моль/л.}

\emph{Решение:}
\( pOH \) раствора найдём используя формулу:
\[
	pOH = -\log{OH^{-}} = \log{10^5} = 5
\]

Откуда найдём \( pH \) через формулу
\[
	pH = 14 - pOH = 9
\]

\pagebreak
%--------------------------------------------------------------------------

\emph{13. Определите \( \Delta H^{0}_{298} \) образования \( PH_3\), 
из уравнения 
\(
    2PH_{3(\text{г})} + 4O_{2(\text{г})} = 
    P_2 O_{5(\text{к})} + 3H_2 O_{(\text{ж})};\quad
    \Delta H = -2360 
\) кДж/моль.}

\emph{Решение:}
\[
	\Delta H = \left( \Delta H_{P_2 O_5} + 3\cdot\Delta H_{H_2 O} \right) -
		\left( 2\cdot\Delta H_{PH_3} + 4\cdot\Delta H_{O_2} \right) 
\]
\[
	\begin{array}{ll}
		\Delta H_{PH_3} = \cfrac{ \Delta H - \Delta H_{P_2 O_5} - 
		3\cdot\Delta H_{H_2 O}}{2} + 2\cdot\Delta H_{O_2} = \\
		\cfrac{-2360 + 1492 + 3\cdot285.8}{2} + 2\cdot0 \approx -5.30 
	\end{array}
\]

\pagebreak
%--------------------------------------------------------------------------

\emph{14. Напишите уравнение гидролиза в молекулярном и ионномолекулярном 
виде следующих солей: \( Ba(NO_2)_2, Na_2 SO_3, Cr_2 S_3 \).}

\emph{Решение:}
\[
	\left\{ \begin{array}{ll}
		Ba(2+) + 2NO2(-) + 2HOH \leftrightarrows Ba(2+) + 2HNO_2 + 2OH(-) \\ 
		Ba(NO_2)_2 + 2H_2 O \leftrightarrows Ba(OH)_2 + 2HNO_2
	\end{array} \right.
\]
\[
	\left\{ \begin{array}{ll}
		\text{I: } Na_2 SO_3 + HOH = NaHSO_3 + NaOH \\
		\quad 2\stackrel{+}{Na} + \stackrel{2-}{SO_3} + H_2 O 
			\leftrightarrows \stackrel{+}{Na} + H\stackrel{-}{SO_3} + 
			\stackrel{+}{Na} +  \stackrel{-}{OH} \\
		\quad \stackrel{2-}{SO_3} + H_2 0 \leftrightarrows 
			H\stackrel{-}{SO_3} + \stackrel{-}{OH} \\
		\text{II: } NaHSO_3 + HOH \leftrightarrows NaOH + H_2 SO_3 \\
		\quad \stackrel{+}{Na} + H\stackrel{SO_3} + HOH \leftrightarrows 
			\stackrel{+}{Na} + \stackrel{-}{OH} + H_2 SO_3 \\
		\quad H\stackrel{-}{SO_3} + HOH \leftrightarrows 
			\stackrel{-}{OH} + H_2 SO_3 
	\end{array} \right.
\]
\[
	\left\{ \begin{array}{ll}
		Cr(3+) + HOH \leftrightarrows CrOH(2+) + H(+) \\
		S(2-) + HOH \leftrightarrows HS(-) + OH(-)
	\end{array} \right.
\]

\pagebreak
%--------------------------------------------------------------------------

\emph{15. Какие процессы будут протекать в гальваническом элементе \\
\( Mn \vert MnCl_2(0.001 M) \vert\vert Sn(NO_3)_2(0.1 M) \vert Sn \)? 
Определите ЭДС гальванического элемента.}

\emph{Решение:}

\[
	\left\{ \begin{array}{ll}
		Mn \vert \stackrel{2+}{Mn} \vert\vert \stackrel{2+}{Sn} \vert Sn \\
		Mn - 2\stackrel{-}{e} = \stackrel{2+}{Mn} \\
		\stackrel{2+}{Sn} + 2\stackrel{-}{e} = Sn \\
		Mn + \stackrel{2+}{Sn} = \stackrel{2+}{Mn} + Sn 
	\end{array} \right.
\]

ЭДС гальванического элемента:
\[
	\eps = \phi_{Mn/\stackrel{2+}{Mn}} - \phi_{Sn/\stackrel{2+}{Sn}} = 
	1.18 - 0.136 = 1.044
\]

\pagebreak
%--------------------------------------------------------------------------

\emph{16. Используя метод электронного баланса, расставьте коэффициенты 
в уравнении реакции, укажите окислитель и восстановитель:
\( V + HNO_3 + HCl = VO_2 Cl + NO + H_2 O \) }

\emph{Решение:}

\[ 
	\stackrel{0}{V} + \stackrel{+}{H}\stackrel{5+}{N}\stackrel{2-}{O_3} + 
	\stackrel{+}{H}\stackrel{-}{Cl} = \stackrel{5+}{V}\stackrel{2-}{O}_2
	\stackrel{-}{Cl} + \stackrel{2+}{N}\stackrel{2-}{O} + 
	\stackrel{+}{H}_2\stackrel{2-}{O} 
\]
\[ 3V + 5HNO_3 + 3HCl = 3VO_2 Cl + 5NO + 4H_2 O \]

Окислитель: \( HNO_3 \)

Восстановитель: \( V \) \\ 

\pagebreak
%--------------------------------------------------------------------------

\emph{17. Составьте схему процессов, происходящих на электродах при 
электролизе расплава \( CaCl_2 \)}

\emph{Решение:}

\pagebreak
%--------------------------------------------------------------------------

\emph{18. Какая масса хлорида натрия разложилась при электролизе раствора 
этой соли, если на аноде выделилось 67.2 \( \text{м}^3 \) хлора, 
измеренного при нормальных условиях?}

\emph{Решение:}

\end{document}
