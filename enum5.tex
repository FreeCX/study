\input{../university/.preambles/01-semester_work}
\input{../university/.preambles/10-russian}
\input{../university/.preambles/20-math}
\input{../university/.preambles/30-physics}
\begin{document}
	\textbf{WARNING}: я не даю гарантию, что задачи решены на 100\% верно!!! \\
	\textbf{Последнее обновление:} 23:50 12/12/12 \\\\
	\textbf{5.1: }
		Найти магнитный момент \( \mu \) и возможные проекции \( \mu_z \)
		атома в состоянии:
		\vspace*{-1em}
		\begin{itemize}\itemsep-8pt
			\item[а)] \( ^1F \)
			\item[б)] \( ^2D_{3/2} \)
		\end{itemize}
		Основные формулы:
		\[ 
			\mu_L = -\mu_\text{Б}\sqrt{L(L+1)};\quad
			\mu_{LZ} = m_L\mu_\text{Б};\quad
			m_L = 0, \pm1, \pm2, ..., \pm L 
		\]
		\[ 
			\mu_S = -2\mu_\text{Б}\sqrt{S(S+1)};\quad
			\mu_{SZ} = 2m_S\mu_\text{Б};\quad
			m_S = -S, -S+1, ..., +S  
		\]
		\[ 
			\mu_J = -\mu_\text{Б}g\sqrt{J(J+1)};\quad
			\mu_{JZ} = m_J g\mu_\text{Б};\quad
			m_J = -J, -J+1, ..., +J  
		\]
		\[
			g = 1 + \cfrac{J(J+1)+S(S+1)-L(L+1)}{2J(J+1)} 
		\]
	\begin{itemize}\itemsep-8pt
		\item[а)] \( ^1F: L = 3, S = 0, J = 3 \)
			\[ g = 0 \]
			\[ \mu_L = -2\mu_\text{Б}\sqrt{3} \]
			\[ \mu_S = 0 \]
			\[ \mu_J = 0 \]
			\[ m_L = 0, \pm1, \pm2, \pm3 \]
			\[ m_S = 0 \]
			\[ m_J = -3, -2, -1, 0, 1, 2, 3 \]
		\item[б)] \( ^2D_{3/2}: L = 2, S = \cfrac{1}{2}, J = \cfrac{3}{2} \)
			\[ g = \cfrac{12}{5} \]
			\[ \mu_L = -\mu_\text{Б}\sqrt{6} \]
			\[ \mu_S = -\mu_\text{Б}\sqrt{3} \]
			\[ \mu_J = -\cfrac{6}{5}\mu_\text{Б}\sqrt{15} \]
			\[ m_L = 0, \pm1, \pm2 \]
			\[ m_S = -\cfrac{1}{2}, \cfrac{1}{2} \]
			\[ m_J = -\cfrac{3}{2}, -\cfrac{1}{2}, \cfrac{1}{2}, \cfrac{3}{2} \]
	\end{itemize}

	\textbf{5.2: }
		Вычислить магнитный момент атома водорода в основном сотоянии.\\
		Используем формулы из предыдущей задачи.
		\[
			S = \cfrac{1}{2} \text{ из условия }
			m_S = \cfrac{1}{2} \text{ и } S = \sum{m_S} = \cfrac{1}{2} 
		\]
		\[ 
			L = 0;\quad
			J = L + S \Rightarrow J = \cfrac{1}{2}
		\]
		\[
			g = 2;\quad 
			\mu_L = 0;\quad
			\mu_S = -\mu_\text{Б}\sqrt{3};\quad
			\mu_J = -\mu_\text{Б}\sqrt{3};\quad
		\]

	\textbf{5.3: }
		Найти механические моменты атома в состояниях \( ^5F \) и \( ^7H \), 
		если известно, что в этих состояниях магнитные моменты равны нулю.\\
		Основные формулы:
		\[
			M_L = \hbar\sqrt{L(L+1)};\quad
			M_S = \hbar\sqrt{S(S+1)};\quad
			M_J = \hbar\sqrt{J(J+1)}
		\]
		\begin{itemize}\itemsep-8pt
			\item[а)] \( ^5F \)
				\[ L = 3,\quad S = 2,\quad J = 0 \] 
				\[ 
					M_L = 2\hbar\sqrt{3};\quad
					M_S = \hbar\sqrt{6};\quad
					M_J = 0
				\]
			\item[б)] \( ^7H \)
				\[ L = 4,\quad S = 3,\quad J = 0 \] 
				\[ 
					M_L = 2\hbar\sqrt{5};\quad
					M_S = 2\hbar\sqrt{3};\quad
					M_J = 0
				\]
		\end{itemize}

	\textbf{5.4 }
		Механический момент атома в состоянии \( ^3F \) прецессирует 
		в магнитном поле B = 500 Гс с угловой скоростью 
		\( \omega = 5.5\cdot10^9 \) рад/с. Определить механический 
		и магнитный моменты атома.
		\[ \mu_\text{Б} = \frac{e\hbar}{2m} \]
		\[ ^3F: S = 1, L = 3, J = L+S = 4 \]
		\[ \dot{\vec{M_J}} = \vec{N} = \vec{\mu_J}\times\vec{B} \]
		\[ M_J\cdot\frac{d\phi}{dt}\cdot\sin\alpha = \mu_J\cdot B\cdot\sin\alpha \]
		\[ M_J\omega = \mu_J\cdot B \]
		\[ M_J = \frac{\mu_JB}{\omega} \]
		\[ g = 1 + \frac{S(S+1)+J(J+1)-L(L+1)}{2J(J+1)} = \frac{5}{4} \]
		\[ \mu_J = \mu_\text{Б}g\sqrt{J(J+1)} = \frac{5\sqrt{5}}{2}\mu_\text{Б} \]
		В результате получаем:
		\[ M_J = \frac{5\sqrt{5}}{2}\cdot\frac{\mu_\text{Б}B}{\omega} \]

	\textbf{5.5 }
		Объяснить с помощью векторной модели, почему механический момент
		атома, находящегося в состоянии \(^6F_{1/2} \) прецессирует в 
		магнитном поле B с угловой скоростью \( \omega \), вектор 
		которого направлен противоположно вектору В.

	\textbf{5.6 }
		Узкий пучок атомов пропускают по методу Штерна и Герлаха через резко
		неоднородное магнитное поле. Определить: 
		\vspace*{-1em}
		\begin{itemize}\itemsep-8pt
			\item[а)] максимальные значения проекций магнитных моментов 
			атомов в состояниях \(^4F \), \( ^6S \) и \( ^5D \), если
			известно, что пучок расщепляется соответственно на 4, 6 и 9 компонент;
			\item[б)] на сколько компонент расщепится пучок атомов, 
			находящихся в состояниях \( ^3D_2 \) и \( ^5F_1 \) ?
		\end{itemize}
		\vspace*{-1em}
		\begin{itemize}\itemsep-8pt
			\item[а)] Используемые формулы:
			\[ 
				\mu_{LZ} = m_L\mu_\text{Б};\quad
				m_L = 0, \pm1, \pm2, ..., \pm L 
			\]
			\[ 
				\mu_{SZ} = 2m_S\mu_\text{Б};\quad
				m_S = -S, -S+1, ..., +S  
			\]
			\[ 
				\mu_{JZ} = m_J g\mu_\text{Б};\quad
				m_J = -J, -J+1, ..., +J  
			\]
			\[ g = 1 + \frac{S(S+1)+J(J+1)-L(L+1)}{2J(J+1)} \]

			Рассмотрим каждое состояние атома в отдельности:
			\[ ^4F: L=3, S=\frac{3}{2}, J= \frac{3}{2} \]
			\[ 
				g = 1 + \frac{\cfrac{3}{2}\cdot\cfrac{5}{2} 
				+ \cfrac{3}{2}\cdot\cfrac{5}{2} - 3\cdot4}{3\cdot\cfrac{5}{2}} 
				= \frac{2}{5}  
			\]
			\[ 
				m_L = 0, \pm1, \pm2, \pm3;\quad
				m_S = -\frac{3}{2}, -\frac{1}{2}, \frac{1}{2}, \frac{3}{2}
			\]
			\[
				m_J = -\frac{3}{2}, -\frac{1}{2}, \frac{1}{2}, \frac{3}{2} \leftarrow 
				\text{расщепление на 4 компоненты}
			\]
			\[ 
				\mu_{maxLZ} = 3\mu_\text{Б};\quad
				\mu_{maxSZ} = 3\mu_\text{Б};\quad
				\mu_{maxJZ} = \frac{6}{5}\mu_\text{Б};
			\] \\

			\[ ^6S: L=0, S=\frac{5}{2}, J= \frac{5}{2} \]
			\[ g = 1 + \frac{\cfrac{35}{4}+\cfrac{35}{4}}{\cfrac{35}{2}} = 2\]
			\[ 
				m_L = 0;\quad
				m_S = -\frac{5}{2}, -\frac{3}{2}, -\frac{1}{2}, 
				\frac{1}{2}, \frac{3}{2}, \frac{5}{2}
			\]
			\[
				m_J = -\frac{5}{2}, -\frac{3}{2}, -\frac{1}{2}, \frac{1}{2}, 
				\frac{3}{2}, \frac{5}{2} \leftarrow \text{расщепление на 6 компонент}
			\]
			\[ 
				\mu_{maxLZ} = 0;\quad
				\mu_{maxSZ} = 5\mu_\text{Б};\quad
				\mu_{maxJZ} = 10\mu_\text{Б}
			\]

			\[ ^5D: L=2, S=2, J=4\]
			\[ g = 1 + \frac{2\cdot3 + 4\cdot5 - 2\cdot3}{8\cdot5} = \frac{3}{2}\]
			\[ 
				m_L = 0, \pm1, \pm2;\quad
				m_S = -2, -1, 0, 1, 2
			\]
			\[
				m_J = -4, -3, -2, -1, 0, 1, 2, 3, 4 \leftarrow 
				\text{расщепление на 9 компонент}
			\]
			\[ 
				\mu_{maxLZ} = 2\mu_\text{Б};\quad
				\mu_{maxSZ} = \mu_\text{Б};\quad
				\mu_{maxJZ} = 3\mu_\text{Б}
			\]
			\item[б)] Запишем для каждого терма значение \( m_J \)
			\[ ^3D_2: L=2, S=1, J=2 \]
			\[ m_J = -2, -1, 0, 1, 2 \leftarrow \text{расщепление на 5 компонент} \]
			\[ ^5F_1: L=3, S=2, J=1 \]
			\[ m_J = -1, 0, 1 \leftarrow \text{расщепление на 3 компоненты} \]
		\end{itemize}

	\textbf{5.7 }
		Атом находится в магнитном поле B = 3,00 кГс. Определить: 
		\vspace*{-1em}
		\begin{itemize}\itemsep-8pt
			\item[а)] полное расщепление, \( \text{см}^{-1} \) , терма \( ^1D \) 
			\item[б)] спектральный символ синглетного терма,
			полная ширина расщепления которого составляет 0,84 \( \text{см}^{-1} \).
		\end{itemize}
		Используемые формулы: 
		\[ \Delta\omega = \frac{\mu_\text{Б}B}{\hbar}(m_{J2}g_{2}-m_{J1}g_{1}) \]
		\[ m_J = -J, -J+1, ..., +J \]
		\[ g = 1 + \frac{S(S+1)+J(J+1)-L(L+1)}{2J(J+1)} \]
		\begin{itemize}\itemsep-8pt
			\item[а)] 
			\[ 
				^1D: L=2, S=0, J=2;\quad
				m_J = -2, -1, 0, 1, 2;\quad
				g = 1
			\]
			В предположении, что полное расщепление образуется в случае разности
			\( m_J \) и постоянства числа \( g \), получаем:
			\[ 
				\Delta\omega = \frac{\mu_\text{Б}B}{\hbar}(2+2) 
				= 4\frac{\mu_\text{Б}B}{\hbar} 
			\]
			\item[б)]
		\end{itemize}

	\textbf{5.8 }
		Спектральная линия L = 0,612 мкм обусловлена переходом между двумя
		синглетными термами атома. Определить интервал L между крайними
		компонентами этой линии в магнитном поле B = 10,0 кГс.

	\textbf{5.9 }
		Построить схему возможных переходов между термами \( ^2P_{3/2} \) и 
		\( ^2S_{1/2} \) в слабом магнитном поле. Вычислить для соответствующей 
		спектральной линии:
		\vspace*{-2em}
		\begin{itemize}\itemsep-8pt
			\item[а)] смещения зеемановских компонент в единицах 
			\( \mu_\text{Б}B/\hbar \)
			\item[б)] интервал, \( \text{см}^{-1} \), между крайними 
			компонентами, если B = 5,00 кГс.
		\end{itemize}
		Используемые формул:
		\[ \Delta\omega = \frac{\mu_\text{Б}B}{\hbar}(m_{J2}g_{2}-m_{J1}g_{1}) \]
		\[ m_J = -J, -J+1, ..., +J \]
		\[ g = 1 + \frac{S(S+1)+J(J+1)-L(L+1)}{2J(J+1)} \]
		Распишем значения двух термов:
		\[ 
			^2P_\frac{3}{2}: L = 1, S = \frac{1}{2}, J = \frac{3}{2};\quad
			m_J = -\frac{3}{2}, -\frac{1}{2}, \frac{1}{2}, \frac{3}{2};\quad
			g = \frac{1}{3}
		\]
		\[ 
			^2S_\frac{1}{2}: L = 0, S = \frac{1}{2}, J = \frac{1}{2};\quad
			m_J = -\frac{1}{2}, \frac{1}{2};\quad
			g = 2
		\]
		\begin{itemize}\itemsep-8pt
			\item[а)]
			\[ 
				\Delta\omega = \frac{\mu_\text{Б}B}{\hbar}
				(\frac{3}{2}\cdot\frac{1}{3} - 2\cdot\frac{1}{2}) = 0 (???)
			\]
			\item[б)]
		\end{itemize}

	\textbf{5.10 }
		Изобразить схему возможных переходов в слабом магнитном поле и
		вычислить смещения (в единицах \( \mu_\text{Б}B/\hbar \) ) 
		зеемановских компонент спектральной линии: 
		\vspace*{-1em}
		\begin{itemize}\itemsep-8pt
			\item[а)] \( ^2D_{3/2} \rightarrow ^2P_{3/2} \)
			\item[б)] \( ^2D_{5/2} \rightarrow ^2P_{3/2} \)
		\end{itemize}
		Используемые формулы: 
		\[ 
			\Delta\omega = \frac{\mu_\text{Б}B}{\hbar}(m_{J2}g_2 - m_{J1}g_1)
			= \Delta\omega_0 (m_{J2}g_2 - m_{J1}g_1)
		\]
		\[ g = 1 + \cfrac{J(J+1)+S(S+1)-L(L+1)}{2J(J+1)} \]
		Рассмотрим пункт решение для пункта а:
		\[ ^2D_{3/2}: L=2, S=\frac{1}{2}, J=\frac{3}{2} \]
		\[ m_{J2} = -\frac{3}{2}, -\frac{1}{2}, \frac{1}{2}, \frac{3}{2} \]
		\[ 
			g_2 = 1 + \frac{\cfrac{3}{2}\cdot\cfrac{5}{2} 
			+ \cfrac{1}{2}\cdot\cfrac{3}{2} - 2\cdot3}{\cfrac{15}{2}} 
			= \frac{24}{30}
		\]
		\[ ^2P_{3/2}: L=1, S=\frac{1}{2}, J=\frac{3}{2} \]
		\[ m_{J1} = -\frac{3}{2}, -\frac{1}{2}, \frac{1}{2}, \frac{3}{2} \]
		\[ g_1 = 1 + \frac{\cfrac{18}{4} - 2}{\cfrac{15}{2}} = \frac{2}{3} \]
		Правило отбора: \( \Delta m_J = 0, \pm1 \).
		\[ -\frac{3}{2} \rightarrow -\frac{3}{2}: 
			\Delta\omega = \Delta\omega_0 (-\frac{3}{2}\cdot\frac{24}{30} 
			+ \frac{3}{2}\cdot\frac{2}{3}) = -\frac{1}{5}\Delta\omega_0 
		\]
		Считаем \( \Delta\omega \) для чисел:
		\[ 
			-\frac{3}{2} \rightarrow -\frac{1}{2};\quad
			-\frac{1}{2} \rightarrow -\frac{1}{2};\quad
			-\frac{1}{2} \rightarrow \frac{1}{2};\quad
		\]
		\[ 
			\frac{1}{2} \rightarrow \frac{1}{2};\quad
			\frac{1}{2} \rightarrow \frac{3}{2};\quad
			\frac{3}{2} \rightarrow \frac{3}{2};\quad
		\]
		И находим разность между двумя \( \Delta\omega \). \\
		Пункт б считается по аналогии.

	\textbf{5.11 }
		Найти значения температуры, при которых средняя кинетическая энергия
		поступательного движения молекул \( H_2 \) и \( N_2 \) равна их 
		вращательной энергии в состоянии с квантовым числом J = 1.
		\par\textbf{Основная идея}: записываем энергию вращательного движения молекулы
		( \( E_J = \frac{\hbar^2}{2I}J(J+1) \) ), где \( I \) -- 
		момент инерции молекулы, \( J \) -- вращательное квантовое число; 
		приравниваем к значению \( \frac{3}{2} kT \), где k -- постоянная Больцмана 
		(\( 1.38\cdot10^{-23} \text{Дж}\cdot\text{К}^{-1} \)), T -- 
		температура; выражаем и находим значение для каждой из молекул.

	\textbf{5.12 }
		Определить момент импульса молекулы кислорода в состоянии с
		вращательной энергией 2,16 мэВ.

	\textbf{5.13 }
		Найти для молекулы HCl вращательные квантовые числа двух соседних
		уровней, разность энергий которых равна 7,86 мэВ.

	\textbf{5.14 }
		Найти механический момент молекулы кислорода, вращательная энергия
		которой E = 2,16 мэВ.

	\textbf{5.15 }
		Для двухатомной молекулы известны интервалы между тремя
		последовательными вращательными уровнями \( \Delta E_1 = \) 0,20 мэВ и
		\(\Delta E_2 = \) 0,30 мэВ. Найти вращательное квантовое число 
		среднего уровня исоответствующий момент инерции молекулы. \\
		Используемая формула: \[ E_r = \frac{\hbar^2}{2I}J(J+1) \]
		Запишем формулы для вращательных уровней:
		\[ E_{J1} = \frac{\hbar^2}{2I}J(J+1) \]
		\[ E_{J2} = \frac{\hbar^2}{2I}(J+1)(J+2) = \frac{\hbar^2}{2I}n(n+1) \]
		\[ E_{J3} = \frac{\hbar^2}{2I}(J+2)(J+3) \]
		Сделав обозначение \( n = J + 1 \). \\
		Распишем разность энергий через формулы для уровней:
		\[ 
			\Delta E_1 = E_{J2} - E_{J1} = 
			\frac{\hbar^2}{2I}((J+1)(J+2) - J(J+1)) = \frac{\hbar^2}{I}(J+1) 
		\]
		\[ 
			\Delta E_1 = E_{J2} - E_{J1} =
			\frac{\hbar^2}{2I}((J+2)(J+3)-(J+1)(J+2)) = \frac{\hbar^2}{I}(J+2)
		\]
		Найдём значение \( J \) поделив \( \Delta E_1 \) на \( \Delta E_2 \):
		\[ \frac{\Delta E_1}{\Delta E_2} = \frac{J+1}{J+2} \]
		Сделав преобразования относительно \( J \) получим:
		\[ J = \frac{\Delta E_2 - 2\Delta E_1}{\Delta E_1 - \Delta E_2} = 1 \]
		Отсюда получаем значение для среднего уровня \( n = J+1 = 2 \). \\
		Для определение момента запишем: 
		\[ 
			\Delta E_2 - \Delta E_1 = \frac{\hbar^2}{I}(J+2-J-1) 
			\Rightarrow I = \frac{\hbar^2}{\Delta E_2 - \Delta E_1}
		\]

	\textbf{5.16 }
		Оценить, сколько линий содержит чисто вращательный спектр молекул
		\( CO \), момент инерции которых равен 
		\( I = 1.44\cdot10^{-39} \text{г}\cdot\text{см}^2 \).

	\textbf{5.17 }
		Найти отношение энергий, которые необходимо затратить для
		возбуждения двухатомной молекулы на первый колебательный и первый
		вращательный уровни. Вычислить это отношение для следующих молекул:
		\vspace*{-1em} 
		\begin{itemize}\itemsep-8pt
			\item[а)] \( H_2 \)
			\item[б)] \( HI \)
			\item[в)] \( I_2 \)
		\end{itemize}
\end{document}