\chapter{Введение}
	История политических учений занимает важное место не только в политологи, 
	но и в системе общественных наук в целом, ибо в политических учениях 
	находят свое выражение социально-экономические интересы различных классов 
	и социальных групп. А поскольку эти интересы нуждаются в защите со 
	стороны политической власти, то эта нужда и побуждает как господствующие 
	классы, так и угнетаемых разрабатывать и обосновывать свои теории.

	Помимо этого, история политических учений -- это колоссальный опыт, без 
	которого не может решаться ни один важный вопрос современной политической 
	жизни. Этот опыт включает в себя множество оригинальных решений, 
	актуальных и для современного периода, предоставляя ученым и практикам 
	уникальную возможность сравнивать свои изыскания с прошлой практикой, с 
	предыдущими оригинальными находками.

	Уже в древних государствах Востока существует своя политическая идеология. 
	Это -- теория божественного происхождения государства и культа верховного 
	правителя.

	Наряду с теорией божественного происхождения власти в учениях Древнего мира 
	зарождаются и основы политической философии, юриспруденции, складываются 
	политико-социалогические подходы к политологическим изысканиям. Это, в 
	первую очередь, касается Древней реции и Рима.

	Особенности ландшафта Греции, ее многочисленные связи с заграницей 
	создавали плодотворные коммуникации с другими государствами, многообразие 
	культурных стилей заключало в себе богатство политической жизни с активным 
	участием граждан в правлении. Именно в Древней Греции сложилось и получило 
	всестороннее развитие такое общественно-политическое образование, как 
	полис, который одновременно был и городом-государством, олицетворяя собой 
	гражданскую жизнь. От него, по существу, произошли все основные 
	политические понятия: политика, политическое искусство, политическое 
	знание, политическое управление и д.р.

	Политические знания в античности существовали в философско-этической форме, 
	поскольку все философы Древней Греции в той или иной мере касались вопросов 
	власти, управления государством. Однако наиболее существенный вклад в 
	развитие политической мысли внесли древнегреческие философы Платон 
	(427 -- 347 до н. э.) -- автор работ <<Государство>> и <<Законы>>, а так же 
	его ученик Аристотель, написавший знаменитый трактат <<Политика>>.

	Проблема государственного устройства была, есть и будет одной из самых сложных 
	и противоречивых проблем, стоящих перед человечеством. Множество людей уже 
	тысячи лет пытаются понять, каким государство должно быть <<в идеале>>. При этом 
	некоторые люди считают лучшим сильное, боеспособное государство с хорошей 
	экономикой, другие - государство, в котором каждый человек ощущает себя вполне 
	свободным и счастливым. Так проблема общественного, государственного устройства 
	перерастает в проблему понимания блага, человеческих ценностей и свободы 
	личности. Этим вопросам и посвящен трактат Платона <<Государство>>.

	Социально-политическим вопросам посвящены несколько произведений Платона: 
	трактат <<Государство>>, диалоги <<Критий>>, <<Политик>>, <<Законы>>. Все они 
	написаны в довольно необычном жанре -- жанре диалога между Сократом и менее 
	известными греческими философами, например, Главконом, Адимантом, Кефалом. 

	В своих произведениях Платон говорит о модели <<идеального>>, лучшего 
	государства. Эта модель не есть описание какого-либо существующего строя, 
	системы, а, напротив, модель такого государства, которого нигде и никогда 
	не было, но которое должно возникнуть, то есть Платон говорит об идее 
	государства, создает проект, либо как принято называть - утопию.

\chapter{Понятие о государстве}
	Существует множество взглядов на понятие о государстве. В <<Словаре 
	античности>> читаем: <<В качестве органа власти экономически 
	господствующего класса, государство возникло в процессе образования 
	частной собственности. Эта ранняя форма эксплуатации первоначально 
	потребовала таких средств власти, которые смогли бы обеспечить порядок 
	и спокойствие угнетателей>>. 

	Однако античные философы полагали иначе. Платон говорил о государстве 
	не как об аппарате подавления, но как о некоем благе. <<Когда люди 
	отведалии того и другого, то есть и поступали несправедливо, и 
	страдали от несправедливости, тогда они <..> нашли целесообразным 
	договориться друг с другом, чтобы не творить несправедливости и не 
	страдать от нее. Отсюда взяло свое начало законодательство и взаимный 
	договор>>. Таким образом государство не карает, но помогает людям. 
	Платон также отводит много места описанию сословия стражей, то есть 
	тех самым сил, с помощью которых должен обеспечиваться <<порядок и 
	спокойствие угнетателей>>. Тем не менее, солдаты нужны Платону не для 
	борьбы со своими же гражданами, но для защиты от внешних врагов. 
	<<Будущий страж нуждается еще вот в чем: мало того, что он яростен -- 
	он должен по своей природе еще и стремиться к мудрости>>. Мудрость дает 
	ему понятие о том, что справедливо, что нет. Воин сравнивается философом 
	состорожевой собакой, которая обязательно должна различать своих и чужих. 
	Платон просто уверен, что в идеальном государстве, воспроизведенным на 
	страницах его трактата, невозможны будут волнения и классовые 
	столкновения, которые заставят воинов выступить против своих же сограждан 
	и сражаться с теми, кого они призваны защищать. Таким образом подлинное 
	государство в понятии Платона настолько идеально, что карательные его 
	функции отходит на второй, если не на третий план, в виду сознательности 
	людей его населяющих.

	Следует отметить, что античные философы все же понимали, что иногда 
	государство может служить для власть имущих средством подавления власть 
	неимущих. Однако они полагали, что это присуще не первозданным моделям 
	государственных устройств,но иногда получается в процессе их развития, 
	когда монархия, аристократия и демократия извращаются и становятся 
	тиранией, олигархией и анархией. Платон, Аристотель и Цицерон, жившие 
	во времена тяжелых кризисов власти, наступавших после периодов расцвета, 
	были склонны рассматривать извращенное устройство не как сущность 
	государства, но как ее искажение. Изначально государство для них являлось 
	справедливым и служило не как машина закрепощения, но как нечто, дающее 
	благо всем гражданам.

\chapter{Идеальное государство}
	На наш взгляд, понятие идеального государственного строя весьма условно, 
	ибо все заключает в себе как положительные, так и отрицательные стороны. 
	Реально не существовало ни одной вещи, которая бы всем понравилась или не 
	понравилась бы никому. Мы не будем развивать данное суждение, но пока 
	просто скажем, что идеальный государственный строй в понятии Платона, 
	Аристотеля и Цицерона - это лишь то политическое устройство, к которому 
	они тяготеют более всего в силу своих субъективных черт.

	В трактате <<Государство>> Платон пишет о том, что главная причина порчи 
	обществ и государств (которые когда-то, во времена <<золотого века>> имели 
	<<совершенный>> строй) заключена в <<господстве корыстных интересов>>, 
	обуславливающих поступки и поведение людей. В соответствии с этим основным 
	недостатком Платон подразделяет все существующие государства на четыре 
	разновидности в порядке увеличения, нарастания <<корыстных интересов>> в 
	их строе. 

	\begin{enumerate}
		\item Тимократия власть честолюбцев, по мнению Платона, еще 
			сохранила черты <<совершенного>> строя. В государстве такого типа 
			правители и воины были свободны от земледельческих и ремесленных 
			работ. Большое внимание уделяется спортивным упражнениям, однако 
			уже заметно стремление к обогащению, и <<при участии жен>> 
			спартанский образ жизни переходит в роскошный, что обуславливает 
			переход к олигархии. 
		\item Олигархия. В олигархическом государстве уже имеется четкое 
			разделение на богатых (правящий класс) и бедных, которые делают 
			возможной совершенно беззаботную жизнь правящего класса. Развитие 
			олигархии, по теории Платона, приводит к ее перерождению в 
			демократию. 
		\item Демократия. Демократический строй еще более усиливает 
			разобщенность бедных и богатых классов общества, возникают 
			восстания, кровопролития, борьба за власть, что может привести к 
			возникновению наихудшей государственной системы тирании. 
		\item Тирания. По мнению Платона, если некое действие делается 
			слишком сильно, то это приводит к противоположному результату. 
			Так и здесь: избыток свободы при демократии приводит к 
			возникновению государства, вообще не имеющего свободы, живущего 
			по прихоти одного человека тирана.
	\end{enumerate}

	Отрицательные формы государственной власти Платон противопоставляет своему 
	видению <<идеального>> общественного устройства. Огромное внимание автор 
	уделяет определению в государстве места правящего класса. По его мнению, 
	правителями <<идеального>> государства должны быть исключительно философы, 
	для того чтобы в государстве властвовали рассудительность, разум. Именно 
	философы обуславливают благосостояние, справедливость государства Платона, 
	ведь им свойственны <<... правдивость, решительное неприятие какой бы то ни 
	было лжи, ненависть к ней и любовь к истине>>. Платон считает, что любое 
	новшество в идеальном государстве неизбежно ухудшит его (нельзя улучшить 
	<<идеальное>>). Очевидно, что именно философы будут охранять <<идеальный>> 
	строй, законы от всяческих нововведений, ведь они обладают <<...всеми 
	качествами правителей и стражей идеального государства>>. Именно поэтому 
	деятельность философов обуславливает существование <<идеального>> 
	государства, его неизменность. По существу, философы охраняют остальных 
	людей от порока, каким является любое нововведение в государстве Платона. 
	Не менее важно и то, что благодаря философам правление и вся жизнь 
	<<идеального>> государства будет построена по законам разума, мудрости, там 
	не будет места порывам души и чувствам. 

	Если в государстве Платона существуют люди, которые занимаются законами и 
	устройством государства, то естественно предположить, что в нем существуют 
	и люди, занимающиеся исключительно земледелием, ремеслом. Действительно, 
	основной закон существования <<идеального>> государства состоит в том, что 
	каждый член общества обязан выполнять только то дело, к которому он 
	пригоден.

	Всех жителей <<идеального>> государства автор разделяет на три класса. 
	Низший класс объединяет людей, которые производят необходимые для 
	государства вещи или способствуют этому; в него входят самые разные люди, 
	связанные с ремеслом, земледелием, рыночными операциями, деньгами, 
	торговлей и перепродажей это земледельцы, ремесленники, торговцы. Не 
	смотря на то, что торговцами и земледельцами могут быть совершенно 
	различные люди, все они, по Платону, стоят приблизительно на одной ступени 
	нравственного развития. Внутри этого низшего класса также существует 
	четкое разделение труда: кузнец не может заняться торговлей, а торговец 
	по собственной прихоти не может стать земледельцем. 

	Принадлежность человека ко второму и третьему классам, а это классы 
	воинов-стражей и правителей-философов, определяется уже не по 
	профессиональным, а по нравственным критериям. Нравственные качества этих 
	людей Платон ставит гораздо выше нравственных качеств первого класса.

	Так Платон создает тоталитарную систему разделения людей на разряды, 
	которая немного смягчается возможностью перехода из класса в класс 
	(это достигается путем длительного воспитания и самосовершенствования). 
	Переход этот осуществляется под руководством правителей. Характерно, что 
	если даже среди правителей появится человек, больше подходящий для низшего 
	класса, то его необходимо <<понизить>> без сожаления. Таким образом, Платон 
	считает, что для благосостояния государства каждый человек должен 
	заниматься тем делом, для которого он приспособлен наилучшим образом. 
	Если человек будет заниматься не своим делом, но внутри своего класса, то 
	это еще не гибельно для <<идеального>> государства. Когда же человек 
	незаслуженно из сапожника становится воином, или же воин незаслуженно 
	становится правителем, то это грозит крахом всему государству, 
	поэтому такой <<перескок>> считается <<высшим преступлением>> против 
	системы, ведь для блага всего государства в целом человек должен делать 
	только то дело, к которому он наилучшим образом приспособлен. 

	Вслед за Сократом, считавшим тремя основными добродетелями умеренность 
	(знание, как обуздывать страсти), храбрость (знание, как преодолеть 
	опасности), справедливость (знание, как соблюдать законы божественные 
	и человеческих), Платон пишет о том, что <<идеальное>> государство должно 
	обладать, по меньшей мере, четырьмя главными добродетелями: мудростью, 
	мужеством, рассудительностью, справедливостью. 

	Мудростью не могут обладать все жители государства, но правители-философы, 
	избранные люди, безусловно, мудры и принимают мудрые решения. Мужеством 
	обладает большее количество людей, это не только правители-философы, но 
	и воины-стражи. Если первые две добродетели были характерны только для 
	определенных классов людей, то рассудительность должна быть присуща всем 
	жителям, она <<подобна некой гармонии>>, она <<настраивает на свой лад решительно 
	все целиком>>. Под четвертой добродетелью справедливостью автор понимает уже 
	рассмотренное деление людей на разряды, касты: <<...заниматься своим делом и не 
	вмешиваться в чужие это есть справедливость>>. Следовательно, разделение людей 
	на классы имеет для Платона огромное значение, определяет существование 
	<<идеального>> государства (ведь оно не может быть несправедливым), и тогда 
	неудивительно, что нарушение кастового строя считается высшим 
	преступлением. Так государство Платона незаметно, ради лучшей цели, 
	приобретает те недостатки, которые рассматривал сам автор, описывая 
	<<порочные>> государства (например, расслоение общества в олигархическом 
	государстве).

	Характерно, что Платон, живший во времена всеобщего рабовладельческого 
	строя, не уделяет рабам особого внимания. В <<Государстве>> все 
	производственные заботы возлагаются на ремесленников и земледельцев. Здесь 
	же Платон пишет, что в рабство можно обращать только <<варваров>>, 
	неэллинов, во время войны. Однако он же говорит, что война зло, 
	возникающее в порочных государствах <<для обогащения>>, и в <<идеальном>> 
	государстве войны следует избегать, следовательно, не будет и рабов. 
	Как писал В. Ф. Асмус, в трактате <<Государство>> <<...класс рабов как один 
	из основных классов образцового государства не предусматривается, не 
	указывается, не называется>>. Это не значит, что автор выступает против 
	угнетения человека человеком, просто, по его мнению, высшие разряды 
	(касты) не должны иметь частной собственности, чтобы сохранить единство. 
	Тем не менее, в диалоге <<Законы>>, где также обсуждаются проблемы 
	государственного устройства, Платон перекладывает основные хозяйственные 
	заботы на рабов и чужестранцев, но осуждает воины. 

	В связи с рассмотренным разделением людей на разряды возникает вопрос: 
	кто же возьмет на себя ответственность определения способности человека к 
	некоему делу, и только к нему? По-видимому, в <<идеальном>> государстве эту 
	функцию возьмут на себя мудрейшие и справедливейшие люди 
	правители-философы. При этом они, естественно, будут выполнять закон, 
	ведь закон важнейшая составляющая <<идеального>> государства, и его выполняют 
	все без исключения (доходит до того, что дети должны играть по законам 
	государства). Таким образом, правители-философы вершат судьбы всех 
	остальных людей. Они не только определяют способности человека, но и 
	осуществляют регламентацию брака, имеют право (и должны) убивать 
	малолетних детей с физическими недостатками (здесь, как и в некоторых 
	других случаях, Платон берет за образец государственное устройство 
	современной ему Спарты). 

	Философы, на основах разума, управляют остальными классами, ограничивая их 
	свободу, а воины играют роль <<собак>>, держащих в повиновении низшее <<стадо>>. 
	Этим усугубляется и без того жестокое разделение на разряды. Например, воины 
	не живут в одних местах с ремесленниками, людьми труда. Люди <<низшей>> породы 
	существуют для обеспечения <<высших>> всем необходимым. <<Высшие>> же охраняют и 
	направляют <<низших>>, уничтожая слабейших и регламентируя жизнь остальных. 

	Можно предположить, что такая всесторонняя мелочная регламентация важнейших 
	поступков человека, которые он, по современным понятиям, должен решать сам, 
	приведет к разобщению людей, недовольству, зависти. Однако в <<идеальном>> 
	государстве этого не происходит, напротив, единство людей Платон считает 
	основой такого государства. Во времена древности, <<золотого века>>, когда 
	сами боги управляли людьми, люди рождались не от людей, как сейчас, но от 
	самой земли. Люди не нуждались в материальных благах и много времени 
	посвящали занятиям философией. Во многом единство древних обуславливалось 
	отсутствием родителей (у всех одна мать земля). Платон хочет достичь того 
	же результата, <<обобществив>> не только людское имущество, но и жен, детей. 
	Автор хочет, чтобы никто не мог сказать: <<Это моя вещь>>, или 
	<<Это моя жена>>. По идее Платона, мужчины и женщины не должны вступать в 
	брак по собственной прихоти. Оказывается, браком тайно управляют философы, 
	совокупляя лучших с лучшими, а худших с худшими. После родов дети 
	отбираются, и отдаются матерям через некоторое время, причем никто 
	не знает, чей ребенок ему достался, и все мужчины (в пределах касты) 
	считаются отцами всех детей, а все женщины общими женами всех мужчин. 
	Как писал В. Асмус, для Платона общность жен и детей является высшей 
	формой единства людей. Такая общность описана им для класса 
	воинов-стражей, которым автор уделяет огромное внимание. По его мнению, 
	отсутствие вражды внутри класса стражей повлечет за собой единство низшего 
	класса и отсутствие восстаний. 

	Таким образом, правящие классы государства Платона составляют 
	коммунистическое единство. Этот коммунизм, как уже говорилось, не 
	допускает среди высших классов бедности или богатства, а, следовательно, 
	по логике автора, уничтожает среди них раздоры. Прообраз власти у Платона 
	это пастух, пасущий стадо. Если прибегнуть к этому сравнению, то в 
	<<идеальном>> государстве пастухи - это правители, воины это сторожевые 
	собаки. Чтобы удержать стадо овец в порядке, пастухи и собаки должны быть 
	едины в своих действиях, чего и добивается автор. Платоновское государство 
	обращается с <<человеческим стадом>>, как мудрый, но жестокосердный пастух 
	со своими овцами. Этого скрестить с тем-то, этих на бойню. Видно, что, по 
	нашим представлениям, это тоталитарная программа, при которой кучка людей 
	(пусть даже мудрейших) подчиняет <<...жалкие вожделения большинства...разумным 
	желаниям меньшинства>>. 

	Платон пытался провести свои идеи в жизнь. Находясь на Сицилии, он 
	старался привить тамошним тиранам любовь к философии, но и Дионисий 
	Старший, и его сын Дионсий Младший, являясь при этом людьми 
	образованными, умными и настроенными по отношению к Платону 
	дружелюбно, все же четко разделяли философию и политику. О том, 
	что мешало им <<избавиться от зол>>. Сейчас же остается подытожить, 
	что идеальный строй для Платона заключается в царской власти. Демократия у 
	него стоит на предпоследнем месте от худшего государственного 
	строя -- тирании. Многим покажется это парадоксальным, ибо именно к 
	демократии с ее принципами равенства и свободы стремятся и стремились 
	большинство людей и государств. Однако уместнее ответит на это словами 
	других философов: Цицерона и Аристотеля. <<Простой народ, являясь монархом, 
	стремится и управлять по-монаршьему и становится деспотом. <..> 
	И крайняя демократия, и тирания поступают деспотически с лучшими 
	гражданами>>; <<если народ применил насилие к справедливому царю или 
	лишил его власти, или даже отведал крови оптиматов и подчинил своему 
	произволу все государство, не думай, Лелий, что найдется море или пламя, 
	успокоить которое, при всей его мощности труднее, чем усмирить толпу. <..> 
	Таким образом величайшая свобода порождает тиранию или несправедливейшее 
	и тяжелейшее рабство>>. Иными словами, одна крайность дает другую. 
	Правление большинства приводит к тому, что у власти стоят далеко не 
	философы, что все становится вседозволенным, начинается произвол и 
	анархия, и уже на этой почве начинает прогрессировать тирания. 
	При этом тирания может выражаться в совершенно различных видах: 
	диктатура пролетариата, диктатура конкретных людей, вознесенных на 
	гребне анархической волны и т.д.

\chapter{Заключение}
	Взгляды Платона на происхождение общества и государства можно 
	охарактеризовать как систему идеальных образований, а именно теорию 
	идей и идеала. <<Идеальное государство>> является сообществом земледельцев, 
	ремесленников, производящих все необходимое для жизни горожан, воинов и 
	правителей, которые являются философами. Такое <<идеальное государство>> 
	Платон противопоставил античной демократии, допускавшей народ к участию 
	в политической жизни, к государственному управлению. По мнению Платона, 
	государством признаны управлять только аристократы, ибо они являются 
	мудрейшими и лучшими из всех сословий.

	Что касается ремесленников и земледельцев, то эти граждане должны 
	заниматься только своим профессиональным делом и не касаться 
	государственных дел и вопросов. Воины в свою очередь – охранять государство 
	и его территорию, не имея при этом частной собственности, и обязаны жить 
	отдельно от других сословий. 

	<<Идеальное государство>>, по Платону, должно всемерно покровительствовать 
	религии, воспитывать в гражданах благочестие, бороться против всякого рода 
	нечестивых. Эти же цели должна преследовать и вся система воспитания и 
	образования.

	Но в этой утопии есть и положительная черта. С редким реализмом Платон 
	понял характерную для античного полиса связь единичного с целым, 
	зависимость личности от более широкого целого, обусловленность индивида 
	государством. Поняв эту связь, Платон превратил ее в норму задуманного 
	им проекта идеального общественно-политического устройства. Более того, 
	создавая диалог <<Государство>>, Платон, в первую очередь, пытался понять, 
	какова идея государства в нашем мире, с какого образца создавались 
	существующие государства. Проблема познания идеи государства, безусловно, 
	шире, чем проблема построения <<лучшего>> государства, поняв идею 
	государства, мы одновременно поймем, к чему надо стремиться. 
	Следовательно, вопрос о том, насколько хорошо государство Платона, 
	вторичен, важнее то, что это не существующее государство, но его <<идея>>.

	В заключении следует отметить, что учение Платона оказало громадное 
	влияние на последующее развитие политико-правовой идеологии. Под его 
	воздействием складывались философские и социально-политические взгляды 
	Аристотеля, стоиков, Цицерона и других представителей античной политической 
	мысли. Выдвинутые Платоном идеи <<правления философов>> и <<мудрых законов>> 
	были восприняты многими мыслителями эпохи Просвещения.

\pagebreak

\chapter{Список использованной литературы}
\begin{enumerate}
	\item Дмитрий Крюков <<Образ идеального государства у Платона, Аристотеля и Цицерона.>>
	\item Философский энциклопедический словарь. М., 1983. С. 710.
	\item Журнал Философия и общество. Выпуск №1(38)/2005 Автор: Голубев С.В.
	\item Спиркин А.Г. Философия: Учебник. – 2-е изд. – М.: Гардарики, 2002. – 736 с.
\end{enumerate}
