\chapter{Введение}
  История политических учений занимает важное место не только в политологи, 
	но и в системе общественных наук в целом, ибо в политических учениях 
	находят свое выражение социально-экономические интересы различных классов 
	и социальных групп. А поскольку эти интересы нуждаются в защите со 
	стороны политической власти, то эта нужда и побуждает как господствующие 
	классы, так и угнетаемых разрабатывать и обосновывать свои теории.

	Помимо этого, история политических учений -- это колоссальный опыт, без 
	которого не может решаться ни один важный вопрос современной политической 
	жизни. Этот опыт включает в себя множество оригинальных решений, 
	актуальных и для современного периода, предоставляя ученым и практикам 
	уникальную возможность сравнивать свои изыскания с прошлой практикой, с 
	предыдущими оригинальными находками.

	Уже в древних государствах Востока существует своя политическая идеология. 
	Это -- теория божественного происхождения государства и культа верховного 
	правителя.

	Наряду с теорией божественного происхождения власти в учениях Древнего мира 
	зарождаются и основы политической философии, юриспруденции, складываются 
	политико-социалогические подходы к политологическим изысканиям. Это, в 
	первую очередь, касается Древней реции и Рима.

	Особенности ландшафта Греции, ее многочисленные связи с заграницей 
	создавали плодотворные коммуникации с другими государствами, многообразие 
	культурных стилей заключало в себе богатство политической жизни с активным 
	участием граждан в правлении. Именно в Древней Греции сложилось и получило 
	всестороннее развитие такое общественно-политическое образование, как 
	полис, который одновременно был и городом-государством, олицетворяя собой 
	гражданскую жизнь. От него, по существу, произошли все основные 
	политические понятия: политика, политическое искусство, политическое 
	знание, политическое управление и д.р.

	Развитие политической мысли Древней Греции характеризуется рядом черт:
	\begin{enumerate}
		\item поиском идеальной модели государства, способной обеспечить 
			справедливость и порядок;
		\item рассмотрением политики как единственной формы цивилизованного 
			бытия человека, предполагавшей нерасчлененность государства, 
			общества и отдельного индивида;
		\item отсутствием четкой границей между философией, этикой и 
			политикой, что определило морализаторский, поучительный характер 
			работ по политической проблематике;
		\item ограниченным рационализмом политической мысли, обусловленным 
			заметным влиянием религии. 
	\end{enumerate}

	Политические знания в античности существовали в философско-этической форме, 
	поскольку все философы Древней Греции в той или иной мере касались вопросов 
	власти, управления государством. Однако наиболее существенный вклад в 
	развитие политической мысли внесли древнегреческие философы Платон 
	(427 --347 до н. э.) -- автор работ <<Государство>> и <<Законы>>, а так же 
	его ученик Аристотель, написавший знаменитый трактат <<Политика>>.

	Проблема государственного устройства была, есть и будет одной из самых сложных 
	и противоречивых проблем, стоящих перед человечеством. Множество людей уже 
	тысячи лет пытаются понять, каким государство должно быть <<в идеале>>. При этом 
	некоторые люди считают лучшим сильное, боеспособное государство с хорошей 
	экономикой, другие - государство, в котором каждый человек ощущает себя вполне 
	свободным и счастливым. Так проблема общественного, государственного устройства 
	перерастает в проблему понимания блага, человеческих ценностей и свободы 
	личности. Этим вопросам и посвящен трактат Платона <<Государство>>.

	Социально-политическим вопросам посвящены несколько произведений Платона: 
	трактат <<Государство>>, диалоги <<Критий>>, <<Политик>>, <<Законы>>. Все они 
	написаны в довольно необычном жанре - жанре диалога между Сократом и менее 
	известными греческими философами, например, Главконом, Адимантом, Кефалом. 

	В своих произведениях Платон говорит о модели <<идеального>>, лучшего 
	государства. Эта модель не есть описание какого-либо существующего строя, 
	системы, а, напротив, модель такого государства, которого нигде и никогда 
	не было, но которое должно возникнуть, то есть Платон говорит об идее 
	государства, создает проект, либо как принято называть - утопию.

\chapter{Понятие о государстве}
	Существует множество взглядов на понятие о государстве. В <<Словаре 
	античности>> читаем: <<В качестве органа власти экономически 
	господствующего класса, государство возникло в процессе образования 
	частной собственности. Эта ранняя форма эксплуатации первоначально 
	потребовала таких средств власти, которые смогли бы обеспечить порядок 
	и спокойствие угнетателей>>. 

	Однако античные философы полагали иначе. Платон говорил о государстве 
	не как об аппарате подавления, но как о некоем благе. <<Когда люди 
	отведалии того и другого, то есть и поступали несправедливо, и 
	страдали от несправедливости, тогда они <..> нашли целесообразным 
	договориться друг с другом, чтобы не творить несправедливости и не 
	страдать от нее. Отсюда взяло свое начало законодательство и взаимный 
	договор>>. Таким образом государство не карает, но помогает людям. 
	Платон также отводит много места описанию сословия стражей, то есть 
	тех самым сил, с помощью которых должен обеспечиваться <<порядок и 
	спокойствие угнетателей>>. Тем не менее, солдаты нужны Платону не для 
	борьбы со своими же гражданами, но для защиты от внешних врагов. 
	<<Будущий страж нуждается еще вот в чем: мало того, что он яростен - 
	он должен по своей природе еще и стремиться к мудрости>>. Мудрость дает 
	ему понятие о том, что справедливо, что нет. Воин сравнивается философом 
	состорожевой собакой, которая обязательно должна различать своих и чужих. 
	Платон просто уверен, что в идеальном государстве, воспроизведенным на 
	страницах его трактата, невозможны будут волнения и классовые 
	столкновения, которые заставят воинов выступить против своих же сограждан 
	и сражаться с теми, кого они призваны защищать. Таким образом подлинное 
	государство в понятии Платона настолько идеально, что карательные его 
	функции отходит на второй, если не на третий план, в виду сознательности 
	людей его населяющих.

	Аристотель также представлял государство как нечто прекрасное по своей 
	сути. <<Целью государства является благая жизнь>>. Он исходил из того 
	понятия, что человек <<существо политическое>>, стремящееся к общению, а 
	потому государство для него необходимо как воздух. <<Всякое государство 
	представляет собой своего рода общение, всякое же общение организуется 
	ради какого-либо блага. Больше других и к высшему из всех благ стремится 
	то общение, которое является наиболее важным из всех и обнимает собой все 
	остальные общения. Это общение и называется государством или общением 
	политическим>>. Если разобрать данное определение, то мы несколько раз 
	встретим слово <<благо>>, но вообще не найдем упоминания о каком-либо 
	подавлении и угнетении. 

	Цицерон также предерживался концепции Аристотеля. Это явствует из его 
	определения: <<Государство есть достояние народа, а народ не любое 
	соединение людей, собранных вместе каким бы то ни было образом, а 
	соединение многих людей, связанных между собой согласием в области права 
	и общностью интересов. Первой причиной для такого соединения людей 
	является не столько их слабость, сколько, так сказать, врожденная 
	потребность жить вместе>>. Если мы говорим об общности интересов и 
	согласии, то не о каких мятежах не может быть и речи.

	Следует отметить, что античные философы все же понимали, что иногда 
	государство может служить для власть имущих средством подавления власть 
	неимущих. Однако они полагали, что это присуще не первозданным моделям 
	государственных устройств,но иногда получается в процессе их развития, 
	когда монархия, аристократия и демократия извращаются и становятся 
	тиранией, олигархией и анархией. Платон, Аристотель и Цицерон, жившие 
	во времена тяжелых кризисов власти, наступавших после периодов расцвета, 
	были склонны рассматривать извращенное устройство не как сущность 
	государства, но как ее искажение. Изначально государство для них являлось 
	справедливым и служило не как машина закрепощения, но как нечто, дающее 
	благо всем гражданам.

\chapter{Идеальное государство}

	На наш взгляд, понятие идеального государственного строя весьма условно, 
	ибо все заключает в себе как положительные, так и отрицательные стороны. 
	Реально не существовало ни одной вещи, которая бы всем понравилась или не 
	понравилась бы никому. Мы не будем развивать данное суждение, но пока 
	просто скажем, что идеальный государственный строй в понятии Платона, 
	Аристотеля и Цицерона - это лишь то политическое устройство, к которому 
	они тяготеют более всего в силу своих субъективных черт.

	Исходя из выводов предыдущей главы, мы можем заявить, что идеальный строй 
	в понятии Платона явно не демократический, ибо философ выступает за 
	строгое прикрепление человека к одному роду занятий, а следовательно 
	отвергает возможность демократических выборов, когда человек из народа 
	может неожиданно стать стратегом, консулом и т.д. Для идеального 
	государства Платона характерна скорее царская власть, при которой 
	царь - <<совершенный страж>> и философ. Будучи философом, правитель знает 
	мудрость - <<такое знание, что с его помощью можно решать не мелкие, а 
	общегосударственные вопросы, наилучшим образом руководя внутренними и 
	внешними отношениями>>. Платон также заявляет: <<Пока в государствах не 
	будут царствовать философы, либо нынешние цари и владыки не станут 
	благородно и основательно философствовать и это не сольется воедино - 
	государственная власть и философия, <..> до тех пор государствам не 
	избавиться от зол>>. Платон пытался провести свои идеи в жизнь. 
	Находясь на Сицилии, он старался привить тамошним тиранам любовь к 
	философии, но и Дионисий Старший, и его сын Дионсий Младший, являясь 
	при этом людьми образованными, умными и настроенными по отношению к 
	Платону дружелюбно, все же четко разделяли философию и политику. О том, 
	что мешало им <<избавиться от зол>>, мы поговорим в главе <<Справедливость 
	и благоразумие>>. Сейчас же остается подытожить, что идеальный строй для 
	Платона заключается в царской власти. Демократия у него стоит на 
	предпоследнем месте от худшего государственного строя - тирании. Многим 
	покажется это парадоксальным, ибо именно к демократии с ее принципами 
	равенства и свободы стремятся и стремились большинство людей и государств. 
	Однако мы ответим на это уместное недоумение словами других философов: 
	Цицерона и Аристотеля. <<Простой народ, являясь монархом, стремится и 
	управлять по-монаршьему и становится деспотом. <..> И крайняя демократия, 
	и тирания поступают деспотически с лучшими гражданами>>; <<если народ 
	применил насилие к справедливому царю или лишил его власти, или даже 
	отведал крови оптиматов и подчинил своему произволу все государство, 
	не думай, Лелий, что найдется море или пламя, успокоить которое, при 
	всей его мощности труднее, чем усмирить толпу. <..> Таким образом 
	величайшая свобода порождает тиранию или несправедливейшее и тяжелейшее 
	рабство>>. Иными словами, одна крайность дает другую. Правление большинства 
	приводит к тому, что у власти стоят далеко не философы, что все становится 
	вседозволенным, начинается произвол и анархия, и уже на этой почве 
	начинает прогрессировать тирания. При этом тирания может выражаться в 
	совершенно различных видах: диктатура пролетариата, диктатура конкретных 
	людей, вознесенных на гребне анархической волны и т.д.

	От Платона мы позволим себе перейти сразу к Цицерону, оставив пока 
	Аристотеля в стороне. Прежде чем высказывать свое мнение по вопросу 
	идеального устройства, Цицерон описывает три основных строя, которые 
	мы встречаем еще у Платона и Аристотеля: царскую власть, аристократию, 
	политию (демократию). Однако, в отличие от Аристотеля, полагавшего, что 
	данные государственные устройства по сути совершенны, но иногда могут 
	иметь уродливые формы в виде тирании, олигархии и анархии - в отличие от 
	Аристотеля, Цицерон вообще был не склонен считать царскую власть, 
	аристократию и демократию совершенными. В каждой из них он видел 
	какую-нибудь червоточину, поэтому его проект о создании идеального 
	государства сводится к совмещению всех трех устройств. При этом необходимо 
	было отбросить все неготивные черты каждого из них, но остаить лишь 
	положительные. На основе подобной интеграции монархии, аристократии 
	и демократии должен был появиться четвертый, наиболее совершенный 
	государственный строй. Вопрос об идеальности данного строя мы считаем 
	спорным, но следует отметить, что мысль Цицерона верна с той точки зрения, 
	что монархия, аристократия и демократия никогда не существуют в жизни в 
	своем чистом виде. Для доказательства мы приведем два примера. 
	Царь - просто человек. Он не в состоянии править один, сколь бы силен ни был. 
	Ему нужны помощники, он обязан опираться на определенные слои населения: 
	войско, богачей и т.д. А это приводит к тому, что при царской власти 
	всегда существует аристократия. Вторым примером может послужить то, что 
	при демократии народ все равно выбирает достойнейших из своей среды для 
	того, чтобы те в течение некоторого срока управляли государством. Несмотря 
	на то, что мы говорим о власти народа, на деле власть сосредоточена лишь в 
	руках малой группы людей. Таким образом мы приходим к выводу, что 
	единственно возможным на практике государственным строем является 
	аристократия (олигархия) с уклоном либо в сторону монархии, либо в сторону 
	демократии, либо вообще без всякого уклона. Тогда это аристократия в 
	чистом виде.

	Касательно непосредственно Цицерона, то в работе <<О государстве>> он пишет 
	о приоритете аристократии с уклоном к царской, но не к демократической власти. 
	Здесь мы наталкиваемся на противоречие. Ведь известно, что Цицерон боролся с
	Катилиной и Цезарем, видя в них будущих императоров. Цицерон-оратор и 
	Цицерон-политик неоднократно выступал за республиканские, демократические 
	принципы. И вдруг он пишет о том, что царская власть предпочтительнее. 
	Поразмыслив, мы нашли два способа объяснить данное несоответсвие. Во-первых, 
	Цицерон с самого начала писал, что в своей работе <<О государстве>> прежде 
	всего излагает не собственные взгляды, но воззрения известных римских мужей. 
	В частности речь о пользе монархии принадлежит Сципиону Африканскому. 
	Во-вторых, Цицерон мог подразумевать под царской властью не именно монархию, 
	но республиканский строй, во главе которого стоит консул, цензор или иное 
	должностное лицо, по влиятельности своей схожее с царем.

	Заканчивая этот длинный ряд измышлений, мы приведем цитату из самого 
	философа, ибо никто не охарактеризует идеальный государственный строй в 
	понимании Цицерона, лучше него самого. <<Из трех указанных вначале видов 
	государственного устройства (монархии, аристократии, демократии - К.Д.), 
	по моему мнению, самым лучшим является царская власть, но самое царскую 
	власть превзойдет такая, которая будет образована путем равномерного 
	смешения трех наилучших видов государственного устройства>>. Если 
	относительно первой части этого суждения взгляды Платона и Цицерона 
	совпадают полностью, то во второй римский мыслитель уходит дальше. 
	Платон не говорил о том, что смешение трех властей есть нечто лучшее, 
	нежели государственный строй во главе с царем-философом.

	Аристотель в данном вопросе являет собой образ человека, который говорит 
	одно, а делает иное. Это проистекает из его определения понятия гражданин. 
	Аристотель единственный из трех философов утверждал, что царская власть хуже
	демократии: <<То положение, чтобы верховная власть находилась в руках 
	большинства, нежели меньшинства, хотя бы состоящего из наилучших, может 
	считаться, по- видимому, удовлетворительным решение вопроса>>. Из приведенных 
	выше доводов Платона, Цицерона и самого Аристотеля против демократии, 
	которая может довести до всеобщего разгула, анархии и даже тирании, мы 
	можем сделать вывод, что данное мнение крайне оспоримо. Кроме того, из 
	наших умозаключений относительно того, что на практике существует лишь 
	власть группы людей, только с уклоном либо в сторону монархии, либо 
	демократии, получается, что власть большинства - утопия. Однако, в отличие 
	от Платона, представившего настоящий утопический проект государственного 
	устройства, Аристотель не был склонен к чрезмерной идеализации. Он, 
	действительно, говорил о такой недостижимой вещи, как власть большинства, 
	но что такое для него это большинство? Так ли оно велико? Ясно, что под 
	большинством понимается большинство граждан, а не жителей государства вообще. 
	Опираясь же на полученные нами в предыдущей главе выводы, мы заявляем, что 
	в своем идеальном государстве Аристотель подразумевает под гражданами 
	только воинов, чиновников и, возможно, деятелей искусства, стоящих выше 
	обычных ремесленников. Выходит, что от общего количества населения в 
	государстве Аристотеля гражданами являются 10-12% жителей. Конечно, в таком 
	случае нам представляется вполне реальным добиться власти для данного 
	<<большинства>>. Тем более в любом обществе понятие воин и чиновник сами по 
	себе означают, что данный человек состоит на службе у государства и имеет 
	хоть минимальную, но власть над теми, кто к государству относится лишь как 
	рядовой гражданин. Получается, что изречение Аристотеля о власти большинства, 
	которое изначально может показаться вполне демократичным, в контексте всего 
	трактата является по сути своей опять-таки высказываением в пользу если не 
	царской, то, по крайней мере, аристократической власти.

	В плане замечания мы можем также отметить, что Аристотель, также как и 
	Цицерон, говорил о необходимости смешивать все государственные устройства: 
	<<Правильнее утверждение тех, кто смешивает несколько видов, потому что 
	тот государственный строй, который состоит из многих видов, действительно 
	является наилучшим>>. О суждении подобного рода мы уже упоминали, когда 
	рассматривали взгляды Цицерона, а потому здесь не будем долго 
	останавливаться на нем.

	Теперь мы в состоянии подвести итог относительно данного вопроса. Мы имеем
	то, что ни один из трех рассматриваемых нами философов не считал демократию
	идеальным строем. Двое из них высказываются за приоритет мудрой царской 
	власти, а третий - за аристократию. Мы также доказали, что единственным 
	подлинно возможным в жизни строем является именно аристократия. Всех 
	философов объединяет то, что они даруют власть наиболее достойному 
	меньшенству, исходя из принципа, выраженного Цицероном: <<Если свободный 
	народ выберет людей, чтобы вверить им себя, - а выберет он, если только 
	заботится о своем благе, только наилучших людей, - то благо государства 
	несомненно>>. Эти мудрейшие выборные мужи должны будут править столь 
	справедливо, что нигде не будет никаких волнений и недовольств. На этом 
	основаны взгляды всех философов на идеальный государственный строй. Тогда 
	мы поставим вопрос: чем отличаются обычные, существующие повсеместно 
	устройства от совершенных? Мы же, исходя из всего вышесказанного, и 
	ответим, что по своей структуре - ничем. Единственное различие состоит в 
	том, что в несовершенном государстве правят люди ничем не выделяющиеся из 
	среды им подвластных, а иногда даже и худшие; а в идеальном государстве у 
	власти стоят наилучшие. Получается, что самое главное в государстве - это 
	личность правителя. Тогда у нас возникает еще один вопрос. Как связаны 
	между собой властители и законы? Мы считаем, равно как и философы 
	античности, что отбросить такую важную категорию, как закон, остановив 
	свое внимание лишь на правителе, было бы опрометчиво и даже ошибочно.

\chapter{Заключение}
	Взгляды Платона на происхождение общества и государства можно 
	охарактеризовать как систему идеальных образований, а именно теорию 
	идей и идеала. <<Идеальное государство>> является сообществом земледельцев, 
	ремесленников, производящих все необходимое для жизни горожан, воинов и 
	правителей, которые являются философами. Такое <<идеальное государство>> 
	Платон противопоставил античной демократии, допускавшей народ к участию 
	в политической жизни, к государственному управлению. По мнению Платона, 
	государством признаны управлять только аристократы, ибо они являются 
	мудрейшими и лучшими из всех сословий.

	Что касается ремесленников и земледельцев, то эти граждане должны 
	заниматься только своим профессиональным делом и не касаться 
	государственных дел и вопросов. Воины в свою очередь – охранять государство 
	и его территорию, не имея при этом частной собственности, и обязаны жить 
	отдельно от других сословий. 

	<<Идеальное государство>>, по Платону, должно всемерно покровительствовать 
	религии, воспитывать в гражданах благочестие, бороться против всякого рода 
	нечестивых. Эти же цели должна преследовать и вся система воспитания и 
	образования.

	Но в этой утопии есть и положительная черта. С редким реализмом Платон 
	понял характерную для античного полиса связь единичного с целым, 
	зависимость личности от более широкого целого, обусловленность индивида 
	государством. Поняв эту связь, Платон превратил ее в норму задуманного 
	им проекта идеального общественно-политического устройства. Более того, 
	создавая диалог <<Государство>>, Платон, в первую очередь, пытался понять, 
	какова идея государства в нашем мире, с какого образца создавались 
	существующие государства. Проблема познания идеи государства, безусловно, 
	шире, чем проблема построения <<лучшего>> государства, поняв идею 
	государства, мы одновременно поймем, к чему надо стремиться. 
	Следовательно, вопрос о том, насколько хорошо государство Платона, 
	вторичен, важнее то, что это не существующее государство, но его <<идея>>.

	В заключении следует отметить, что учение Платона оказало громадное 
	влияние на последующее развитие политико-правовой идеологии. Под его 
	воздействием складывались философские и социально-политические взгляды 
	Аристотеля, стоиков, Цицерона и других представителей античной политической 
	мысли. Выдвинутые Платоном идеи <<правления философов>> и <<мудрых законов>> 
	были восприняты многими мыслителями эпохи Просвещения.


\chapter{Список использованной литературы}
\begin{enumerate}
	\item Дмитрий Крюков <<Образ идеального государства у Платона, Аристотеля и Цицерона.>>
	\item Философский энциклопедический словарь. М., 1983. С. 710.
	\item Журнал Философия и общество. Выпуск №1(38)/2005 Автор: Голубев С.В.
	\item Спиркин А.Г. Философия: Учебник. – 2-е изд. – М.: Гардарики, 2002. – 736 с.
\end{enumerate}
