\documentclass[pscyr, 12pt]{hedlab}
\usepackage[russian]{babel}
\usepackage{graphicx}
\usepackage{listings}

\labname{Разработка баз данных в СУБД Oracle}
\labnum{2}
\student{Голубев~А.~В., САПР-1.1п}
\labdate{}

\makeatletter
\newcommand{\lstuppercase}{\uppercase\expandafter{\expandafter\lst@token
                           \expandafter{\the\lst@token}}}
\newcommand{\lstlowercase}{\lowercase\expandafter{\expandafter\lst@token
                           \expandafter{\the\lst@token}}}
\makeatother

\lstdefinestyle{Oracle}{%
    basicstyle=\footnotesize\ttfamily,
    keywordstyle=\lstuppercase,
    emphstyle=\itshape,
    showstringspaces=false,
}

\lstdefinelanguage[Oracle]{SQL}[]{SQL}{
    morekeywords={ACCESS, MOD, NLS_DATE_FORMAT, NVL, REPLACE, SYSDATE, TO_CHAR, TO_NUMBER, TRUNC},
}
\lstset{language=[Oracle]SQL, style=Oracle,}

\begin{document}
    \makeheader
    \noindent\textbf{Цель:} получение навыков создания базы данных и работы с ней в диалоговом режиме.

    \noindent\textbf{Постановка задачи:}
    \vspace*{-1em}
    \begin{itemize}\itemsep-5pt
        \item создать таблицу с исплозьванием оператора \emph{CREATE TABLE}
        \item заполнить таблицу произвольными данными
        \item произвести индексацию таблицы с использованием оператора \emph{CREATE INDEX}
        \item распечать содержимое таблицы
    \end{itemize}
  
    \noindent\textbf{Оператор CREATE TABLE:}
    \begin{lstlisting}
CREATE TABLE documents (
    id number(5) primary key,
    d_body varchar2(254),
    d_date date default (sysdate),
    d_lang varchar(254),
    d_code varchar(254)
)
    \end{lstlisting}

    \noindent\textbf{Операторор CREATE INDEX:}
    \begin{lstlisting}
CREATE INDEX i_fast ON documents(d_body, d_lang, d_code);
    \end{lstlisting}

    \noindent\textbf{Код выводящий содержимое таблицы:}
    \begin{lstlisting}
declare
    cursor doc_cur is        
        select * from documents;
begin
    for cur in doc_cur loop
        dbms_output.put_line( cur.id || ' : ' || cur.d_body || ' : ' || cur.d_date || 
                              ' : ' || cur.d_lang || ' : ' || cur.d_code );
    end loop;
end;
    \end{lstlisting}

    \newpage

    \noindent\textbf{Распечатка содержимого таблицы:}\\
    \begin{minipage}{0.47\textwidth}
        \begin{lstlisting}
1 : test : 28-APR-15 : eng : ff01
2 : physics : 28-APR-15 : eng : 14084265
3 : physics : 28-APR-15 : fra : 9393713
4 : exam : 28-APR-15 : eng : 12501962
5 : math : 28-APR-15 : ukr : 1257529
6 : apple : 28-APR-15 : eng : 12044507
7 : physics : 28-APR-15 : rus : 3091905
8 : apple : 28-APR-15 : jap : 16124295
9 : physics : 28-APR-15 : fra : 12625939
10 : apple : 28-APR-15 : jap : 6064049
11 : math : 28-APR-15 : fra : 8539662
12 : physics : 28-APR-15 : bel : 14271115
13 : physics : 28-APR-15 : ita : 4136508
14 : physics : 28-APR-15 : fra : 2604624
15 : apple : 28-APR-15 : kor : 219395
        \end{lstlisting}
    \end{minipage}
    \hspace*{1em}
    \begin{minipage}{0.47\textwidth}
            \begin{lstlisting}
16 : apple : 28-APR-15 : fra : 3588368
17 : orange : 28-APR-15 : bel : 11003793
18 : apple : 28-APR-15 : bel : 5664197
19 : physics : 28-APR-15 : ukr : 7619159
20 : exam : 28-APR-15 : fra : 6383862
21 : orange : 28-APR-15 : ukr : 10033407
22 : physics : 28-APR-15 : kor : 8309077
23 : exam : 28-APR-15 : ukr : 11482178
24 : orange : 28-APR-15 : eng : 1890677
25 : exam : 28-APR-15 : jap : 15609198
26 : orange : 28-APR-15 : fra : 9086164
27 : apple : 28-APR-15 : ita : 4378353
28 : orange : 28-APR-15 : fra : 9420126
29 : math : 28-APR-15 : ukr : 2688763
30 : apple : 28-APR-15 : fra : 10371860
        \end{lstlisting}
    \end{minipage}

    \noindent\textbf{Вывод:} в результате проделанной работы
    \vspace*{-1em}
    \begin{enumerate}\itemsep-5pt
        \item получил навыки по созданию и работе с базой данных
        \item узнал структуру, которая используется для формирования таблиц
        \item основные команды для работы с таблицами
    \end{enumerate}
\end{document}