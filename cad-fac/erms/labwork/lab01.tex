\documentclass[pscyr, 12pt]{hedlab}
\usepackage[russian]{babel}
\usepackage{graphicx}
\usepackage{listings}

\labname{Разработка баз данных в СУБД Oracle}
\labnum{1}
\student{Голубев~А.~В., САПР-1.1п}
\labdate{}

\makeatletter
\newcommand{\lstuppercase}{\uppercase\expandafter{\expandafter\lst@token
                           \expandafter{\the\lst@token}}}
\newcommand{\lstlowercase}{\lowercase\expandafter{\expandafter\lst@token
                           \expandafter{\the\lst@token}}}
\makeatother

\lstdefinestyle{Oracle}{%
    basicstyle=\footnotesize\ttfamily,
    keywordstyle=\lstuppercase,
    emphstyle=\itshape,
    showstringspaces=false,
}

\lstdefinelanguage[Oracle]{SQL}[]{SQL}{
    morekeywords={ACCESS, MOD, NLS_DATE_FORMAT, NVL, REPLACE, SYSDATE, TO_CHAR, TO_NUMBER, TRUNC},
}
\lstset{language=[Oracle]SQL, style=Oracle,}

\begin{document}
    \makeheader
    \textbf{Цель:} ознакомиться с основными возможностями и средствами распределенной СУБД Oracle.

    \textbf{Постановка задачи:} написать программу, которая производит простые математические 
    операции.
  
    \textbf{Исходный код программы PL/SQL:}
    \begin{lstlisting}
SET SERVEROUTPUT ON;
SET ECHO ON;
DECLARE
    TextBlock1 CONSTANT VARCHAR2(40) := '10 + Ln(3.14) * Exp(2.71) * Cos(45) = ';
    TextBlock2 CONSTANT VARCHAR2(40) := '3.14 * exp( 2.71 ) * Sin(42) = ';
    f_pi NUMBER := 3.14;
    f_e NUMBER := 2.71;
    res NUMBER := 0;
    angle NUMBER := 0;
BEGIN
    DBMS_OUTPUT.enable;
    res := 10 + ln( f_pi ) * exp( f_e ) * cos( 45 );
    DBMS_OUTPUT.put_line( TextBlock1 || res );
    res := f_pi * exp( f_e ) * sin( 42 );
    DBMS_OUTPUT.put_line( TextBlock2 || res );
    WHILE ( angle <= 90 ) LOOP
        res := sin( angle ) * sin( angle ) + cos( angle ) * cos( angle );
        DBMS_OUTPUT.put_line( 'sin^2 ( ' || angle || ' ) + cos^2( ' || 
                              angle || ' ) = ' || res );
        angle := angle + 15;
    END LOOP;
    FOR number IN 1 .. 16 LOOP
        IF ( MOD( number, 2 ) = 0 ) THEN
            DBMS_OUTPUT.put_line( number || ' % 2 == 0' );
        ELSE
            DBMS_OUTPUT.put_line( number || ' % 2 != 0'  );
        END IF;
    END LOOP;
END;
    \end{lstlisting}

    \newpage

    \textbf{Вывод программы:}
    \begin{lstlisting}
10 + Ln(3.14) * Exp(2.71) * Cos(45) = 19.03387803804822974923603871245099709864
3.14 * exp( 2.71 ) * Sin(42) = -43.252416257084592172957599720632519481
sin^2 ( 0 ) + cos^2( 0 ) = 1
sin^2 ( 15 ) + cos^2( 15 ) = 1.00000000000000000000000000000000000002
sin^2 ( 30 ) + cos^2( 30 ) = 1.00000000000000000000000000000000000002
sin^2 ( 45 ) + cos^2( 45 ) = .9999999999999999999999999999999999999944
sin^2 ( 60 ) + cos^2( 60 ) = .9999999999999999999999999999999999999891
sin^2 ( 75 ) + cos^2( 75 ) = .999999999999999999999999999999999999992
sin^2 ( 90 ) + cos^2( 90 ) = .9999999999999999999999999999999999999402
1 % 2 != 0
2 % 2 == 0
3 % 2 != 0
4 % 2 == 0
5 % 2 != 0
6 % 2 == 0
7 % 2 != 0
8 % 2 == 0
9 % 2 != 0
10 % 2 == 0
11 % 2 != 0
12 % 2 == 0
13 % 2 != 0
14 % 2 == 0
15 % 2 != 0
16 % 2 == 0
    \end{lstlisting}

    \textbf{Вывод:} в результате проделанной работы ознакомился с
    \begin{enumerate}\itemsep-5pt
        \item основными возможностями СУБД Oracle
        \item синтаксисом и структурой языка PL/SQL
        \item типами существующими типами данных
    \end{enumerate}
\end{document}