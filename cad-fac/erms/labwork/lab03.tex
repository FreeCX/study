\documentclass[pscyr, 12pt]{hedlab}
\usepackage[russian]{babel}
\usepackage{graphicx}
\usepackage{listings}

\labname{Разработка баз данных в СУБД Oracle}
\labnum{3}
\student{Голубев~А.~В., САПР-1.1п}
\labdate{}


\lstdefinestyle{Oracle}{%
    basicstyle=\footnotesize\ttfamily,
    keywordstyle=\lstuppercase,
    emphstyle=\itshape,
    showstringspaces=false,
}

\makeatletter
\newcommand{\lstuppercase}{\uppercase\expandafter{\expandafter\lst@token
                           \expandafter{\the\lst@token}}}
\newcommand{\lstlowercase}{\lowercase\expandafter{\expandafter\lst@token
                           \expandafter{\the\lst@token}}}
\makeatother

\lstdefinelanguage[Oracle]{SQL}[]{SQL}{
    morekeywords={ACCESS, MOD, NLS_DATE_FORMAT, NVL, REPLACE, SYSDATE, TO_CHAR, TO_NUMBER, TRUNC},
}
\lstset{language=[Oracle]SQL, style=Oracle,}

\begin{document}
    \makeheader
    \noindent\textbf{Цель:} ознакомиться с ключевыми понятиями распределенной СУБД Oracle -- транзакциями, блокировками и целостностью данных

    \noindent\textbf{Постановка задачи:}
    \vspace*{-1em}
    \begin{itemize}\itemsep-5pt
        \item создать таблицу с испльзованием ограничений целостности
        \item заполнить таблицу произвольными данными
        \item произвести изменение данных таблицы по параметру
        \item распечать содержимое таблицы
    \end{itemize}
  
    \noindent\textbf{SQL Запрос:}
    \lstinputlisting{source/lab03.sql}

    \pagebreak

    \noindent\textbf{Результат выполнения:}
    \lstinputlisting{source/lab03.out}

    \noindent\textbf{Вывод:} в результате проделанной работы\vspace*{-0.5em}
    \begin{enumerate}\itemsep-5pt
        \item получил навыки по созданию и работе с базой данных
        \item изучил основы работе с\vspace*{-1em}
        \begin{itemize}\itemsep-5pt
            \item ограничением целостности
            \item блокировкой таблицы
        \end{itemize}\vspace*{-0.5em}
        \item получил информацию по новым командам для работы с таблицами
    \end{enumerate}
\end{document}