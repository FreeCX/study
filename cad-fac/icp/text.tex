% \begin{frame}
%     \frametitle{Title}
%     \framesubtitle{Subtitle}
% \end{frame}

\begin{frame} % Слайд 1
    \begin{center}
        \small
        Волгоградский Государственный Технический Университет \\
        Факультет электроники и вычислительной техники \\
        Кафедра САПРиПК \\
        \vspace{1.5cm}
        \normalsize
        \textbf{Метод построения маршрутов общественного транспорта на основе 
        предпочтений жителей.}\\
        \vspace{1.0cm}
        \raggedleft\small
        \textbf{Исполнитель:}\\Голубев~А.~В.\\
        \textbf{Руководитель:}\\Щербаков~М.~В.\\
        \vspace{1.5cm}
        \vspace{\fill}
        \centeringВолгоград \the\year
    \end{center}
\end{frame}

\begin{frame} % Слайд 2
    % \emph{Актуальность.} Изменения в городской среде требуют формирования новых 
    % механизмов планирования инфраструктуры города. Для получения эффективных 
    % результатов, следует осуществлять принятие решений на основе актуальных 
    % данных, отражающих предпочтения жителей.  В рамках магистерской диссертации 
    % следует разработать метод построения маршрутов общественного транспорта на 
    % основе предпочтений жителей.
    
    % Объект исследования
    % Формулировка проблемы
    % Предмет исследования
\end{frame}

\begin{frame} % Слайд 3
    \textbf{Цель работы} -- разработка метода генерации маршрутов общественного 
    транспорта на основе предпочтений жителей для минимизации дискомфорта 
    перемещения в городе.
    % Теоретические и практические задачи
\end{frame}

\begin{frame} % Слайд 4
    % Понятийный аппарат
\end{frame}

\begin{frame} % Слайд 5
    \frametitle{Список литературы}
    % Список проанализированных источников
\end{frame}

\begin{frame} % Слайд 6
    % Описание прототипа
\end{frame}

\begin{frame} % Слайд 7
    % Описание прототипа
\end{frame}

\begin{frame} % Слайд 8
    % Результаты первого этапа
\end{frame}