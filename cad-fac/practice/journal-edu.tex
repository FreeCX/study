\documentclass[a4paper, 14pt]{extreport}
\usepackage[T2A]{fontenc}
\usepackage[utf8]{inputenc}
\usepackage[english, russian]{babel}
\usepackage[left=2.5cm, right=1.5cm, top=2.0cm, bottom=2.0cm]{geometry}
\usepackage[column]{hedfeatures}
\usepackage{paralist, enumerate, enumitem}
\usepackage{multirow}
\renewcommand{\rmdefault}{ftm}

\newcolumntype{L}[1]{>{\arraybackslash}m{#1\textwidth}}

\begin{document}
    \begin{titlepage}
        \begin{center}
            Министерство образования и науки РФ \\
            Государственное образовательное учреждение\\
            Высшего профессионального образования\\
            <<Волгоградский государственный технический университет>>\\
            Кафедра <<САПР~и~ПК>>
        \end{center}
        \vspace{2cm}
        \begin{center}
            \large \textbf{ДНЕВНИК} \\
            по педагогической практике \the\year\ г.
        \end{center}
        \begin{flushleft}
            Студента\\
            Фамилия \underline{Голубева\hspace{3.1cm}} 
            Имя \underline{Алексея\hspace{2.1cm}}\\
            Отчество \underline{Владимировича\hspace{1.6cm}}\\
            Факультет \underline{ФЭВТ\hspace{3.45cm}} курс \underline{2\hspace{1.5cm}} 
            группа \underline{САПР-2.1п\hspace{2.6cm}}\\
            \vspace{1cm}
            Направление подготовки \underline{Информатика и вычислительная техника\hspace{2.6cm}}\\
            \underline{\hspace{\textwidth}}
            База проведения практики \underline{ВолгГТУ\hspace{9.1cm}}\\
            \underline{\hspace{\textwidth}}\vspace{1cm}
            КАЛЕНДАРНЫЕ СРОКИ ПРАКТИКИ:\\
            По учебному плану: \hspace{0.2cm} начало \underline{\hspace{4.5cm}} 
            конец \underline{\hspace{4.5cm}}\\
            Дата прибытия на практику: <<\underline{\hspace{1cm}}>> \underline{\hspace{3cm}} 
            \the\year\ г.\\
            Дата выбытия: \hspace{3cm}<<\underline{\hspace{1cm}}>> \underline{\hspace{3cm}} 
            \the\year\ г.
        \end{flushleft}
        \vspace{2cm}
        \begin{flushleft}
            РУКОВОДИТЕЛЬ ПРАКТИКИ\\
            Кафедра \underline{САПР~и~ПК\hspace{2.4cm}} Должность \underline{профессор\hspace{2.8em}} \\
            Фамилия \underline{Садовникова\hspace{2.1cm}} Имя \underline{Наталья\hspace{3.45cm}}\\
            Отчество \underline{Петровна\hspace{2.8cm}}
        \end{flushleft}
        \vspace{\fill}
        \begin{center}
            Волгоград \the\year
        \end{center}
    \end{titlepage}
    \begin{center}
        \begin{tabular}{|C{.13}|L{.6}|L{.2}|L{.1}|}
            \hline
            Сроки (недели/ даты) & Работа, выполненная студентом & Отметка руководителя, подпись\\ \hline
            1-2 & Выбор основных направлений педагогической деятельности и оформление Индивидуального плана 
                практики (ИП) & \\ \hline
            \multicolumn{1}{|C{.13}|}{\multirow{8}{*}{3-8}} & Работа со студентами 4-го курса по бакалаврской 
                работе (16 часов). & \\ \cline{2-3}
            & Подготовка презентации к лекции <<Генетические алгоритмы>>. & \\ \cline{2-3}
            & Разработка примеров к лабораторной работе <<Генетические алгоритмы>>. & \\ \cline{2-3}
            & Разработка комплекта заданий к лабораторной работе <<Генетические алгоритмы>>. & \\ \cline{2-3}
            & Структурирование информации для методических указаний к лаб. работе <<Технология OpenStreetMap и 
                библиотека Leaflet>>. & \\ \cline{2-3}
            & Подготовка макета методических указаний <<Технология OpenStreetMap и библиотека Leaflet>>. 
                & \\ \cline{2-3}
            & Подготовка макета магистерской работы. & \\ \hline
            9 & Подготовка материалов отчёта. & \\ \hline
        \end{tabular}
    \end{center}
    \pagestyle{empty}
    \vspace{\fill}
    \noindent<<\underline{\hspace{1cm}}>> \underline{\hspace{5cm}} \the\year\ г.\\
    Студент \hspace{1cm} \underline{Голубев~А.~В.\hspace{3.2cm}} \hspace{2cm} (\underline{\hspace{5cm}})\\
    Руководитель \underline{Садовникова~Н.~П.\hspace{2.06cm}} \hspace{2cm} (\underline{\hspace{5cm}})\\
\end{document}
