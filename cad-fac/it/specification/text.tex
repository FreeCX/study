\chapter{Общие сведения}
\section{Полное наименование системы и её условное обозначение}
Полное наименование системы -- Автоматизированная игровая система.
Краткое наименование системы -- <<ИС>>. В дальнейшем просто -- система.

\section{Шифр темы или шифр (номер) договора}
Тема курсовой работы по Информационным технологиям номер 11: 
<<Игровые программы. Основы проектирования>>

\section{Наименование предприятий (объединений) разработчика и заказчика (пользователя)}
Заказчик -- кандидат технических наук Шкурина Галина Леонидовна\\
Исполнитель -- студент группы САПР-1.1п Голубев Алексей

\section{Плановые сроки начала и окончания работы по созданию системы}
Начало разработки -- 09.09.2014 г. Окончание разработки -- 01.04.2015 г.

\section{Порядок оформления и предъявления заказчику результатов работ по созданию системы}
Результаты работы предъявляются Заказчику в виде:
\begin{enumerate}
    \item исполняемых модулей и исходных текстов ПО на компакт-диске;
    \item дополнительные материалы: реферат, презентация, методическое описание.
\end{enumerate}

\chapter{Назначение и цели создания (развития) Системы}
\section{Назначение Системы}
Основным функциональным назначением системы является представление базовой структуры в 
проектировании игровых программ, ознакомление с основными алгоритмами используемыми в разработке, 
а также предоставление студенту базовых знаний работы с графическими библиотеками SDL2 и OpenGL.

\section{Цели создания Системы}
Целью создания является разработка базовой структуры игровой программы, тестового приложения 
использующую данную структуру, а также методических описаний по разработке.

\chapter{Характеристика объекта автоматизации}
\section{Краткие сведения об объекте автоматизации или ссылки на документы, содержащие такую 
    информацию}
В ходе проведения работ по разработке Системы автоматизируются процессы Заказчика. Система будет 
эксплуатироваться на персональном компьютере по выбору Исполнителя.

\section{Сведения об условиях эксплуатации объекта автоматизации}
Объектом автоматизации является:
\begin{enumerate}
    \item проектирование структуры игровой программы;
    \item 
\end{enumerate}
Условия эксплуатации
% Особых условий эксплуатации технических средств, использующихся для автоматизации объекта 
% автоматизации, не предъявляются.

\chapter{Требования к системе}

\subsection{Требования к структуре и функционированию системы}
В состав Системы должны входить следующие подсистемы
\begin{enumerate}
    \item Подсистема графической отрисовки;
    \item Подсистема визуальных эффектов;
    \item Подсистема управления игровым персонажем;
    \item Подсистема загрузки данных;
    \item Подсистема контроля ИИ;
    \item Подсистема контроля игровых событий;
    \item Подсистема обработки столновений;
    \item Подсистема подсчёт игрового счёта.
\end{enumerate}

\subsubsection{перечень подсистем, их назначение и основные характеристики, требования к числу 
    уровней иерархии и степени централизации системы}
\paragraph{Подсистема графической отрисовки}
предназначена для:
\begin{itemize}
    \item настройки и создания графического интерфейса;
    \item формирования графической составляющей игровой программы;
    \item работы с библиотекой OpenGL;
    \item вывод графической составляющей на экран.
\end{itemize}

\paragraph{Подсистема управления игровым персонажем}
предназначена для:
\begin{itemize}
    \item работы с устройствами ввода информации;
    \item привязки игровых действий к устройствам ввода.
\end{itemize}

\paragraph{Подсистема аудио вывода}
предназначена для:
\begin{itemize}
    \item работы с устройством вывода аудио информации;
    \item создания фонового аудио сопровождения;
    \item создания звукового отображения игровых событий.
\end{itemize}

\paragraph{Подсистема загрузки игровых данных}
предназначена для:
\begin{itemize}
    \item работы с различными файловыми форматами;
    \item загрузки конфигурационных, графических и аудио данных;
    \item обработки данных и преобразование к внутрисистемному формату.
\end{itemize}

\paragraph{Подсистема контроля ИИ}
предназначена для:
\begin{itemize}
    \item создания игровой логики для <<противников>>;
    \item реакции <<противников>> на игровые события.
\end{itemize}

\paragraph{Подсистема контроля игровых событий}
предназначена для:
\begin{itemize}
    \item создания игрового события;
    \item создания <<противников>> и астеройдов;
    \item создания игровых бонусов.
\end{itemize}

\paragraph{Подсистема обработки столновений}
предназначена для:
\begin{itemize}
    \item обработки столкновений между игровыми объектами;
    \item обработки столкновений между управляемыми объектами.
\end{itemize}

\paragraph{Подсистема игрового счёта}
предназначена для:
\begin{itemize}
    \item учёта игрового счёта;
    \item вывода игрового счёта;
    \item сохранения игрового счёта.
\end{itemize}

\subsection{Требования к режимам функционирования системы}
Режим функционирования -- по требованию пользователя.

\subsection{Требования по диагностированию системы}
Система должна удовлетворять следующим требованиям по диагностированию:
\begin{itemize}
    \item запись при возникновении системных ошибок в ходе выполения работы в системный журнал;
    \item журналирование работы подсистем;
    \item выдача пользователю сообщений, содержащих адекватное описание нарушения 
        работоспособности.
\end{itemize}
Во время опытной эксплуатации рекомендуется работа скомпилированного в отладочном режиме 
программного обеспечения для сохранения полной отладочной информации.

\subsection{Перспективы развития, модернизации системы}
Для приведения Системы к готовности для эксплуатации по результатам работы могут быть 
проведены работы в следующих направлениях:
\begin{itemize}
    \item Масштабируемость системы за счёт вынесения функций в отдельные модули с 
        последующей структуризацией;
    \item Создания модификаций на основе системы (замена игровых ресурсов, логических 
        структур и т.п.);
    \item Адаптации логики работы системы к изменениям в документах, регламентирующих 
        деятельность Заказчика.
\end{itemize}

\subsection{Требования к численности и квалификации персонала системы и режиму его работы}
\subsubsection{Требования к численности персонала (пользователей) АС}
С учётом типа разрабатываемой программы конкретных требований к численности персонала не 
приводится. В Системе предполагается наличие ролей пользователей -- пользователь 
обладающий возможностью работы с игровой программой.

\subsubsection{Требования к квалификации персонала}
Пользователи, использующие игровую программу, должны обладать базовыми навыками работы на 
персональном компьютере.

\subsection{Требования к надежности}
Надежное (устойчивое) функционирование программы должно быть обеспечено выполнением 
Заказчиком совокупности организационно-технических мероприятий, перечень которых приведен 
ниже: 
\begin{itemize}
    \item использованием лицензионного программного обеспечения; 
    \item использованием нового программного обеспечения;
    \item использованием отказоустойчивого оборудования;
    \item соблюдение сохранности игровых данных.
\end{itemize}

\subsection{Требования к эргономике и технической эстетике}
Требования к пользовательскому интерфейсу не специфицируются.

\subsection{Требования к эксплуатации, техническому обслуживанию, ремонту и хранению компонентов 
    системы}
Требования к эксплуатации, техническому обслуживание, ремонту и хранению компонентов системы 
не предъявляются.

\subsection{Требования к защите информации от несанкционированного доступа}
Обеспечение требований по защите информации от несанкционированного доступа возлагается на 
систему безопасности операционной системы.

\subsection{Требования по сохранности информации при авариях}
При авариях не должна нарушаться целостность данных. 

Требования надежности работы в целом и сохранности информации во время аварии должны быть 
учтены при выборе аппаратного обеспечения и квалификации обслуживающего персонала.

\section{Требования к видам обеспечения}
\subsection{Информационное обеспечение системы}
Информационный обмен между подсистемами должен удовлетворять следующим положениям:
\begin{itemize}
    \item все взаимодействия должны контролироваться подсистемой обработки ошибок;
    \item все подсистемы должны работать только со своими структурами данных.
\end{itemize}

\subsection{Программное обеспечение системы}
В Системе должны максимально использоваться программные продукты с открытой лицензией. 
Реализация программных модулей должна соответствовать текущим требованиям оформления 
программного кода с открытой лицензией.

\subsection{Техническое обеспечение системы}
\subsubsection{Требования к клиентскому аппаратному обеспечению}
Система должна функционировать на аппаратном обеспечении, на котором может быть запущено 
клиентское программное обеспечение, но для достижения оптимальной производительности 
необходима конфигурация приведенная ниже:
\begin{itemize}
    \item Одноядерный процессор 1 ГГц
    \item Не менее 32 Мб видеокарта
    \item Не меннее 512 Мб оперативной памяти
    \item По крайней мере, 200 Мб свободного места на диске
\end{itemize}

\subsection{Требования к математическому обеспечению}
<<НАПИСАТЬ>>

\subsubsection{Эксплуатационные требования}
Особых эксплуатационных требований к Системе не предъявляется.

\chapter{Состав и содержание работ по созданию (развитию) системы}
\textbf{Этап 1.}\\
Сроки исполнения первого этапа: 01.12.2014 -- 01.03.2015.

На первом этапе будут проведены следующие работы:
\begin{itemize}
    \item Разработка архитектуры Системы;
    \item Разработка модулей Системы;
    \item Разработка первой рабочей версии программной части Системы.
\end{itemize}
Итоговыми результатами по первому этапу являются:
\begin{itemize}
    \item Структура архитектуры Системы;
    \item Составление методического описания;
    \item Первая рабочая версия программной части.
\end{itemize}

\textbf{Этап 2.}\\
Сроки исполнения первого этапа: 01.01.2015 -- 01.03.2014.

На втором этапе будут проведены следующие работы:
\begin{itemize}
    \item Разработка графической составляющей;
    \item Разработка аудио составляющей;
\end{itemize}
Итоговыми результатами по первому этапу являются:
\begin{itemize}
    \item Рабочий прототип Системы.
\end{itemize}

\textbf{Этап 3.}\\
Сроки исполнения первого этапа: 01.03.2015 -- 01.04.2014.

На третьем этапе будут проведены следующие работы:
\begin{itemize}
    \item Тестирование Системы на наличие ошибок реализации;
    \item Отладка компонентов Системы;
    \item Внесение изменений в программный код.
\end{itemize}
Итоговыми результатами по первому этапу являются:
\begin{itemize}
    \item Рабочая версия разрабатываемой системы
\end{itemize}

\chapter{Порядок контроля и приемки Системы}
\section{Состав, объем и методы испытаний системы и ее составных частей}
Первая версия Системы должна пройти предварительные испытания, состоящие из функционального 
тестирования. Будут проведены испытания работы модулей системы с целью сбора перечня 
предложений и выявления недостатков. 

\section{Общие требования к приемке работ}
В процессе приёмки работ должна быть осуществлена проверка Системы на соответствии требованиям 
настоящего <<Технического задания>>.

В процессе приёмочных испытаний должен вестись журнал, в котором будут фиксироваться результаты 
выполненных работ, замечания по работе программного обеспечения и предложения по изменению работы 
программного обеспечения.

По результатам испытаний возможны доработки и исправления. Выявленные в ПО и документации 
недостатки Исполнитель исправляет за свой счёт в специально оговоренные после проведения 
испытаний сроки.

\chapter{Требования к составу и содержанию работ по подготовке объекта автоматизации к вводу 
    системы в действие}
Для подготовки объекта автоматизации к вводу системы в действие должны быть проведены 
следующие мероприятия:

\section{Технические мероприятия}
Подготовить аппаратные средства в соответствии с пунктом <<Техническое обеспечение системы>> 
данного Технического задания. Выполняется Заказчиком.

Установить на аппаратные средства операционную системы. Выполняется Заказчиком.

Настроить на аппаратных средствах программного обеспечения Системы. Выполняется Заказчиком.

\section{Организационные мероприятия}
Ознакомить пользователя с графическим интерфейсом и устройством управления. Выполняется совместно 
Исполнителем и ответственным подразделением Заказчика.

\chapter{Требования к документированию}
\begin{enumerate}
    \item Пояснительная записка к техническому проекту;
    \item Методические документы;
    \item Презентационные материалы.
\end{enumerate}