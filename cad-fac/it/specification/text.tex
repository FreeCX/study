\chapter{Общие сведения}
\section{Полное наименование системы и её условное обозначение}
Игровые программы. Основы проектирования.

\section{Шифр темы или шифр (номер) договора}
Тема курсовой работы по Информационным технологиям №11

\section{Наименование предприятий (объединений) разработчика и заказчика (пользователя)}
Заказчик -- кандидат технических наук Шкурина Галина Леонидовна\\
Исполнитель -- студент группы САПР-1.1п Голубев Алексей

\section{Перечень документов, на основании которых создается система}
\section{Плановые сроки начала и окончания работы по созданию системы}
Начало разработки -- 01.12.2014 г. Окончание разработки -- 01.04.2015 г.

\section{Порядок оформления и предъявления заказчику результатов работ по созданию системы}
Результаты работы предъявляются Заказчику в виде:
\begin{enumerate}
    \item исполняемых модулей и исходных текстов ПО на компакт-диске;
    \item дополнительные материалы: реферат, презентация.
\end{enumerate}

\chapter{Назначение и цели создания (развития) Системы}
\section{Назначение Системы}
Разрабатываемая система предназначена для ознакомления с основой проектирования игровых программ и 
используемыми для этого алгоритмами.

\section{Цели создания Системы}
Целью создания является разработка игровой программы, показывающая основы проектирования систем 
такого типа.

\chapter{Характеристика объекта автоматизации}
\section{Краткие сведения об объекте автоматизации или ссылки на документы, содержащие такую 
    информацию}
\section{Сведения об условиях эксплуатации объекта автоматизации}

\chapter{Требования к системе}
\section{Требования к системе в целом}
\subsection{Требования к структуре и функционированию системы}
\subsubsection{перечень подсистем, их назначение и основные характеристики, требования к числу 
    уровней иерархии и степени централизации системы}
\paragraph{(подсистемы)}
\subsection{Требования к характеристикам взаимосвязей создаваемой системы со смежными системами}
\subsection{Требования к режимам функционирования системы}
\subsection{Требования по диагностированию системы}
\subsection{Перспективы развития, модернизации системы}
\subsection{Требования к численности и квалификации персонала системы и режиму его работы}
\subsubsection{Требования к численности персонала (пользователей) АС}
\subsubsection{Требования к квалификации персонала}
\subsubsection{Требуемый режим работы персонала АС}
\subsection{Требования к надежности}
\subsection{Требования безопасности}
\subsection{Требования к эргономике и технической эстетике}
\subsection{Требования к эксплуатации, техническому обслуживанию, ремонту и хранению компонентов 
    системы}
\subsection{Требования к защите информации от несанкционированного доступа}
\subsection{Требования по сохранности информации при авариях}
\section{Требования к функциям (задачам), выполняемым системой}
\subsection{(описание подсистем)}
\section{Требования к видам обеспечения}
\subsection{Информационное обеспечение системы}
\subsection{Программное обеспечение системы}
\subsection{Техническое обеспечение системы}
\subsubsection{Требования к клиентскому аппаратному обеспечению}
\subsubsection{Эксплуатационные требования}

\chapter{Состав и содержание работ по созданию (развитию) системы}

\chapter{Порядок контроля и приемки Системы}

\section{Состав, объем и методы испытаний системы и ее составных частей}
\section{Общие требования к приемке работ}

\chapter{Требования к составу и содержанию работ по подготовке объекта автоматизации к вводу 
    системы в действие}
\section{Технические мероприятия}
\section{Организационные мероприятия}

\chapter{Требования к документированию}

\chapter{Источники разработки}
