\begin{frame}
    \begin{center}
        \vspace{3.0cm}
        \normalsize
        \textbf{Игровые программы. Основы проектирования.} \\
        \vspace{1.5cm}
        \raggedleft\small\textbf{Выполнил:}\\Голубев~А.~В.\\САПР-1.1п\\
        \vspace{1.8cm}
        \vspace{\fill}
        \centeringВолгоград \the\year
    \end{center}
\end{frame}

\begin{frame}
    Оглавление:
    \tableofcontents
\end{frame}

\section{Введение}
\begin{frame}
    \frametitle{Введение}
    \raggedleftКомпьютерные игры -- это искусство.
    \begin{figure}
        \begin{minipage}{0.47\textwidth}
            \includegraphics[width=1.0\textwidth]{skyrim}
        \end{minipage}
        \begin{minipage}{0.51\textwidth}
            \includegraphics[width=1.0\textwidth]{watch_dogs}
        \end{minipage}
        \begin{minipage}{0.47\textwidth}
            \includegraphics[width=1.0\textwidth]{mario}
        \end{minipage}
        \begin{minipage}{0.51\textwidth}
            \includegraphics[width=1.0\textwidth]{fallout2}
        \end{minipage}
    \end{figure}
\end{frame}

\section{История развития}
\begin{frame}
    \frametitle{История развития}
    Вехи развития компьютерных игр:
    \begin{itemize}
        \item 1889 -- Фусадзиро Ямаути основал игровую компанию Marufuku по производству и 
            продаже игральных карт Ханафуда (ныне Nintendo).
        \item 1947 -- \emph{Ракетный симулятор} -- первое известное развлекательное средство, похожее на 
            компьютерную игру.
        \item 1948—1950 -- Алан Тьюринг и Дэйвид Чампернаун разработали алгоритм шахматной игры. 
        \item 1971 -- Биллом Питтсом создаётся первый аркадный автомат Galaxy Game на 
            базе PDP-11.
        \item 1975 -- Atari выходит на рынок игровых приставок с моделью Pong.
        \item 1978 -- Taito выпускает новый аркадный автомат Space Invaders.
    \end{itemize}
\end{frame}

\section{Классификация компьютерных игр}
\begin{frame}
    \frametitle{Классификация компьютерных игр}
    Классификация компьютерный игр по жанрам:
    \begin{itemize}
        \item Action (\emph{Space Invaders, Doom})\\
            \tiny Действие таких игр развивается очень динамично и требует напряжения внимания и 
            быстрой реакции на происходящие в игре события.
        \item \normalsize Adventure (\emph{Deponia, Monkey Island})\\
            \tiny Игра-повествование, в которой управляемый игроком герой продвигается по сюжету 
            и взаимодействует с игровым миром посредством применения предметов, общения с другими 
            персонажами и решения логических задач.
        \item \normalsize Arcade (\emph{Battle City, Mega Man})\\
            \tiny Игра, в которой игроку приходится действовать быстро, полагаясь в первую очередь 
            на свои рефлексы и реакцию.
        \item \normalsize Fighting (\emph{Mortal Kombat, Street Fighter})\\
            \tiny Геймплей состоит исключительно из поединков двух и более противников с применением 
            рукопашного боя.
        \item \normalsize FPS и TPS (\emph{Max Payne, Tomb Raider})\\
            \tiny В шутерах от первого лица (FPS) игрок не видит персонажа со стороны -- он наблюдает 
            за происходящим от лица персонажа -- <<глазами персонажа>>, и наблюдаемая игроком картина 
            совпадает с тем, что <<видит>> персонаж. В шутерах от третьего лица (TPS) игрок видит 
            персонаж со стороны с фиксированной или произвольной точки зрения.
    \end{itemize}
\end{frame}

\begin{frame}
    \frametitle{Классификация компьютерных игр}
    \begin{itemize}
        \item Hack and Slash (\emph{Diablo, Devil May Cry}) \\
            \tiny Игры с видом от третьего лица, основной частью игрового процесса, в которых 
            являются фехтовальные поединки с применением холодного и другого оружия.
        \item \normalsize Puzzle (\emph{Sokoban, Portal}) \\
            \tiny Название жанра компьютерных игр, целью которых является решение логических задач, 
            требующих от игрока задействования логики, стратегии и интуиции или в иных случаях 
            некоторого наличия удачи.
        \item \normalsize Rhythm game (\emph{Guitar Hero, Rock Band}) \\
            \tiny В музыкальных играх геймплей строится на взаимодействие игрока с музыкой. 
            Жанр же может быть любой, от головоломок до ритм игр.
        \item \normalsize RPG (\emph{Fallout, TES}) \\
            \tiny Жанр компьютерных игр, основанный на элементах игрового процесса традиционных 
            настольных ролевых игр.
        \item \normalsize RTS и TBS (\emph{Civilization, C\&C}) \\
            \tiny Игры, требующие планирования и выработки определенной стратегии для достижения 
            некоей конкретной цели, например, победы в военной операции.
        \item \normalsize Simulator (\emph{NFS, FIFA}) \\
            \tiny Игры, предоставляющие возможность симуляции и управления тем или иным процессом 
            из реальной жизни.
        \item \normalsize Traditional (\emph{карты, шашки}) \\
            \tiny Компьютерная реализация настольных игр.
    \end{itemize}
\end{frame}

\begin{frame}
    \frametitle{Классификация компьютерных игр}
    Так же существуют другие виды классификаций компьютерных игр:
    \begin{itemize}
        \item По платформе
        \item По графическому изображению игры
        \item По содержанию
        \item По цели
        \item По издательским критериям
        \item По издательскому формату
        \item По количеству игроков
    \end{itemize}
\end{frame}

\section{Специализации разработчиков}
\begin{frame}
    \frametitle{Специализации разработчиков}
    В состав типичной современной команды разработчиков обычно входят представители разных 
    специализаций представленных ниже:
    \begin{figure}
        \begin{minipage}{0.47\textwidth}
            \begin{itemize}
                \item Графика
                \item Дизайн
                \item Звук
                \item Контроль качества
                \item Программирование
                \item Управление
            \end{itemize}
        \end{minipage}
        \begin{minipage}{0.5\textwidth}
            \includegraphics[width=1.0\textwidth]{ubisoft}
        \end{minipage}
    \end{figure}
\end{frame}

\section{Процесс разработки}
\begin{frame}
    \frametitle{Процесс разработки}
    Процесс разработки игры меняется в зависимости от компании и проекта. Однако разработка 
    коммерческой игры обычно включает следующие стадии:
    \begin{figure}
        \begin{minipage}{0.47\textwidth}
            \begin{itemize}
                \item Предпроизводственный процесс
                \item Производство
                \item Поддержка
            \end{itemize}
        \end{minipage}
        \begin{minipage}{0.5\textwidth}
            \includegraphics[width=1.0\textwidth]{work}
        \end{minipage}
    \end{figure}
\end{frame}

\begin{frame}
    \frametitle{Процесс разработки}
    \framesubtitle{Предпроизводственный процесс}
    Предпроизводственный процесс включает в себя следующие пункты:
    \begin{figure}
        \begin{minipage}{0.47\textwidth}
            \begin{itemize}
                \item Формирование идеи
                \item Определение жанра
                \item Создание геймплея
                \item Эскизный проект
                \item Документация проектировщика
            \end{itemize}
        \end{minipage}
        \begin{minipage}{0.5\textwidth}
            \includegraphics[width=1.0\textwidth]{idea}
        \end{minipage}
    \end{figure}
\end{frame}

\begin{frame}
    \frametitle{Процесс разработки}
    \framesubtitle{Производственный процесс}
    Производственный процесс состоит из работы над:
    \begin{figure}
        \begin{minipage}{0.47\textwidth}
            \begin{itemize}
                \item Процесс разработки
                \begin{itemize}
                    \item Программирование
                    \item Графика
                    \item Дизайн
                    \item Эффекты
                    \item Звук
                \end{itemize}
                \item Контроль качества
                \item Управление разработкой
            \end{itemize}
        \end{minipage}
        \begin{minipage}{0.5\textwidth}
            \includegraphics[width=1.0\textwidth]{programming}
        \end{minipage}
    \end{figure}
\end{frame}

\begin{frame}
    \frametitle{Процесс разработки}
    \framesubtitle{Процесс поддержки}
    Поддержка готового продукта состоит из:
    \begin{figure}
        \begin{minipage}{0.47\textwidth}
            \begin{itemize}
                \item Исправление ошибок и выпуск патчей
                \item Расширение игровой функциональности
                \item Выпуск DLC
            \end{itemize}
        \end{minipage}
        \begin{minipage}{0.5\textwidth}
            \includegraphics[width=1.0\textwidth]{bugs}
        \end{minipage}
    \end{figure}
\end{frame}

\section{Аутсорсинг}
\begin{frame}
    \frametitle{Аутсорсинг}
    Планы касательно аутсорсинга рассматривают на этапе подготовки производства; именно тогда 
    рассчитывают необходимые временные и финансовые затраты на работу, которая будет произведена вне 
    компании-разработчика.
    \begin{figure}
        \begin{minipage}{0.47\textwidth}
            \begin{itemize}
                \item Модульные инструменты
                \item Музыкальные треки
                \item Актёрская озвучка
                \item Захват движений
            \end{itemize}
        \end{minipage}
        \begin{minipage}{0.5\textwidth}
            \includegraphics[width=1.0\textwidth]{mocap}
        \end{minipage}
    \end{figure}
\end{frame}

\section{Перспективность}
\begin{frame}
    \frametitle{Перспективность}
    Примеры удачных стартов:
    \begin{itemize}
        \item \emph{FEZ} (\url{http://fezgame.com/})
        \item \emph{Minecraft} (\url{https://minecraft.net/})
        \item \emph{World of Goo} (\url{http://worldofgoo.com/})
    \end{itemize}
    Использование цифровой дистрибуция:
    \begin{itemize}
        \item Steam
        \item Direct2Drive
        \item GameTap
        \item Origin
    \end{itemize}
\end{frame}

\section{Список источников}
\begin{frame}
    \frametitle{Список источников}
    \begin{itemize}
        \item \url{http://www.dtf.ru/}
        \item \url{http://www.gamedev.net/}
        \item \url{http://www.gamedev.ru/}
        \item \url{http://www.habrahabr.ru/}
        \item \url{http://www.kotaku.com/}
        \item \url{http://www.gamasutra.com/}
        \item \url{http://www.gamedaily.com/}
        \item \url{http://www.ign.com/}
    \end{itemize}
\end{frame}

\section{Вывод}
\begin{frame}
    \frametitle{Вывод}
\end{frame}

\begin{frame}
    \Huge\centeringСпасибо за внимание!
\end{frame}