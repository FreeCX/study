\chapter{Введение} % черновой вариант текста [изменения неизбежны]
% Что почитать
% http://habrahabr.ru/post/250745/
% http://habrahabr.ru/company/mosigra/blog/177249/
% http://inwebwetrust.org/trust/Game_Development_Dynamics_Playdeck.html
% https://developer.valvesoftware.com/wiki/Game_Mechanics_%28Portal_2%29:ru

В данном методическом руководстве я не буду давать информацию по настройке и установки того или иного компонента, так же как и советовать какую-то 
определенную среду разработки (IDE). Попытаюсь рассказать некоторые тонкости в разработке структуры программы: частые ошибки, возможные проблемы и на что 
стоит обратить внимание при проектировании и разработке.

Примеры в основном будут на языке C++ с использованием следующих библиотек:
\begin{enumerate}
    \item SDL2
    \item OpenAL
\end{enumerate}
но также буду стараться прилагать блок-схемы и псевдокод (а также поясняющие изображения).

Я глубоко надеюсь что прежде чем разрабатывать игру определил для себя главные пункты в разработке игры, а именно: \emph{идея}, \emph{жанр}, \emph{история}, 
\emph{стиль}, \emph{игровая механика} разрабатываемой игры (не все, но желательно). Но не стоит расстраиваться, что у вас нет ещё идеи для игры -- давайте 
вместе шаг за шагом сделаем ремейк старой игры под названием Asteroids.

% предполагаемая структура книги / мануала
\chapter{Основа игровой программы}
\section{Введение}
\section{Проектирование структуры игры}
\section{Блок-схемы и псевдокод}
\section{Программная реализация скелета}
\section{Игровая механика}
\subsection{Физика}
Что использовать: простой самописный движок или готовое решение?
\subsection{Рейтинг, бонусы и прочее}
\section{Логика игры}
\section{...}

\chapter{Работа с графикой}
\section{Загрузка}
\section{Отрисовка}
\section{Анимация}
\section{Создание шрифтов}
\section{Эффекты (создание и использование)}
\section{...}

\chapter{Работа со звуком}
\section{Инициализация}
\section{Воспроизведение}
\section{...}

\chapter{Работа с игровыми ресурсами}
\section{Модуль загрузки / выгрузки}
\section{...}

\chapter{Игровые события}
\section{Связь игровых событий с графикой и звуком}
\section{Обратная связь (с игроком)}
\section{...}

\chapter{Справка}
\section{Где взять игровые ресурсы}
Список источников и лицензирование.
\section{Библиотека SDL2}
\section{Библиотека OpenGL}
\section{Библиоткеа OpenAL}
\section{...}

\renewcommand{\bibname}{Список используемой литературы}
\addcontentsline{toc}{chapter}{Список используемой литературы}
\begin{thebibliography}{10}
    \bibitem{main} Sylvester,~T. Designing Games: A Guide to Engineering Experiences. O'Reilly Media, 2013, 
\end{thebibliography}