\chapter{Введение} % черновой вариант текста [изменения неизбежны]
% Что почитать
% http://habrahabr.ru/post/250745/
% http://habrahabr.ru/company/mosigra/blog/177249/
% http://inwebwetrust.org/trust/Game_Development_Dynamics_Playdeck.html
% https://developer.valvesoftware.com/wiki/Game_Mechanics_%28Portal_2%29:ru

% http://habrahabr.ru/post/255561/

В данном методическом руководстве я не буду давать информацию по настройке и установки того или иного компонента, так же как и советовать какую-то 
определенную среду разработки (IDE). Попытаюсь рассказать некоторые тонкости в разработке структуры программы: частые ошибки, возможные проблемы и на что 
стоит обратить внимание при проектировании и разработке.

Примеры в основном будут на языке C++ с использованием следующих библиотек:
\begin{itemize}
    \item SDL2
    \item OpenGL
    \item GLUT
    \item OpenAL
    \item vorbis
\end{itemize}
но также буду стараться прилагать блок-схемы и псевдокод (а также поясняющие изображения).

Я глубоко надеюсь что прежде чем разрабатывать игру определил для себя главные пункты в разработке игры, а именно: \emph{идея}, \emph{жанр}, \emph{история}, 
\emph{стиль}, \emph{игровая механика} разрабатываемой игры (не все, но желательно). Но не стоит расстраиваться, что у вас нет ещё идеи для игры -- давайте 
вместе шаг за шагом сделаем ремейк старой игры под названием Asteroids.

% предполагаемая структура книги / мануала
\chapter{Основа игровой программы}
\section{Введение}
\section{Проектирование структуры игры}
\section{Блок-схемы и псевдокод}
\section{Программная реализация скелета}
\section{Игровая механика}
\subsection{Физика}
Что использовать: простой самописный движок или готовое решение?
\subsection{Рейтинг, бонусы и прочее}
\section{Логика игры}
\section{...}

\chapter{Работа с графикой} % описание работы с графикой
\section{Загрузка}
\section{Отрисовка}
\section{Анимация}
\section{Создание шрифтов}
\section{Эффекты (создание и использование)}
\section{...}

\chapter{Работа со звуком} % описание работы с библиотеками OpenAL, vorbis
\section{Инициализация}
\section{Воспроизведение}
\section{...}

\chapter{Работа с игровыми ресурсами}
\section{Модуль загрузки / выгрузки}
\section{...}

\chapter{Игровые события}
\section{Связь игровых событий с графикой и звуком}
\section{Обратная связь (с игроком)}
\section{...}

\chapter{Справка}
\section{Полезные ресурсы разработчику}
Если у вас в команде нет художников или музыкантов, или вам нужно узнать определенные подробности по работе с тем или иным инструментом, то во многих случаях 
можно воспользоваться помощью интернета. Заказать у художника арты, у композитора фоновую музыку или обсудить тонкий вопрос по использования специфичного 
алгоритма, то в данном случае лучшим решением будет обратиться к следующим интернет источникам.

\begin{itemize}\itemsep-5pt
    \item FAQ по геймдизайну \url{http://www.sloperama.com/advice.html}
    \item Авторский блог Антона Карпова по разработке игр \url{http://www.ant-karlov.ru/}
    \item Аналог stackoverflow по gamedev \url{http://gamedev.stackexchange.com/}
    \item Блог посвященный созданию 2D графики \url{http://2dgameartforprogrammers.blogspot.ru/}
    \item Блог Райана Швайдера по gamedesign \url{http://www.nerfbat.com/lessons/}
    \item Источник свободных игровых ресурсов (CC0) \url{http://opengameart.org/}
    \item Крупнейшее сообщество модостроителей \url{http://moddb.com/}
    \item Крупные сообщества по игрострою \url{http://www.gamedev.ru/} (rus) и \url{http://www.gamedev.net/} (eng)
    \item Любительские конкурсы по разработке игр \url{http://igdc.ru/}
    \item Новости индустрии, блоги разработчиков и многое другое \url{http://gamasutra.com/}
    \item Сайт для специалистов по компьютерной графике \url{http://devmaster.net/}
    \item Сайт посвященный pixelart \url{http://www.pixeljoint.com/}
    \item Уроки по созданию 3D \\
        \url{http://www.digitaltutors.com/subject/game-development-asset-creation-tutorials}
    \item Игровые спрайты старый игр \url{http://www.videogamesprites.net/}
    \item Свободный игровой контент \url{http://freegamearts.tuxfamily.org/}
    \item База свободных звуков \url{http://www.freesound.org/}
    \item Открытый музыкальный архив \url{http://www.openmusicarchive.org/}
    \item Базы 3D моделей \url{http://www.katorlegaz.com/3d_models/index.php}\\
        \url{https://3dwarehouse.sketchup.com/?redirect=1}
    \item CG текстуры \url{http://www.cgtextures.com/}
    \item Свободные архивы по текстурам \url{http://www.mayang.com/textures/}\\
        \url{http://www.texturearchive.com/}
    \item Библиотеки шрифтов \url{http://openfontlibrary.org/}\\
        \url{https://www.acidfonts.com/}
    \item Хабрахабр \url{http://habrahabr.ru}
\end{itemize}

\renewcommand{\bibname}{Список используемой литературы}
\addcontentsline{toc}{chapter}{Список используемой литературы}
\begin{thebibliography}{10}
    \bibitem{main} Sylvester,~T. Designing Games: A Guide to Engineering Experiences. O'Reilly Media, 2013, 
    \bibitem{pat} Nystrom,~R. Game Programming Patterns. Available at: \url{http://gameprogrammingpatterns.com/}
\end{thebibliography}