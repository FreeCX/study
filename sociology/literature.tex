\renewcommand{\bibname}{Список литературы}
\addcontentsline{toc}{chapter}{Список литературы}
\begin{thebibliography}{99}
    \bibitem{volch} Волчкова Л.Т., Минина В.Н. Стратегии социологического 
        исследования бедности // Социс. 1999. № 1.
    \bibitem{volch_s1} Быкова С.Н., Любин В.П. Бедность по-русски и 
        по-итальянски. Рец. на кн. Корни бедности: социальная ткань, семья и 
        бедность в Болонье в 90-е годы. Милан, 1992 г. 389 с. 
        // Социол. исслед. 1993. № 2. С. 134.
    \bibitem{volch_s2} Гордон Л . Четыре рода бедности в современной 
        России //Социологический журнал. 1994. № 4. С. 25.
    \bibitem{volch_s4} Ключевский В.О. Исторические портреты. М., 1990. С. 
        77-94; Сперанский С. К истории нищенства в России. СПб., 1887. С. 2.
    \bibitem{volch_s5} Линев Д.А. Причины русского нищенства и необходимые 
        против них меры. СПб. 1891
    \bibitem{volch_s6} Джордж Г. Преступность бедности. Пер. с англ. С.Д. 
        Николаева. М., 1906.
    \bibitem{vestnik} Институт социологии Российской академии наук 
        [Электронный ресурс] : Бедность и неравенства в современной России: 
        10 лет спустя // Москва -- Аналитический доклад -- Режим доступа: 
        \url{http://www.isras.ru/analytical_report_bednost_i_neravenstva.html} 
    \bibitem{test01} Львов С.В. Образы бедности и богатства в Российском 
        общественном сознании [Электронный ресурс] : Мониторинг общественного 
        мнения : электронный журнал №1 (81), январь -- март 2007. 
        -- Режим доступа: \url{http://ecsocman.hse.ru/data/2013/07/19/1251237630/s34-44_Journal_Monitoring81.pdf}
\end{thebibliography}

\newpage