\chapter{Теоретическое исследование}

\section{Введение в проблему}
О бедности и бедных в современной России говорят и пишут многие. При этом 
кто-то представляет бедных как членов общества с относительно малыми доходами 
и состояниями. Кто-то связывает состояние бедности с общим стандартом уровня 
жизни в данном обществе. Есть и такие, кто, не мудрствуя лукаво, считает 
бедными тех людей, которые не имеют достаточных или необходимых средств к 
существованию. И в данном случае приходится задумываться, а какова мера 
<<достаточности и необходимости>> подобных средств, и какими показателями 
должен характеризоваться и кем устанавливаться общий стандарт жизненного 
существования жителей Российской Федерации? 

Феномен бедности стал предметом исследования в современной отечественной 
социологии в начале 90-х годов. В советский период понятие бедности в науке 
не использовалось. В социально-экономической литературе официальное признание 
получила категория малообеспеченности, которая раскрывалась в рамках теории 
благосостояния и социалистического распределения. По мнению Н. Римашевской, 
термин <<малообеспеченность>> использовался в двух значениях. В одном из них 
под малообеспеченностью понималось потребление ниже общественно необходимого 
минимального уровня. Второе расширенное толкование значения данного термина 
связывалось с существованием семей, уровень потребления которых был ниже 
наиболее распространенного. Оценка масштабов малообеспеченности в 
доперестроечной России осуществлялась на основе использования показателей 
минимального, нормативного, перспективного и рационального бюджетов, 
исчисляемых в бывшем СССР. Исходя из минимального бюджета определялись 
семьи, которые получали государственные пособия. В представлениях населения 
бедными считались неблагополучные семьи, неспособные обеспечить условия 
своего существования на уровне общепринятых норм.

Анализ современных публикаций, посвященных феномену бедности, и обращение к 
практике административного управления позволяют выделить четыре класса 
исследований, различающихся по целям, задачам и направленности: практически 
ориентированные мониторинги общественного мнения россиян об уровне их жизни, 
социальном самочувствии, о волнующих их проблемах для непосредственного 
использования полученной информации в системе социального управления; 
эмпирические исследования уровня и качества жизни россиян, социальной 
дифференциации населения, а также исследования собственно феномена бедности в 
России для определения ее масштабов, динамики и тенденций, для выявления 
стратегий поведения различных социальных групп, оказавшихся в состоянии нужды; 
теоретические исследования: освоение зарубежных концепций бедности, 
применение классических социологических теорий для объяснения данного явления, 
а также формирование авторских концепций бедности в российском обществе; 
прикладные исследования тенденций благосостояния населения, ориентированные 
на проверку теоретических конструкций, научных гипотез, выработку рекомендаций 
по преодолению или смягчению проблемы бедности.

В административном управлении при разработке государственной социальной 
политики, законов, направленных на социальную защиту населения, программ 
вспомоществования, как правило, опираются на статистическую информацию и в том 
числе данные бюджетных обследований семей. На этой базе рассчитывают 
прожиточный минимум, включающий в себя минимально необходимую потребительскую 
корзину, а также расходы на обязательные платежи. Он дифференцирован по 
социально-демографическим группам и учитывает региональную специфику. Данный 
показатель в соответствии с существующим законодательством предназначается 
для регулирования минимального уровня заработной платы, пенсий, стипендий, 
других социальных трансфертов. В настоящее время он является критерием для 
принятия решения об оказании социальной помощи в системе административного 
управления \cite{volch}.

Описывая феномен бедности, некоторые авторы предпринимают попытку 
охарактеризовать новые формы, в которых она проявляется в обществе. В 
частности, С. Быкова и В. Любин обращают внимание на различие понятий старой 
(традиционной, материальной) и новой (символической) бедности. Авторы 
ссылаются на мнение итальянских исследователей, согласно которому бедность не 
существует в единой форме, не ограничивается старыми традиционными группами, 
она как бы <<расползается>>. Поэтому наблюдается расхождение между 
функционированием служб социальной помощи и некоторыми формами бедности, 
не вписывающимися в деятельность этих служб, поскольку последние продолжают 
использовать критерии и методы, разработанные 30 лет назад \cite{volch_s1}.

Л. Гордон наряду с традиционным рассмотрением абсолютных и относительных 
разновидностей бедности предлагает различать бедность <<слабых>> и <<сильных>>. 
Бедность <<слабых>> -- это так называемая социальная бедность 
(нетрудоспособных, инвалидов, больных и т.д.), которая требует постоянного к 
себе внимания в любом обществе. Бедность <<сильных>> возникает в чрезвычайных 
условиях, когда работники лишаются возможности за счет своего труда обеспечить 
общепринятый уровень благосостояния. С этой точки зрения, бедность 
<<сильных>> можно обозначить как производственно-трудовую, экономическую 
\cite{volch_s2}. Выделение производственно-трудовой и социальной бедности важно 
в связи с необходимостью решения задач социальной политики. Смягчение 
социальной бедности требует преимущественно прямой помощи в виде денежных 
выплат или предоставления натуральных благ. Экономическая бедность устраняется 
главным образом косвенно посредством со-здания условий, стимулирующих и 
развивающих трудовую активность. Такое различие видов бедности, на наш взгляд, 
имеет принципиальное значение для обоснования направлений деятельности 
администраций в отношении нетрудоспособной части бедного населения и 
экономически активной ее части. Среди них социальная помощь, социальная 
поддержка и социальная защита.

\section{Изучение бедности как социальной проблемы}
Под бедностью в широком смысле слова мы понимаем такое состояние, при котором 
возникает несоответствие между достигнутым средним уровнем удовлетворения 
потребностей и возможностями их удовлетворения у от-дельных социальных 
групп, слоев населения. Она характеризуется неразвитостью самих потребностей, 
стремлением удовлетворять материальные потребности и нужды в ущерб духовным и 
социальным, нарушением социальных связей. Это приводит к низкой материальной 
обеспеченности определенных групп людей, к изменению их системы ценностей, к 
формированию особого социального мира и своей культуры (субкультуры бедности), 
жизненного стиля, диссонирующего с общепринятым, утверждавшимся в обществе, 
что вызывает угрозу нормального функционирования последнего.

Богатство и бедность выступают характеристиками социального развития общества. 
Они порождаются действием механизма социальной дифференциации в сферах 
производства, распределения и потребления жизненных благ. Богатство и бедность 
связаны с диалектикой прогресса и регресса в социальном развитии, которая 
состоит в том, что прогрессу многостороннего развития человека и общества 
как высшей формы выражения богатства постоянно противостоит регрессивная 
тенденция: сковывание потенциала личности, порабощение духа, развитие одной 
сферы общества за счет другой или даже застой и всеобщий упадок всех его 
сфер. Более того, регресс как момент отрицания содержится в самом прогрессе. 
Это означает, что прогресс в развитии общества и человека содержит в себе 
регресс, заключающийся в воспроизводстве и при определенных условиях усилении 
социального неравенства между людьми.

Понятия богатства и бедности относительные. Так, например, человек, 
материально не обеспеченный, то есть относящийся к бедным, может быть богат 
духом. В этой связи обращает на себя внимание то, что в русском православии 
нищенство, особенно добровольное, рассматривалось как состояние, которое 
несравненно выше состояния богатства -- как особый христианский подвиг, 
который осуществляется страждущим. Причем последний принимает на себя заботу о 
духе тех, кто пребывает в благости, вознося за них молитвы Богу. Об этом 
писали В. Ключевский, С. Сперанский\cite{volch_s4}.

В социальных науках и в общественной жизни бедность и ее крайняя форма -- 
нищета -- оцениваются как социальное зло. Зло бедности в том, что она есть 
преступление общества, т.е. всех людей: и бедных, и богатых, -- с точки зрения 
тех социальных последствий, которые она имеет. Об этом писали в свое время 
Д. Линев \cite{volch_s5}, Г. Джордж \cite{volch_s6}. <<Порок, преступления, 
невежество, подлость, порождаемые бедностью, -- по словам Г. Джорджа, -- 
отравляют, так сказать, самый воздух, которым дышат как бедные, так и 
богатые>>.

Изменения взглядов россиян на причины бедности, как и особенности портрета 
бедных в массовом сознании в современной России, во многом объясняют, почему 
в последние годы восприятие бедности индивидуализируется. Действительно, одно 
дело, когда люди оказываются в бедности из-за смерти кормильца семьи, тяжёлой 
болезни кого-то из членов домохозяйства и т. п., а государство не учитывает 
возникающих при этом рисков бедности и практически в подобных ситуациях не 
оказывает помощь -- к таким людям россияне и сегодня в массе своей относятся 
с сочувствием и жалостью. И совсем другое, когда к бедности приводят 
алкоголизм и наркомания, что, судя по всему, происходит всё чаще. Таким бедным 
типичный россиянин отнюдь не склонен сочувствовать, и не понимает, почему за 
счёт его благосостояния им надо помогать из бюджетных средств как бы плохо 
они не жили -- тем более что <<всё равно ведь пропьют>>. И в этом отношении 
\emph{идеология усиления адресности социальной помощи с выделением степени 
нуждаемости как главного критерия для помощи человеку, активно 
пропагандируемая в последние годы, приходит в полное противоречие с жизненным 
опытом и взглядами рядовых граждан страны.}

Примечательно, что представления бедных <<по доходу>> о причинах скатывания 
людей в бедность в нынешних условиях заметно отличаются от представлений всего 
населения. Правда, это касается не столько оценок причин бедности их 
ближайшего окружения, сколько причин их собственной бедности. Сами бедные 
относительно чаще выбирают сегодня три главные причины собственной бедности -- 
длительная безработица, недостаточность государственных пособий по социальному 
обеспечению и семейные несчастья. При этом они гораздо реже, чем россияне в 
целом, говорят о таких причинах бедности как алкоголизм, лень, 
неприспособленность к жизни, нежелание менять ради заработков привычный образ 
жизни и даже болезнь. Таким образом, они не видят в массе своей собственной 
ответственности за свою бедность и винят в ней государство, которое не 
обеспечило их ни работой, ни пособиями, достаточными для преодоления состояния 
бедности. При этом причины подобного существования своих знакомых бедные 
<<по доходам>> оценивают достаточно близко к группе подобных причин, которые 
характерны для массового восприятия россиян в целом \cite{vestnik}.

\section{Неравенства и справедливость}
Бедность в России тесно связана с еще одной ключевой социально-экономической 
проблемой -- проблемой избыточных социальных неравенств, ситуация с которыми в 
последние годы в стране, судя по данным официальной статистики и 
социологических исследований, демонстрирует тенденцию к ухудшению. Согласно 
данным Росстата, коэффициент Джини по распределению доходов составил для 
России в 2012 г. 0,42, показав заметный рост за последние 10 лет (в 2002 г. 
он составлял 0,397) и преодолев критический для экономического роста 
уровень\footnotemark[1], а децильный коэффициент фондов составил 16,4 (по 
сравнению с 14,0 в 2002 г.)\footnotemark[2].

\footnotetext[1]{По результатам обширных статистических исследований 
Мирового банка в разных странах мира было показано, что высокое неравенство -- 
выше 0,4 для коэффициента Джини -- отрицательно сказывается на экономической 
динамике и препятствует экономическому росту (см. World Bank. Equity and 
Development: World Development Report 2006. -- N. Y.: Te World Bank and Oxford 
University Press, 2006).}
\footnotetext[2]{Данные Росстата [Электронный ресурс] // Официальный сайт 
Министерства труда и социальной защиты Российской Федерации URL: 
\url{http://www.gks.ru/free_doc/new_site/population/urov/urov_32g.htm}}

Динамика данных показателей -- далеко не академический вопрос, ведь негативные 
последствия избыточных неравенств (особенно тех, которые воспринимаются 
обществом как несправедливые) многообразны и варьируются от углубления 
дифференциации самих бедных и появления среди них массового <<социального 
дна>> до утраты обществом социальной стабильности и делегитимизации власти. В 
связи с этим можно говорить о том, что избыточные социальные неравенства 
представляют собой даже более опасное по своим последствиям для общества 
явление, нежели бедность как таковая. В этих условиях особенно важно понимать, 
какими особенностями характеризуется восприятие различных типов неравенств в 
современном российском обществе самими бедными и в чем его характерные отличия 
от восприятия остального населения; какие неравенства воспринимаются ими как 
справедливые и несправедливые, каким вообще бедные россияне видят справедливое 
общество и насколько современное российское общество отвечает их 
представлениями о справедливости.

Итак, социальные неравенства имеют разнообразные проявления в российском 
обществе, при этом чаще всего в контексте самых болезненных неравенств как на 
микро, так и на макроуровне и бедными, и не бедными россиянами упоминается 
неравенство по доходам. Означает ли это, что бедные вообще не толерантны к 
неравенству по доходам и считают его априори несправедливым? Данные 
исследования показывают, что это не совсем так. Даже при явно избыточных 
неравенствах в современной России, от которых страдают, прежде всего, именно 
бедные, они, тем не менее, не являются безусловными противниками существования 
значительной разницы в доходах, если она обусловлена справедливыми, в их 
представлениях, причинами: так, с утверждением о том, что большая разница в 
доходах людей необходима, чтобы отразить разницу в их талантах и усилиях, 
согласились 35\% бедных <<по доходам>>, аналогичная доля выразила свое 
несогласие, а 30\% затруднились дать определенный ответ. Среди бедных <<по 
лишениям>> с этим согласились 31\%, не согласились 34\%, и еще 35\% не 
определились с ответом. Кроме того, 43\% бедных <<по доходам>> считают, что в 
любом обществе всегда есть и будут неравенства, это естественно и справедливо 
(при более низкой доле -- 27\% -- несогласных с этим утверждением и 30\% не 
определившихся). Среди бедных <<по лишениям>> эти доли составили 40\% 
согласных, 29\% несогласных и 31\% затруднившихся. Наличие сопоставимых по 
численности групп сторонников того и другого мнения свидетельствует о том, что 
сама группа бедных не является однородной и объединяет в себе сторонников 
разных нормативных моделей <<справедливого>> общества; кроме того, процесс 
формирования этих моделей еще продолжается (в пользу этого говорит, в 
частности, высокая доля неопределившихся со своим мнением) -- возможно, под 
воздействием конфликта нормативных представлений и реальности, о чем еще будет 
сказано ниже. В целом же утверждать, что бедные отрицают любое неравенство 
доходов как таковое, нельзя.

Что же касается таких конкретных проявлений неравенств в современном 
российском обществе как лучшее жилье, большая пенсия, доступ к качественной 
медицине и лучшему образованию, то толерантность к ним оказывается ниже, чем к 
тем неравенствам, о которых речь шла выше -- причем как среди бедных россиян, 
так и среди населения в целом. Однако при этом готовых принять как 
справедливый тот факт, что люди с большими доходами могут покупать себе лучшее 
жилье, даже среди бедных все же оказывается заметно больше, чем не готовых к 
этому (40\% и 25\% соответственно среди бедных <<по доходам>> и 41\% и 28\% 
среди бедных <<по лишениям>>). Однако в отношении справедливости большей 
пенсии у тех, кто имеет более высокую зарплату, небольшой перевес уже 
оказывается на стороне тех бедных, кто считает это скорее несправедливым 
(34\% согласных против 37\% несогласных среди <<бедных по доходам>> и 29\% 
согласных против 37\% не согласных среди бедных <<по лишениям>>).

Отдельно следует остановиться на возрастных различиях среди бедных россиян с 
точки зрения восприятия ими неравенств, существующих сегодня в российском 
обществе и оценки их справедливости. Так, среди бедных старше 50 лет 
неравенство по доступу к медицинской помощи воспринималось гораздо острее, 
чем среди более молодых бедных (как болезненное лично для себя его отметили 
65\% бедных <<по доходам>> и 58\% бедных <<по лишениям>> в возрасте выше 50 
лет). Неудивительно, что и возможность для людей с высокими доходами 
пользоваться медицинскими услугами более высокого качества бедные из старших 
возрастных когорт чаще расценивают как несправедливое -- для них за этим стоят 
реальные проблемы, с которыми они сталкиваются в повседневной жизни чаще, чем 
молодые бедные \cite{vestnik}.

Обратные тенденции были характерны для оценок остроты неравенств в доступе к 
образованию -- они чаще отмечались молодыми представителями бедных слоев (среди 
тех бедных <<по доходам>, кто был младше 30 лет, данный тип неравенств отметили 
как болезненный для себя лично 28\% -- при 17\% среди тех, кто был старше 50; 
среди бедных <<по лишениям>> эти доли составили 22\% и 16\%, соответственно). 
Тем не менее, говоря о справедливости или несправедливости возможностей людей 
с высокими доходами дать своим детям лучшее образование, молодые бедные реже 
считали подобную ситуацию несправедливой, чем бедные в возрасте старше 50 лет, 
хотя различия здесь оказывались уже меньше, чем в случае с медицинскими 
услугами. Более остро воспринимается молодыми россиянами в силу большей 
актуальности для них данного вопроса и неравенство в доступе к хорошим рабочим 
местам.

Наконец, молодые бедные также чаще страдали от неравенства в досуговых 
возможностях -- этот тип неравенства как болезненный для них лично отметили 
19\% бедных <<по доходам>> в возрасте до 30 лет и лишь 3\% бедных старше 50 лет 
(среди бедных <<по лишениям>> -- 16\% и 8\%), что свидетельствует о том, что за 
всеми остальными, более насущными для них проблемами, вопрос досуга вообще не 
воспринимается ими как значимый, и основные их потребности связаны с 
преодолением неравенств в других сферах жизни -- прежде всего, связанных со 
здоровьем. При этом разнообразный и насыщенный досуг представляет собой не 
только важную характеристику качества жизни, но и способ инвестиций в 
человеческий капитал, недоступный для бедных россиян, даже для молодых.

Таким образом, оценивая возрастные особенности восприятия неравенств бедным 
населением, можно говорить о том, что различные потребности молодых и пожилых 
бедных приводят к тому, что они по-разному оценивают те типы неравенств, от 
которых они больше всего страдают. Так, для представителей бедных из старшего 
поколения более остро, чем для молодежи, стоит вопрос неравенства по доступу 
к медицинским услугам, а молодежь более болезненно, чем пожилые, воспринимает 
неравенства по доступу к образованию, досуговым возможностям и хорошим рабочим 
местам. При этом толерантность к неравенствам как таковым и их различным типам 
ярче выражена именно среди молодежи. Однако даже среди молодых бедных со 
значительным перевесом преобладает восприятие такой ситуации, когда люди с 
высокими доходами имеют доступ к лучшем образованию и здравоохранению, как 
несправедливой. В связи с этим можно утверждать: межгенерационная динамика 
вряд ли качественно изменит картину восприятия неравенств российским 
населением \cite{vestnik}.