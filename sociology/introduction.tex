\chapter*{Введение}
\addcontentsline{toc}{chapter}{Введение}

Неравенства в обществе и проблема бедности стоит очень остро в современно 
развитом мире. Каждый день происходят события сдвигающее грань между бедностью 
и богатством, уважением и порицанием, восхвалением и унижением в сторону тех 
или других. Никто из нас не застрахован от непредвиденных обстоятельств.

О бедности и бедных говорят и пишут многие, не исключением для и является 
современная Россия. При этом кто-то представляет бедных как членов общества с 
относительно малыми доходами и состояниями. Правда, возникает вопрос -- 
как относятся люди к этой проблеме, в особенности молодёжное течение, которое 
изменчиво и является одним из движущих механизмов в становлении общества.

Социальные неравенства и проблема бедности взаимосвязаны, поэтому должны как 
причина и следствие в независимости от того, какая из этих проблем является 
главенствующей.

\textbf{Актуальность} \\
\textbf{Теоретические основы} \\
\textbf{Эмпирические основы} \\
\textbf{Объект} \\
\textbf{Предмет} \\
\textbf{Цель} \\
\textbf{Задачи} \\
\textbf{Практическая значимость} \\
\textbf{Структура} \\

\pagebreak