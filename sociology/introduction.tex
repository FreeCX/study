\section*{Введение}
\addcontentsline{toc}{section}{Введение}

Неравенства в обществе и проблема бедности стоит очень остро в современно 
развитом мире. Каждый день происходят события сдвигающее грань между бедностью 
и богатством, уважением и порицанием, восхвалением и унижением в сторону тех 
или других. Никто из нас не застрахован от непредвиденных обстоятельств.

О бедности и бедных говорят и пишут многие, не исключением для и является 
современная Россия. При этом кто-то представляет бедных как членов общества с 
относительно малыми доходами и состояниями. Правда, возникает вопрос -- 
как относятся люди к этой проблеме, в особенности молодёжное течение, которое 
изменчиво и является одним из движущих механизмов в становлении общества.

Социальные неравенства и проблема бедности взаимосвязаны, поэтому должны как 
причина и следствие в независимости от того, какая из этих проблем является 
главенствующей.

\textbf{Актуальность темы}. Проблема неравенства в обществе, в том числе и 
последующая её проблема бедности в современной России плавно превращается из 
экономической в социальную. Наиболее незащищёнными признаны инвалиды, люди 
пожилого возраста, молодёжь и женщины. Формирование жизненных ценностей у 
молодого поколения происходит на начальных этапах, поэтому важно исключить 
пагубное влияние на них.

\textbf{Цель исследования} заключается в рассмотрении проблем неравенства и 
бедности, а также анализ отношения к ним сформированный у молодого поколения.  

\textbf{Задачи} поставленные в данной работе:
\begin{itemize}
    \item[-] исследовать проблему социальной неоднородности общества, а также 
        выявить существующее неравенство
    \item[-] провести анализ бедности и неравенства
    \item[-] определить отношение молодежи к данной проблеме
\end{itemize}

\textbf{Объектом} данного исследования выступает неравенство и бедность в 
обществе. \textbf{Предметом} работы является описание и анализ отношения 
группы молодых людей к данной проблеме.

В данной работе используется методы описательного исследования поставленной 
социальной проблемы. Методом сбора информации -- опрос в виде анкетирования.
Масштаб исследования -- молодые пользователи сети интернет в возрасте от 18 до
29 лет включительно.

Данная работа состоит из введения, двух глав, заключения, списка использованной 
литературы и приложения. В приложении представлен опрос используемый в данной 
работе. Общий объём работы составляет 31 страницу.

\newpage