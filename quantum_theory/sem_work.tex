\input{../.preambles/01-semester_work}
\input{../.preambles/10-russian}
\input{../.preambles/20-math}
\input{../.preambles/30-physics}
\begin{document}
\maketitlepage{Факультет электроники и вычислительной техники}{физики}
{Квантовая теория}{}{6}{студент группы Ф-369\\Голубев~А.~В.}
{}{доцент Жуков~С.~С.}{}{}

%-------------------------------------------------------------------------------

\emph{Задача №1:} Вычислить длину волны де Бройля для тепловых 
(\( T = 300 \) K) нейтронов. Следует ли учитывать волновые свойства нейтронов 
при анализе их взаимодействия с кристаллом? Расстояние между атомами в 
кристалле принять равным 0,50 нм.

\emph{Решение:}

Рассмотрим нерелятивистский случай \( v_\text{тепл.} \ll c \).

Длина волны де Бройля выражается формулой: \( \lambda = \cfrac{h}{p} \)

Импульс теплового нейтрона можно выразить из формулы: 
\[	
	E = \frac{p^2}{2m}
\] 

где \( E = kT \), \( m \) -- масса покоя нейтрона.

Подставляя в формулу для длины волны получим:
\[
	\lambda = \frac{h}{\sqrt{2mkT}} = 
	\frac{6.63\cdot10^{-34}}
	{\sqrt{2\cdot300\cdot1.67\cdot10^{-27}\cdot1.38\cdot10^{-23}}} =
	178 \text{ пм}
\]

Воспользуемся принципом неопределённости Гейзенберга:
\[
	\Delta x \cdot \Delta p_x \geq \frac{\hbar}{2}
\]

Положим \( l \sim \Delta x \), \( p \sim \Delta p_x \). Тогда получаем:
\[
	l\cdot \sqrt{2mkT} \geq \frac{\hbar}{2} 
\]

или
\[ 
	0.5\cdot10^{-9} \cdot 
	\sqrt{2\cdot300\cdot1.67\cdot10^{-27}\cdot1.38\cdot10^{-23}} \geq
	\frac{1.06\cdot10^{-34}}{2}
\]

и сравнивая полученные величины по их порядкам
\[
	10^{-24} \geq 10^{-34}
\]

Отсюда следует, что при взаимодействии с кристаллом можно не учитывать волновые 
свойства нейтронов.

\pagebreak
%-------------------------------------------------------------------------------

\emph{Задача №2:} Частица массы \( m \) движется в одномерной прямоугольной 
потенциальной яме с бесконечно высокими стенками. Ширина ямы \( l \). Найти 
значения энергии частицы, имея в виду, что возможны лишь такие состояния, для 
которых в яме укладывается целое число дебройлевских полуволн. 

\emph{Решение:}

\emph{ ПЕРЕДЕЛАТЬ РЕШЕНИЕ ДЛЯ НЕСТАЦИОНАРНОГО СЛУЧАЯ }

Запишем однородное уравнение Шрёдингера для частицы в 
прямоугольной потенциальной яме с бесконечно высокими стенками.
\[
	-\frac{\hbar^2}{2m}\dder{\psi(x)}{x} = E\psi(x)
\] 
\[
	\dder{\psi(x)}{x}+\frac{2mE}{\hbar^2}\psi(x) = 0
\]

Обозначим производную по координате штрихами:
\[
	\psi''(x) + k^2 \psi(x) = 0
\]

Представим решение в виде функции 
\( \psi(x) = A\sin\left( kx + \alpha \right) \). Применим граничные условия:
\[
	\psi(0) = \psi(l) = 0; \quad
\]
\[
	\psi(0) = A\sin\alpha = 0 \Rightarrow \alpha = 0
\]
\[
	\psi(l) = A\sin\left( kl \right) = 0 \Rightarrow kl = \pi n, 
	\text{ где } n = 1, 2, 3, ...; \quad k = \frac{\pi n}{l}
\]

Найдём первую и вторую производные:
\[
	\psi' = A\frac{\pi n}{l}\cos\left( \frac{\pi nx}{l} \right);\quad
	\psi'' = -A\left(\frac{\pi n}{l} \right)^2 
		\sin\left( \frac{\pi nx}{l} \right)
\]

и подставляя полученные функции в уравнение Шрёдингера, получим:
\[
	\frac{2mE}{\hbar^2} = \left( \frac{\pi n}{l} \right)^2
\]

Откуда получаем значения энергии частицы:
\[
	E_n = \frac{1}{2m}\left( \frac{n\pi\hbar}{l} \right)^2
\]

\pagebreak
%-------------------------------------------------------------------------------

\emph{Задача №3:} Частица массы \( m \) падает на прямоугольный потенциальный 
барьер \( U(x) = \alpha\delta(x) \). Энергия частицы \( E \). Найти коэффициент 
прозрачности \( D \) и коэффициент прохождения \( R \) барьера.

\emph{Решение:}

\pagebreak
%-------------------------------------------------------------------------------

\emph{Задача №4:} Доказать эрмитовость следующих операторов: 
а) \( \hat{p}_x \);
б) \( x\hat{p}_x \);
в) \( \hat{p}^2_x \);
д) \( \hat{H} \);
Иметь в виду, что на бесконечности волновые функции и их производные обращаются 
в ноль.

\emph{Решение:}

\pagebreak
%-------------------------------------------------------------------------------

\emph{Задача №5:} Найти собственное значение оператора и собственные функции 
оператора \( \hat{A} = \cfrac{1}{x^n}\cfrac{d}{dx} \).

\emph{Решение:}

%-------------------------------------------------------------------------------

\end{document}
