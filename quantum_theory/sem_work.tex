\input{../.preambles/01-semester_work}
\input{../.preambles/10-russian}
\input{../.preambles/20-math}
\input{../.preambles/30-physics}
\begin{document}
\maketitlepage{Факультет электроники и вычислительной техники}{физики}
{Квантовая теория}{}{6}{студент группы Ф-369\\Голубев~А.~В.}
{}{доцент Жуков~С.~С.}{}{}

%-------------------------------------------------------------------------------

\emph{Задача №1:} Вычислить длину волны де Бройля для тепловых 
(\( T = 300 \) K) нейтронов. Следует ли учитывать волновые свойства нейтронов 
при анализе их взаимодействия с кристаллом? Расстояние между атомами в 
кристалле принять равным 0,50 нм.

\emph{Решение:}

\pagebreak
%-------------------------------------------------------------------------------

\emph{Задача №2:} Частица массы \( m \) движется в одномерной прямоугольной 
потенциальной яме с бесконечно высокими стенками. Ширина ямы \( l \). Найти 
значения энергии частицы, имея в виду, что возможны лишь такие состояния, для 
которых в яме укладывается целое число дебройлевских полуволн. 

\emph{Решение:}

\pagebreak
%-------------------------------------------------------------------------------

\emph{Задача №3:} Частица массы \( m \) падает на прямоугольный потенциальный 
барьер \( U(x) = \alpha\delta(x) \). Энергия частицы \( E \). Найти коэффициент 
прозрачности \( D \) и коэффициент прохождения \( R \) барьера.

\emph{Решение:}

\pagebreak
%-------------------------------------------------------------------------------

\emph{Задача №4:} Доказать эрмитовость следующих операторов:
\begin{itemize}
	\item \( \hat{p}_x \);
	\item \( x\hat{p}_x \);
	\item \( \hat{p}^2_x \);
	\item \( \hat{H} \);
\end{itemize}
Иметь в виду, что на бесконечности волновые функции и их производные обращаются 
в ноль.

\emph{Решение:}

\pagebreak
%-------------------------------------------------------------------------------

\emph{Задача №5:} Найти собственное значение оператора и собственные функции 
оператора \( \hat{A} = \cfrac{1}{x^n}\cfrac{d}{dx} \).
\emph{Решение:}

%-------------------------------------------------------------------------------

\end{document}
